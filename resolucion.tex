\documentclass[a4paper,10pt]{article}
\usepackage[paper=a4paper, hmargin=1.5cm, bottom=1.5cm, top=3.5cm]{geometry}
\usepackage[latin1]{inputenc}
\usepackage[T1]{fontenc}
\usepackage[spanish]{babel}
\usepackage{fancyhdr}
\usepackage{lastpage}
\usepackage{xspace}
\usepackage{xargs}
\usepackage{calc}
\usepackage{ifthen}
\usepackage{aed2-symb,aed2-itef,aed2-tad, caratula}
\usepackage{algorithmicx, algpseudocode, algorithm}
\usepackage[colorlinks=true, linkcolor=blue]{hyperref}

% jonathan: incluyendo este pkg no se ponen las sangrias automaticamente. lo puse porque sino los reps no se vean tan horribles. Saquenlos si genera probelmas
\usepackage{parskip}

% Estos comandos deben ir antes del \maketitle
\materia{Algoritmos y Estructuras de Datos II} 
\submateria{Segundo Cuatrimestre de 2016} 
\titulo{Trabajo Pr\'actico 2} 
\subtitulo{Dise\~no}
\grupo{Grupo: 4 gigas de RAM}

\integrante{Jonathan Seijo}{592/15}{jon.seijo@gmail.com} % obligatorio 
\integrante{Lucas Mauricio C\'ordoba}{094/15}{lmcordobaa@gmail.com} % obligatorio 
\integrante{Lucas Gabriel De Bortoli}{736/15}{lu\_cas\_.97@hotmail.com.ar} % obligatorio 
\integrante{Luciano Galli}{534/15}{lucianogalli@outlook.com} % obligatorio 

\pagestyle{fancy}
\thispagestyle{fancy}
\addtolength{\headheight}{1pt}
\lhead{Algoritmos y Estructuras de Datos II}
\rhead{Trabajo Pr\'actico 2}

%%%%%%%%%%%%%%%%%%%%%%%%%%%%%%%%%%%%%%%%%%%%%%%%%%%%%%%%%%%%
% COMANDOS QUE ALGUN DIA PUEDAN FORMAR UN PAQUETE.
%%%%%%%%%%%%%%%%%%%%%%%%%%%%%%%%%%%%%%%%%%%%%%%%%%%%%%%%%%%%
\newcommand{\moduloNombre}[1]{\textbf{#1}}

\let\NombreFuncion=\textsc
\let\TipoVariable=\texttt
\let\ModificadorArgumento=\textbf
\newcommand{\res}{$res$\xspace}
\newcommand{\tab}{\hspace*{7mm}}

\newcommandx{\TipoFuncion}[3]{%
  \NombreFuncion{#1}(#2) \ifx#3\empty\else $\to$ \res\,: \TipoVariable{#3}\fi%
}
\newcommand{\In}[2]{\ModificadorArgumento{in} \ensuremath{#1}\,: \TipoVariable{#2}\xspace}
\newcommand{\Out}[2]{\ModificadorArgumento{out} \ensuremath{#1}\,: \TipoVariable{#2}\xspace}
\newcommand{\Inout}[2]{\ModificadorArgumento{in/out} \ensuremath{#1}\,: \TipoVariable{#2}\xspace}
\newcommand{\Aplicar}[2]{\NombreFuncion{#1}(#2)}

\newlength{\IntFuncionLengthA}
\newlength{\IntFuncionLengthB}
\newlength{\IntFuncionLengthC}
%InterfazFuncion(nombre, argumentos, valor retorno, precondicion, postcondicion, complejidad, descripcion, aliasing)
\newcommandx{\InterfazFuncion}[9][4=true,6,7,8,9]{%
  \hangindent=\parindent
  \TipoFuncion{#1}{#2}{#3}\\%
  \textbf{Pre} $\equiv$ \{#4\}\\%
  \textbf{Post} $\equiv$ \{#5\}%
  \ifx#6\empty\else\\\textbf{Complejidad:} #6\fi%
  \ifx#7\empty\else\\\textbf{Descripci\'on:} #7\fi%
  \ifx#8\empty\else\\\textbf{Aliasing:} #8\fi%
  \ifx#9\empty\else\\\textbf{Requiere:} #9\fi%
}

\newenvironment{Interfaz}{%
  \parskip=2ex%
  \noindent\textbf{\Large Interfaz}%
  \par%
}{}

\newenvironment{Representacion}{%
  \vspace*{2ex}%
  \noindent\textbf{\Large Representaci\'on}%
  \vspace*{2ex}%
}{}

\newenvironment{Algoritmos}{%
  \vspace*{2ex}%
  \noindent\textbf{\Large Algoritmos}%
  \vspace*{2ex}%
}{}


\newcommand{\Title}[1]{
  \vspace*{1ex}\par\noindent\textbf{\large #1}\par
}

\newenvironmentx{Estructura}[2][2={estr}]{%
  \par\vspace*{2ex}%
  \TipoVariable{#1} \textbf{se representa con} \TipoVariable{#2}%
  \par\vspace*{1ex}%
}{%
  \par\vspace*{2ex}%
}%

\newboolean{EstructuraHayItems}
\newlength{\lenTupla}
\newenvironmentx{Tupla}[1][1={estr}]{%
    \settowidth{\lenTupla}{\hspace*{3mm}donde \TipoVariable{#1} es \TipoVariable{tupla}$($}%
    \addtolength{\lenTupla}{\parindent}%
    \hspace*{3mm}donde \TipoVariable{#1} es \TipoVariable{tupla}$($%
    \begin{minipage}[t]{\linewidth-\lenTupla}%
    \setboolean{EstructuraHayItems}{false}%
}{%
    $)$%
    \end{minipage}
}

\newcommandx{\tupItem}[3][1={\ }]{%
    %\hspace*{3mm}%
    \ifthenelse{\boolean{EstructuraHayItems}}{%
        ,#1%
    }{}%
    \emph{#2}: \TipoVariable{#3}%
    \setboolean{EstructuraHayItems}{true}%
}

\newcommandx{\RepFc}[3][1={estr},2={e}]{%
  \tadOperacion{Rep}{#1}{bool}{}%
  \tadAxioma{Rep($#2$)}{#3}%
}%

\newcommandx{\Rep}[3][1={estr},2={e}]{%
  \tadOperacion{Rep}{#1}{bool}{}%
  \tadAxioma{Rep($#2$)}{true \ssi #3}%
}%

\newcommandx{\Abs}[5][1={estr},3={e}]{%
  \tadOperacion{Abs}{#1/#3}{#2}{Rep($#3$)}%
  \settominwidth{\hangindent}{Abs($#3$) \igobs #4: #2 $\mid$ }%
  \addtolength{\hangindent}{\parindent}%
  Abs($#3$) \igobs #4: #2 $\mid$ #5%
}%

\newcommandx{\AbsFc}[4][1={estr},3={e}]{%
  \tadOperacion{Abs}{#1/#3}{#2}{Rep($#3$)}%
  \tadAxioma{Abs($#3$)}{#4}%
}%


\newcommand{\DRef}{\ensuremath{\rightarrow}}



\begin{document}

\maketitle

\thispagestyle{fancy}
\tableofcontents


\pagebreak

\section{Informe}

\subsection{Juego}

\begin{itemize}
\item{La operaci\'on cantMismaEspecie de la especificaci\'on recive como parametro un multiconjunto. 
Reemplazamos ese parametro por un $juego$, porque usando el $juego$ podemos obtener la cantidad de cada especie pokemon}

\end{itemize}

\subsection{Cola de Prioridad}

\begin{itemize}
\item{En la cola de entrenadores, se usand nodos y punteros para la estructura. Los nodos se mantienen ``fijos'' una vez que se agregar, hasta que son borrados. Cada vez que hay que hacer alg\'un cambio o ``swap'' lo \'unico que se modifican son los punteros ``padre'', ``izq'' y ``der''. Es decir que si hay alg\'un puntero apuntando al nodo y se realiza un swap entre ese nodo y otro, dicho puntero seguir\'a apunt\'andolo.}

\end{itemize}


\subsection{Algoritmos privados}

\begin{itemize}

\item{Se inlcuye pre y post en castellano de los algoritmos privados que hacen manejo de memoria.}

\end{itemize}


\subsection{IterDiccString}

\begin{itemize}

\item{Si bien iterDiccString se explica con IteradorUnidireccional, hicimos un cambio en la aridad de ``siguiente''. En su TAD, el tipo que se devuelve es del mismo tipo que recibe en $crear$, pero nosotros cambiamos eso para que devuelva las tuplas que al usuario le interesan (particularmente en la funcion Pokemons() del juego)}

\end{itemize}




\pagebreak

\section{Juego}
\begin{Interfaz}


\textbf{se explica con}: \tadNombre{Juego}.

\textbf{g\'eneros}: \TipoVariable{juego}.

\Title{Operaciones b\'asicas de Juego}



% GENERADORES %


\InterfazFuncion{crearJuego}{\In{m}{map}}{juego}%
{$res \igobs crearJuego(m)$}%
[$O((Tam(m))^2)$]
[Genera una juego con el mapa \texttt{m} y sin jugadores.]


\InterfazFuncion{agregarPok\'emon}{\In{p}{pokemon}, \In{c}{coord}, \Inout{j}{juego}}{}
[$j_0 \igobs j \land posExistente(c, mapa(j)) \land p \not \in pokemones(j) Y PUEDOAGREGARPOKEMON() QUE CHECKEE LAS DISTANCIAS $]  
{$j \igobs agregarPokemon(p, c, j_0)  \land p \in pokemones(j)$}
[$\Theta(¡¡completar!!)$]
[agrega pok\'emon \texttt{p} al juego \texttt{j} en la coordenada \texttt{c}.]


\InterfazFuncion{agregarJugador}{\In{j}{juego}}{nat}% REVISAR SI HAY QUE DEVOLVER NAT
[$ j_0 \igobs j $]
{$res \igobs ProxId(j_0) \land j \igobs agregarJugador(j_0)$}%
[$O(1)$]
[agrega un jugador al juego \texttt{j} con id igual a ProxId(j).]


\InterfazFuncion{conectarse}{\Inout{j}{juego}, \In{c}{coor}, \In{e}{jugador}}{}%
[$j \igobs j_0 \land e \in jugadores(j) \yluego \neg estaConectado(e, j) \land posExistente(c, mapa(j)) $]
{$estaConectado(e, j) Y cambian capturables y contadores y sarasa$}%
[$O(log(EC))$] % Es asi por enunciado
[Conecta al jugador e en la posicion c]


\InterfazFuncion{desconectarse}{\Inout{j}{juego}, \In{e}{jugador}}{}%
[$j \igobs j_0 \land e \in jugadores(j) \yluego estaConectado(e, j)$]
{$\neg estaConectado(e, j) Y cambian capturables y contadores y sarasa$}%
[$O(log(EC))$] % Es asi por enunciado
[Desconecta al jugador e del juego]


\InterfazFuncion{moverse}{\Inout{j}{juego}, \In{c}{coor}, \In{e}{jugador}}{}%
[$j \igobs j_0 \land e \in jugadores(j) \yluego estaConectado(e, j) \land posExistente(c, mapa(j)) $]
{$cambia posicion y cambian capturables y contadores y sarasa$}%
[$O((PC + PS) * |P| + EC * log(EC))$] % Es asi por enunciado
[Mueve al jugador e en la posicion c si es valido, sino sanciona o captura o sarasa]

% OBSERVADORES

%mapa : juego −→ map
\InterfazFuncion{Mapa}{\In{j}{juego}}{map}%
[true]
{$res = mapa(j)$}%
[$O(1)$] % HAGAMOS QUE LO DEVUELVA POR REFERENCIA
[Devuelve el mapa del juego]
{Es por referencia, produce aliasing}


%jugadores : juego −→ conj(jugador)
\InterfazFuncion{Jugadores}{\In{j}{juego}, \In{e}{jugador}}{ITERADOR}%
[true]
{$TENGO QUE DEVOLVER ITERADOR VER ENUNCIADO$}%
[$O(1)$] % por enunciado. "Ver que siguiente no tiene por que ser O(1)"
[Devuelve un iterador al jugador. Puede borrarse]



%estaConectado : jugador e × juego j −→ bool  {e ∈ jugadores(j)}
\InterfazFuncion{EstaConectado}{\In{j}{juego}, \In{e}{jugador}}{bool}%
[$e \in jugadores(j)$]
{$res = estaConectado(e, j)$}%
[$O(1)$] % por enunciado. "Ver que siguiente no tiene por que ser O(1)"
[Devuelve true si el jugador esta conectado]



%sanciones : jugador e × juego j −→ nat {e ∈ jugadores(j)}
\InterfazFuncion{Sanciones}{\In{j}{juego}, \In{e}{jugador}}{nat}%
[$e \in jugadores(j)$]
{$res = sanciones(e, j)$}%
[$O(1)$] % por enunciado. 
[Devuelve la cantidad de sanciones de un jugador]



%posicion : jugador e × juego j −→ coor  {e ∈ jugadores(j) ∧ L estaConectado(e, j)}
\InterfazFuncion{Posicion}{\In{j}{juego}, \In{e}{jugador}}{coor}%
[$e \in jugadores(j)  \yluego estaConectado(e, j)$]
{$res = posicion(e, j)$}%
[$O(1)$] % por enunciado. 
[Devuelve la posicion actual de un jugador]



%pokémons : jugador e × juego j −→ multiconj(pokémon) {e ∈ jugadores(j)}
%VER
\InterfazFuncion{Pokemons}{\In{j}{juego}, \In{e}{jugador}}{multiconj}%
[$e \in jugadores(j)$]
{$res = ITERADOR?? VER ENUNCIADO POCO CLARO$}%
[$O(1)$] % por enunciado. 
[FRUTA]
{TENEMOS QUE HACER EL MULTICONJUNTO? ITERADOR A TUPLA POR ENUNCIADO? VER SI ESTA EN MODS BASICOS SINO QUE SEA VECTOR Y CHAU}



%expulsados : juego −→ conj(jugador)
\InterfazFuncion{Expulsados}{\In{j}{juego}}{conj(jugador)}%
[true]
{$res = expulsados(j)$}%
[$O(J)$]  
[Devuelve un conjunto con los jugadores expulsados]


%posConPokémons : juego −→ conj(coor)
\InterfazFuncion{PosConPokemons}{\In{j}{juego}}{conj(coor)}%
[true]
{$res = posConPokemons(j)$}%
[$O(1)$] % Devuelve por referencia el conjunto, asi que habria que ir construyendolo
[Devuelve un conjunto con las posiciones del mapa que tienen pokemons. Aliasing: devuelve el conjunto por referencia]
% ACA DEVOLVEMOS LAS COORDENADAS DE LOS POKENODOS



%pokémonEnPos : coor c × juego j −→ pokémon {c ∈ posConPokémons(j)}
\InterfazFuncion{PokemonEnPos}{\In{j}{juego}, \In{c}{coor}}{pokemon}%
[$c \in posConPokemons(j)$]
{$res = pokemonEnPos(c, j)$}%
[$O(1)$] % Ibamos a guardar el pokemon en la tupla del pokenodo
[Devuelve el pokemon que se encuentra en la posicion c. Sin aliasing, devuelve por copia]


%cantMovimientosParaCaptura : coor c × juego j −→ nat {c ∈ posConPokémons(j)}
\InterfazFuncion{CantMovimientosParaCaptura}{\In{c}{coor}, \In{j}{juego}}{nat}%
[$c \in posConPokemons(j)$] %requiere que c sea pos de pokenodo
{$res = cantMovimientosParaCaptura(c, j)$}%
[$O(1)$] % Ibamos a guardar el contador en la tupla del pokenodo
[Devuelve el numero de movimientos que indican cuando se captura un pokemon]


% otras operaciones publicas

% puedoAgregarPokémon : coor c × juego j −→ bool
\InterfazFuncion{PuedoAgregarPokemon}{\In{c}{coor}, \In{j}{juego}}{bool}%
[true] 
{$res = puedoAgregarPokemon(c, j)$}  % Ver que mucho no me convence
[$O(1)$] 
[Devuelve verdadero si la coordenada es valida y no hay ningun pokemon en el territorio]


% hayPokémonCercano : coor c × juego j −→ bool
\InterfazFuncion{HayPokemonCercano}{\In{c}{coor}, \In{j}{juego}}{bool}%
[true] 
{$res = hayPokemonCercano(c, j)$}  % Ver que mucho no me convence
[$O(1)$] 
[Devuelve verdadero si hay algun pokemon en el territorio]


% posPókemonCercano : coor c × juego j −→ coor {hayPokémonCercano(c,j)}
\InterfazFuncion{PosPokemonCercano}{\In{c}{coor}, \In{j}{juego}}{coor}%
[hayPokemonCercano(c, j)] 
{$res = posPokemonCercano(c, j)$}  % Ver que mucho no me convence
[$O(1)$] 
[Devuelve la posicion del pokemon que esta en territorio]


% entrenadoresPosibles : coor c × conj(jugador) es × juego j −→ conj(jugador)
% {hayPokémonCercano(c,j) ∧ es ⊆ jugadoresConectados(j)}
\InterfazFuncion{EntrenadoresPosibles}{\In{c}{coor},\In{es}{conj(jugador)}, \In{j}{juego}}{conj(jugador)}%
[hayPokemonCercano(c, j) $\land$ es $\subset$ jugadoresConectados(j)] 
{$res = entrenadoresPosibles(c, es, j)$}  % Ver que mucho no me convence
[$O(1)$] % Se supone que vamos a guardar los posibles en un "AVL" en la coordenada del pokenodo
[Devuelve un conjunto con los entrenadores que estan en condiciones de capturar el pokemon que se encuentra en el rango de c. Que esten en condiciones de capturar significa que estan en rango2 del pokemon y que existe un camino hacia el]


% indiceRareza : pokémon p × juego j −→ nat  {p ∈ todosLosPokémons(j)}
\InterfazFuncion{IndiceRareza}{\In{p}{pokemon}, \In{j}{juego}}{nat}%
[p $\in$ todosLosPokemons(j)] 
{$res = indiceRareza(p, j)$}  % Ver que mucho no me convence
[$O(|P|)$] % es una cuenta, obtener los numeros de p nos cuesta recorrer el trie
[Devuelve el indice de rareza del pokemon dado]


% cantPokémonsTotales : juego −→ nat
\InterfazFuncion{cantPokemonsTotales}{\In{j}{juego}}{nat}%
[true] 
{$res = cantPokemonsTotales(p, j)$}  % Ver que mucho no me convence
[$O(1)$] % tiene que ser O(1) para que el indice de rareza lo calculemos rapido, 
% Podemos guardar el dato como un valor de la estructura del juego
[Devuelve la cantidad de pokemons totales del juego]


% cantMismaEspecie : pokémon × multiconj(pokémon) −→ nat
VERVERVER \\
HAY DIFERENCIA ENTRE LA ESPECIFICACION Y SU SIGNATURA \\
\InterfazFuncion{cantMismaEspecie}{\In{j}{juego}}{nat}%
[true] 
{$res = cantMismaEspecie(j)$}  % Ver que mucho no me convence
[$O(iNFINITO)$] % tiene que ser O(1) para que el indice de rareza lo calculemos rapido, 
% Podemos guardar el dato como un valor de la estructura del juego
[Devuelve la cantidad de pokemons totales del juego]


\end{Interfaz}

~ 

\begin{Representacion}

% se representa con una tupla(variable, variable, pokenodos, )


NO ES DEFINITIVO, PERO PARA IR TENIENDO LA IDEA
 
\begin{Estructura}{Juego}[pokgo]

\begin{Tupla}[pokgo]  
	\tupItem{cantPokemon}{diccTrie}  % Tendria la cantidad total de cada pokemon
	\tupItem{\\ cantPokemonsTotales}{nat} % Para acceso O(1) al total
	\tupItem{\\ map}{mapa}  % Tendria el mapa con el que se crea
	\tupItem{\\ jugadores}{vector(jugStruc)}  % Tendria los datos de los jugadores, o iteradores, VER
	\tupItem{\\ grillaJugs}{vector(vector(lista(jugStruc)))} % Una grilla R2 del tamaño del mapa donde en cada posicion hay un vector de los jugadores que estan en esa posicion
	\tupItem{\\ pokenodos}{vector(vector(puntero(pokeStruc)))}  % Una grilla R2 del tamaño del mapa donde en cada coordenada puede haber un pokenodo (ver en REP que tienen que estar a cierta distancia)
\end{Tupla}
\end{Estructura}

    \begin{Tupla}[pokeStruc]
      \tupItem{poke}{pokemon}
      \tupItem{\\ contador}{nat}%
	  \tupItem{\\ entrenadores}{HEAP(JUGADOR)} % Deberia ser como sea que se llame el modulo, cola_de_prioridadJug o pokeHeap, nose, pero en el fondo seria un heap
    \end{Tupla}
    
    
    \begin{Tupla}[jugStruc]
      \tupItem{id}{nat}
      \tupItem{\\ sanciones}{nat}%
	  \tupItem{\\ conectado}{bool}
	  \tupItem{\\ pos}{coor}
 	  \tupItem{\\ pokemons}{LISTA(TUPLA(POKEMON, CANTIDAD))} % Hay que ver como nos conviene mantener a los pokemones que capturo el jugador para que no nos arruine la complejidad    
	  \tupItem{\\ indiceEnPokenodo}{nat} % Solo me interesa el indice, un puntero es al pedo
	  \tupItem{\\ iteradorAPos}{itLista(nat)}   
	  \tupItem{\\ cantCapt}{nat}
		
    \end{Tupla}


\RepFc[]{true}

  

\AbsFc[]{}

\end{Representacion}

\begin{Algoritmos}


% crearJuego
\begin{algorithm}[H]
{\textbf{iCrearJuego}(\In{map}{mapa}) $\to$ $res$ : juego}
\begin{algorithmic}[1]

\State dictTrie $cantPokemon \gets $ Vacio()  \Comment $O(1)$
\State nat $cantPokemonsTotales \gets $ 0  \Comment $O(1)$
$ $\newline
\State vector(jugStruc)  $jugs \gets$ Vacio()     \Comment $O(1)$

% Creo la grilla de listas de posiciones (con las listas vacias)
\State vector(vector(lista(jugador)))   $grillaJugs$
$ $\newline
\For{$i \gets 0$ to Tam($map$) $-$ 1} \Comment Se repite Tam(map) veces $O(1)$
	
	\State vector(lista(jugador)) $vectorInterno \gets$ Vacio()  \Comment $O(1)$
	
	\For{$j \gets 0$ to Tam($map$) $-$ 1}  \Comment Se repite Tam(map) veces $O(1)$
		\State lista(jugador) $jugsVacia \gets$ Vacia()   \Comment $O(1)$
		\State AgregarAtras($vectorInterno$, $jugsVacia$) \Comment $O(1)$	 
	\EndFor
	
	AgregarAtras($jugs$, $vectorInterno$)   \Comment $O(1)$

\EndFor
$ $\newline
\State vector(vector(pokeStruc))  $pokenodos \gets$ Vacio()  \Comment $O(1)$

\For{$i \gets 0$ to Tam($map$) $-$ 1}		\Comment Se repite Tam(map) veces $O(1)$
	\State vector(puntero(pokeStruc)) $vectorInterno \gets$ Vacio()  \Comment $O(1)$
	\For{$j \gets 0$ to Tam($map$) $-$ 1}    \Comment Se repite Tam(map) veces $O(1)$
		\State puntero(pokeStruc) $pokePuntero \gets$ NULL   \Comment $O(1)$
		\State AgregarAtras($vectorInterno$, $pokePuntero$)	\Comment $O(1)$
	\EndFor
	AgregarAtras($pokenodos$, $vectorInterno$)   \Comment $O(1)$
\EndFor

\State $res \gets \langle cantPokemon$, $cantPokemonsTotales$, $map$, $jugs$, $grillaJugs$, $pokenodos \rangle$   \Comment $O(1)$

\medskip
\Statex \underline{Complejidad:} $O((Tam(map))^2)$
\Statex \underline{Justificaion:} Se crean 2 vectores de vectores, de Tam(map) elementos tanto el vector interno como el externo $O((Tam(map))^2) + O((Tam(map))^2) = O((Tam(map))^2)$. Crear vectores vacios cuesta O(1) y el mapa lo pasamos por referencia, $O(1) + O((Tam(map))^2) = O((Tam(map))^2)$

\end{algorithmic}
\end{algorithm}


\begin{algorithm}[H]
{\textbf{iAgregarJugador}(\In{j}{juego}) $\to$ $res$ : nat}
\begin{algorithmic}[1]

\State nat $proxId$ $\gets$ Longitud($j$.jugadores)   \Comment $O(1)$

%\State $nat sanciones \gets 0$ 
%\State $bool conectado \gets false$

% Le pongo (0,0) pero en realidad no importa porque no tenemos que chequear la posicion si esta desconectado)
\State coor $pos$ $\gets$ CrearCoor(0,0) \Comment $O(1)$

\State vector($<$pokemon, nat$>$) $pokes$ $\gets$ Vacio() \Comment $O(1)$
 
% Le pongo 0 pero en realidad no nos importa porque no hay que usarlo si no tiene pokemon cerca
\State nat $indiceEnPokenodo$ $\gets$ 0  \Comment $O(1)$

% Creo el iterador de una lista vacia para poder crearlo
\State lista(nat) $listaDummy$ $\gets$ Vacia() \Comment $O(1)$
\State itLista(nat) $iteradorAPos$ $\gets$ CrearIt($listaDummy$) \Comment $O(1)$

\State $j$.jugadores.AgregarAtras($<proxId$, 0, $false$, $pos$, $pokes$, $indiceEnPokenodo$, $iteradorAPos>$) \Comment $O(1)$

\State $res \gets proxId$ \Comment $O(1)$

\medskip
\State \underline{Complejidad:} $O(1)$
\State \underline{Justificaion:} Todas las operaciones son O(1)

\end{algorithmic}
\end{algorithm}

\begin{algorithm}[H]
{\textbf{iConectarse}(\Inout{p}{juego}, \In{c}{coordenada}, \In{j}{jugador})}
\begin{algorithmic}[1]

\State $p$.jugadores[\hspace{0.5mm}j\hspace{0.5mm}].conectado $\gets$ true \Comment $O(1)$
\State itLista(nat) $it$ $\gets$ AgregarAtras($p$.grillaJugs[latitud(c)][longitud(c)], $j$) \Comment $O(1)$
\State $p$.jugadores[j].iteradorAPos $\gets$ $it$ \Comment $O(1)$
\If {$HayPokemonCercano(c, p) \land HayCamino(c, PosPokemonCercano(c, p), Mapa(p))$} \Comment $O(1)$
	\State nat $latPok$ $\gets$ latitud(PosPokemonCercano($c$, $p$)) \Comment $O(1)$
	\State nat $lonPok$ $\gets$ longitud(PosPokemonCercano($c$, $p$)) \Comment $O(1)$
	\State $p$.pokenodos[$latPok$][$lonPok$].contador $\gets$ 0 \Comment $O(1)$
	\State nat $i$ $\gets$ Longitud($p$.pokenodos[$latPok$][$lonPok$] $\to$ entrenadores) \Comment $O(1)$
	\State AgregarAtras($p$.pokenodos[$latPok$][$lonPok$] $\to$ entrenadores, $j$) \Comment $O(1)$ Amortizado
	\While{$(i>0) \land \newline
	(($p$.pokenodos[$latPok$][$lonPok$] \to entrenadores[i].cantCapt < \newline 
	$p$.pokenodos[$latPok$][$lonPok$] \to entrenadores[(i-1)/2].cantCapt) \vee \newline
	(($p$.pokenodos[$latPok$][$lonPok$] \to entrenadores[i].cantCapt = \newline 
	$p$.pokenodos[$latPok$][$lonPok$] \to entrenadores[(i-1)/2].cantCapt) \Rightarrow \newline
	($p$.pokenodos[$latPok$][$lonPok$] \to entrenadores[i].id < \newline
	$p$.pokenodos[$latPok$][$lonPok$] \to entrenadores[(i-1)/2].id)))$} \Comment $O(log(EC))$
		\State jugStruc $aux$ $\gets$ ($p$.pokenodos[$latPok$][$lonPok$] $\to$ entrenadores[i]) \Comment $O(1)$
		\State ($p$.pokenodos[$latPok$][$lonPok$] $\to$ entrenadores[i]) $\gets$ \newline
		($p$.pokenodos[$latPok$][$lonPok$] $\to$ entrenadores[(i-1)/2]) \Comment $O(1)$
		\State ($p$.pokenodos[$latPok$][$lonPok$] $\to$ entrenadores[(i-1)/2]) $\gets$ $aux$ \Comment $O(1)$
		\State ($p$.jugadores[$p$.pokenodos[$latPok$][$lonPok$] $\to$ entrenadores[i].id]) $\gets$ $i$ \Comment $O(1)$
		\State $i$ $\gets$ (i-1)/2 \Comment $O(1)$
	\EndWhile
	\State $p$.jugadores[j].indiceEnPokenodo $\gets$ $i$ \Comment $O(1)$
\EndIf

\medskip
\State \underline{Complejidad:} $O(log(EC))$ Amortizado por uso de vector
\State \underline{Justificaion:} Todas las operaciones de asignaci\'on, acceso a posiciones de vectores y desreferenciaci\'on de punteros son O(1). Las funciones "HayPokemonCercano", "HayCamino", "PosPokemonCercano", "Mapa"\hspace{0.5mm}son O(1). La funci\'on AgregarAtras de lista enlazada es O(1) y AgregarAtras de vector es O(1) amortizado. Suma y divisi\'on entera de nats es O(1).
En el peor caso, el ciclo realiza O(log(EC)) iteraciones: \newline
Sea $n$ la cantidad de elementos del vector de entrenadores (en el peor caso n ser\'a EC), $i$ inicia el ciclo con el valor $n-1$. En el peor de los casos tiene que llegar a 0 para alcanzar la cota y terminar el ciclo. 
Por invariante del ciclo, despu\'es de la k-\'esima iteraci\'on \newline
\newline
$i = \tfrac{n-1}{2^{k}} - \sum_{j=0}^{k-1}\tfrac{1}{2^{k-j}}= \tfrac{n-1}{2^{k}} - \tfrac{2^{k}-1}{2^{k}} = \tfrac{n-2^{k}}{2^{k}}$ \newline
\newline
Calculando el valor de $k$ para que $i$ = 0 y se deje de cumplir la guarda se tiene que: \newline
0 = $\tfrac{n-2^{k}}{2^{k}}$ \newline
0 = $n-2^{k}$ \newline
$2^{k}$ = $n$ \newline
$k$ = $\log_{2}(n)$ \newline
Por lo tanto la cantidad de iteraciones que se necesitan para terminar el ciclo en el peor de los casos es $\log_{2}(EC)$ \newline
\newline
Por \'algebra de \'ordenes, sumando los costos de cada una de las operaciones, queda que la complejidad del algoritmo es $O(log(EC))$ amortizado.


\end{algorithmic}
\end{algorithm}


% cantPokemonsTotales

\begin{algorithm}[H]
{\textbf{iCantPokemonTotales}(\In{j}{juego}) $\to$ $res$ : nat}
\begin{algorithmic}[1]

\State $res \gets$ $j$.cantPokemonsTotales \Comment $O(1)$

\medskip
\State \underline{Complejidad:} $O(1)$

\end{algorithmic}
\end{algorithm}

% indiceRareza

\begin{algorithm}[H]
{\textbf{iIndiceRareza}(\In{p}{pokemon}, \In{j}{juego}) $\to$ $res$ : nat}
\begin{algorithmic}[1]

\State nat $pokecant \gets$ Obtener(j.cantPokemon, $p$) \Comment $O(|P|)$

\State $res \gets$ 100 $-$ (100 $*$ $pokecant$ $/$ j.cantPokemonsTotales) \Comment $O(1)$

\medskip
\State \underline{Complejidad:} $O(|P|)$
\State \underline{Justificacion:} j.cantPokemon es un diccionario sobre trie. La complejidad de buscar (y obtener el significado) en peor caso es la longitud de la string mas larga entre sus claves, eso es O($|P|$). j.cantPokemonsTotales es un dato guardado en la estructura del juego, y se accede en O(1). El resto son una resta, multiplicacion y division, que tambien son O(1). O($|P|$) $+$ O(1) $=$ O($|P|$)

\end{algorithmic}
\end{algorithm}


% entrenadoresPosibles
% por precondicion estan conectados

\begin{algorithm}[H]
{\textbf{iEntrenadoresPosibles}(\In{c}{coor}, \In{ec}{conj(jugador)}, \In{j}{juego}) $\to$ $res$ : conj(jugador)}
\begin{algorithmic}[1]

\State coor $pokeCoor \gets$ PosPokemonCercano($c$, $j$) \Comment $O(1)$
\State puntero(pokeStruc) $pokePuntero \gets j.pokeNodos[longitud(pokeCoor)][latitud(pokeCoor)]$ \Comment $O(1)$

\State itConj(jugador) $itPosibles \gets$ CrearIt($ec$) \Comment $O(1)$

\While{HaySiguiente($itPosibels$)}   \Comment O($Cardinal(ec) * O(HaySiguiente) * Cardinal(jugsEnNodo)$)
	
	% Asumo que tengo una forma de recorrer el heap
	\State itHeap $itJugsEnNodo \gets$ CrearIt(($*pokePuntero$).entrenadores)	  \Comment Depende del HEAP $O(Inifinito)$
	
	\While{HaySiguiente($itJugsEnNodo$)}   \Comment O($HaySiguiente * \#jugsEnNodo$)

		\If{Siguiente($itPosibles$) = Siguiente($itJugsEnNodo$)}   \Comment Depende del HEAP $O(Inifinito)$
		
			\State Agregar($res$, Siguiente($itPosible$))
	
		\EndIf
		
		\State Avanzar($itJugsEnNodo$))
	
	\EndWhile
	
	\State Avanzar($itPosible$)   \Comment()



\EndWhile


\medskip
\State \underline{Complejidad:} $O(|P|)$
\State \underline{Justificacion:} 

\end{algorithmic}
\end{algorithm}


$ $\newline
$ $\newline
$ $\newline
% hayPokemonCercano
%\begin{algorithm}[H]
{\textbf{iHayPokemonCercano}(\In{c}{coor}, \In{j}{juego}) $\to$ $res$ : coor}
\begin{algorithmic}[1]

\State nat $x \gets$ latitud(c)    \Comment $O(1)$
\State nat $y \gets$ longitud(c)   \Comment $O(1)$

% j.pokenodos[x][y] != NULL   =>  pos con pokemon

\State bool $hayPokemon \gets$ false    \Comment $O(1)$

\If{j.pokenodos[$x$][$y$] $\neq$ NULL}     \Comment $O(1)$
    \State $hayPokemon \gets$ true
\EndIf

\If{$x > 0$}
    \If{j.pokenodos[$x-1$][$y$] $\neq$ NULL}     \Comment $O(1)$
        \State $hayPokemon \gets$ true
    \EndIf

    \If{$y > 0$}

        \If{j.pokenodos[$x-1$][$y-1$] $\neq$ NULL}     \Comment $O(1)$
            \State $hayPokemon \gets$ true
        \EndIf

    \EndIf

    % tam(m) no puede ser 0 porque existe una coordenada al menos (x,y)
    \If{$y < tam(m) - 1$}
        \If{j.pokenodos[$x-1$][$y+1$] $\neq$ NULL}     \Comment $O(1)$
            \State $hayPokemon \gets$ true
        \EndIf
    \EndIf

    \If{$x-1 > 0$}
        \If{j.pokenodos[$x-2$][$y$] $\neq$ NULL}     \Comment $O(1)$
            \State $hayPokemon \gets$ true
        \EndIf
    \EndIf

\EndIf

\If{$y > 0$}

    \If{j.pokenodos[$x$][$y-1$] $\neq$ NULL}     \Comment $O(1)$
        \State $hayPokemon \gets$ true
    \EndIf

    \If{$y-1 > 0$}

        \If{j.pokenodos[$x$][$y-2$] $\neq$ NULL}     \Comment $O(1)$
            \State $hayPokemon \gets$ true
        \EndIf

    \EndIf

\EndIf

\If{$y < tam(m) - 1$}

    \If{j.pokenodos[$x$][$y+1$] $\neq$ NULL}     \Comment $O(1)$
        \State $hayPokemon \gets$ true
    \EndIf


    \If{$tam(m) > 1 \land y < tam(m) - 2$}
        \If{j.pokenodos[$x$][$y+2$] $\neq$ NULL}     \Comment $O(1)$
            \State $hayPokemon \gets$ true
        \EndIf


    \EndIf

\EndIf

\If{$x < tam(m) - 1$}
    \If{j.pokenodos[$x+1$][$y$] $\neq$ NULL}     \Comment $O(1)$
        \State $hayPokemon \gets$ true
        \EndIf

    \If{$y > 0$}
        \If{j.pokenodos[$x+1$][$y-1$] $\neq$ NULL}     \Comment $O(1)$
            \State $hayPokemon \gets$ true
        \EndIf
    \EndIf

    \If{$y < tam(m)-1$}
        \If{j.pokenodos[$x+1$][$y+1$] $\neq$ NULL}     \Comment $O(1)$
            \State $hayPokemon \gets$ true
        \EndIf

    \EndIf
\EndIf

\If{$tam(m) > 1 \land x < tam(m) - 2$}
    \If{j.pokenodos[$x+2$][$y$] $\neq$ NULL}     \Comment $O(1)$
        \State $hayPokemon \gets$ true
    \EndIf
\EndIf

\State $res \gets hayPokemon$

\medskip
\State \underline{Complejidad:} $O(1)$ % por enunciado
\State \underline{Justificacion:} Reviso 13 posiciones O(1)

\end{algorithmic}
%\end{algorithm}

$ $\newline
$ $\newline
$ $\newline


% posPokemonCercano

%\begin{algorithm}[H]
{\textbf{iPosPokemonCercano}(\In{c}{coor}, \In{j}{juego}) $\to$ $res$ : coor}
\begin{algorithmic}[1]

\State nat $x \gets$ latitud(c)    \Comment $O(1)$
\State nat $y \gets$ longitud(c)   \Comment $O(1)$

% j.pokenodos[x][y] != NULL   =>  pos con pokemon

\State coor $coorConPokemon$    \Comment $O(1)$

\If{j.pokenodos[$x$][$y$] $\neq$ NULL}     \Comment $O(1)$
    \State $coorConPokemon \gets CrearCoor(x, y)$    \Comment $O(1)$
\EndIf

\If{$x > 0$}
    \If{j.pokenodos[$x-1$][$y$] $\neq$ NULL}     \Comment $O(1)$
        \State $coorConPokemon \gets CrearCoor(x, y)$    \Comment $O(1)$
    \EndIf

    \If{$y > 0$}

        \If{j.pokenodos[$x-1$][$y-1$] $\neq$ NULL}     \Comment $O(1)$
            \State $coorConPokemon \gets CrearCoor(x, y)$    \Comment $O(1)$
        \EndIf

    \EndIf

    % tam(m) no puede ser 0 porque existe una coordenada al menos (x,y)
    \If{$y < tam(m) - 1$}
        \If{j.pokenodos[$x-1$][$y+1$] $\neq$ NULL}     \Comment $O(1)$
            \State $coorConPokemon \gets CrearCoor(x, y)$    \Comment $O(1)$
        \EndIf
    \EndIf

    \If{$x-1 > 0$}
        \If{j.pokenodos[$x-2$][$y$] $\neq$ NULL}     \Comment $O(1)$
            \State $coorConPokemon \gets CrearCoor(x, y)$    \Comment $O(1)$
        \EndIf
    \EndIf

\EndIf

\If{$y > 0$}

    \If{j.pokenodos[$x$][$y-1$] $\neq$ NULL}     \Comment $O(1)$
        \State $coorConPokemon \gets CrearCoor(x, y)$    \Comment $O(1)$
    \EndIf

    \If{$y-1 > 0$}

        \If{j.pokenodos[$x$][$y-2$] $\neq$ NULL}     \Comment $O(1)$
            \State $coorConPokemon \gets CrearCoor(x, y)$    \Comment $O(1)$
        \EndIf

    \EndIf

\EndIf

\If{$y < tam(m) - 1$}

    \If{j.pokenodos[$x$][$y+1$] $\neq$ NULL}     \Comment $O(1)$
        \State $coorConPokemon \gets CrearCoor(x, y)$    \Comment $O(1)$
    \EndIf


    \If{$tam(m) > 1 \land y < tam(m) - 2$}
        \If{j.pokenodos[$x$][$y+2$] $\neq$ NULL}     \Comment $O(1)$
            \State $coorConPokemon \gets CrearCoor(x, y)$    \Comment $O(1)$
        \EndIf


    \EndIf

\EndIf

\If{$x < tam(m) - 1$}
    \If{j.pokenodos[$x+1$][$y$] $\neq$ NULL}     \Comment $O(1)$
        \State $coorConPokemon \gets CrearCoor(x, y)$    \Comment $O(1)$
        \EndIf

    \If{$y > 0$}
        \If{j.pokenodos[$x+1$][$y-1$] $\neq$ NULL}     \Comment $O(1)$
            \State $coorConPokemon \gets CrearCoor(x, y)$    \Comment $O(1)$
        \EndIf
    \EndIf

    \If{$y < tam(m)-1$}
        \If{j.pokenodos[$x+1$][$y+1$] $\neq$ NULL}     \Comment $O(1)$
            \State $coorConPokemon \gets CrearCoor(x, y)$    \Comment $O(1)$
        \EndIf

    \EndIf
\EndIf

\If{$tam(m) > 1 \land x < tam(m) - 2$}
    \If{j.pokenodos[$x+2$][$y$] $\neq$ NULL}     \Comment $O(1)$
        \State $coorConPokemon \gets CrearCoor(x, y)$    \Comment $O(1)$
    \EndIf
\EndIf

\State $res \gets coorConPokemon$

\medskip
\State \underline{Complejidad:} $O(1)$ % por enunciado
\State \underline{Justificacion:} Reviso 13 posiciones O(1)

\end{algorithmic}
%\end{algorithm}



% puedoAgregarPokemon

\begin{algorithm}[H]
{\textbf{iPuedoAgregarPokemon}(\In{c}{coor}, \In{j}{juego}) $\to$ $res$ : bool}
\begin{algorithmic}[1]

\State bool $puedo \gets$ false  \Comment $O(1)$

\If{PosExistente(c, j.mapa)}   \Comment $O(1)$
	\If{$\neg$ HayPokemonCercano(c, j)}   \Comment $O(1)$
		\State $puedo \gets$ true        \Comment $O(1)$
	\EndIf
\EndIf

\State $res \gets puedo$  \Comment $O(1)$

\medskip
\State \underline{Complejidad:} $O(1)$
\State \underline{Justificacion:} Todas las operaciones son O(1)

\end{algorithmic}
\end{algorithm}


% cantMovimientosParaCaptura

\begin{algorithm}[H]
{\textbf{iCantMovimientosParaCaptura}(\In{c}{coor}, \In{j}{juego}) $\to$ $res$ : 	nat}
\begin{algorithmic}[1]

\State puntero(pokeStruc) $pokenodo \gets$ j.pokenodos[latitud($c$)][longitud($c$)] \Comment $O(1)$
\State $res \gets$ (*pokenodo).contador  \Comment $O(1)$

\medskip
\State \underline{Complejidad:} $O(1)$
\State \underline{Justificacion:} Todas las operaciones son O(1)

\end{algorithmic}
\end{algorithm}


% expulsados

\begin{algorithm}[H]
{\textbf{iExpulsados}(\In{j}{juego}) $\to$ $res$ : conj(jugador)}
\begin{algorithmic}[1]

\For{$nat i \gets$ 0 to Longitud(j.jugadores) $-$ 1}   \Comment $O(J)$
	\If{j.jugadores[$i$].sanciones $geq$ 5}   \Comment $O(1)$
		AgregarRapido(res, j.jugadores[$i$].id)  \Comment $O(1)$		
	\EndIf
\EndFor

\medskip
\State \underline{Complejidad:} $O(J)$
\State \underline{Justificacion:} Aplico operaciones que son O(1) la cantidad de veces que ejecuto el ciclo. El ciclo se ejecuta J veces (porque j.jugadores tiene todos los jugadores que fueron agregados) Entonces es O(J), siendo J la cantidad de jugadores que fueron agregados.

\end{algorithmic}
\end{algorithm}


% pokemons

VER TODO EL TEMILLA DEL MULTICONJUNTO... HAY QUE HACERLO?? 

\begin{algorithm}[H]
{\textbf{iPokemons}(\In{e}{jugador}, \In{j}{juego}) $\to$ $res$ : multiconj(jugador)}
\begin{algorithmic}[1]

\State $res \gets $ CrearIt(j.jugadores[e].pokemons) \Comment $O(1)$

\medskip
\State \underline{Complejidad:} $O(1)$
\State \underline{Justificacion:} Devuelvo un iterador

\end{algorithmic}
\end{algorithm}

  
\end{Algoritmos}

\pagebreak

\section{Mapa}
\begin{Interfaz}


\textbf{se explica con}: \tadNombre{Mapa}.

\textbf{g\'eneros}: \TipoVariable{map}.

\Title{Operaciones b\'asicas de Mapa}


\InterfazFuncion{crearMapa}{}{map}%
{$res \igobs crearMapa$}%
[$O(1)$]
[genera una mapa vac\'io.]


\InterfazFuncion{agregarCoor}{ \In{c}{coord}, \Inout{m}{map}}{}
[$m_0 \igobs m \land \lnot posExistente(c, m_0) $]  
{$m \igobs agregarCoor(c, m_0)$}
[$O(  max(latitud(c), longitud(c))^4  )$]
[agrega la coordenada \texttt{c} al mapa \texttt{m}]


\InterfazFuncion{coordenadas}{\In{m}{map}}{$conj(coord)$}%
{$res \igobs coordenadas(m)$}%
[$O(tam(m))$]  % Algoritmo: recorro la diagonal principal
[devuelve el conjunto de todas las coordenadas de m.]


\InterfazFuncion{posExistente}{ \In{c}{coord}, \In{m}{map}}{bool}
{$res \igobs posExistente(c, m)$}
[$O(1)$]
[verifica si la coordenada \texttt{c} existe en el mapa \texttt{m}]


\InterfazFuncion{hayCamino}{ \In{c_1}{coord}, \In{c_2}{coord}, \In{m}{map}}{bool}
[$posExistente(c_1, m) \land posExistente(c_2, m) $]
{$res \igobs hayCamino(c_1, c_2, m)$}
[$O(1)$]
[verifica si existe una forma de llegar desde \texttt{$c_1$} a \texttt{$c_2$}]


\end{Interfaz}

\begin{Representacion}

\begin{Estructura}{mapa}[map]
	\begin{Tupla}[map]  
		\tupItem{tam}{nat}
		\tupItem{grilla}{vector(vector(vector(vector(bool)))}  
  	\end{Tupla}
  	
\end{Estructura}

\RepFc[]{true}



  

  \AbsFc[]{}

\end{Representacion}

\begin{Algoritmos}


% crearMapa
\begin{algorithm}[H]
{\textbf{iCrearMapa}()$\to$ $res$ : map}
\begin{algorithmic}[1]
   
  							% Como digo que aca va un vector(vector(v.... vacio? 	
\State $res \gets  \langle 0 , Vacio() \rangle $     COMO SABER QUE VACIO ES UN VECTOR?  \Comment Crear vector vacio es O(1) // $O(1)$

\medskip
\Statex \underline{Complejidad:} $O(1)$

\end{algorithmic}
\end{algorithm}

% coordenadas
\begin{algorithm}[H]
{\textbf{iCoordenadas}(\In{m}{map})$\to$ $res$ : conj(coor)}
\begin{algorithmic}[1]
   
 
\State $res \gets Vacio() $     COMO SABER QUE VACIO ES UN CTO?  \Comment Crear conjunto vacio es O(1) // $O(1)$

\For{$i=0$ to $tam(m)$}
	
	\For{$j=0$ to $tam(m)$}
		
		\If{iPosExistente(i, j)}  \Comment $O(1)$
			
			\State $c \gets NuevaCoor(i, j) $  \Comment $O(1)$			
			\State $Agregar(res, c) $  \Comment $O(log(\#res))$			
			
		\EndIf
		
	\EndFor

\EndFor


\medskip
\Statex \underline{Complejidad:} $O(log(tam(m))*(tam(m))^2)$

\Statex \underline{Justificacion:} Son dos for anidados, por cada iteracion externa, se ejecutan $tam(m)$ iteraciones internas. En total son $(tam(m))^2$ iteraciones. En cada una de ellas, puede ejecutarse el $Agregar(rec, c)$ con complejidad $O(log(\#res))$. $res$ solo contiene coordenadas recorridas por la iteracion, por lo que $\#res <= tam(m)$, es decir $O(log(\#res)) = O(log(tam(m)))$, por lo que la complejidad final es $O(log(tam(m))*(tam(m))^2)$.

\end{algorithmic}
\end{algorithm}

% posExistente

% EXPLICACION:
% CUANDO AGREGO UNA COORDENADA, LA CONECTO EN LA GRILLA CONSIGO MISMA, SIGNIFICANDO QUE EXISTE 
\begin{algorithm}[H]
{\textbf{iPosExistente}(\In{m}{map}, \In{c}{coor})$\to$ $res$ : bool}
\begin{algorithmic}[1]
   

\If{latitud(c) $>=$ tam(m) OR longitud(c) $>=$ tam(m)}
	\State $res \gets false$ \Comment $O(1)$
\Else
	\State $x \gets latitud(c)$ \Comment $O(1)$
	\State $y \gets longitud(c)$ \Comment $O(1)$
	\State $res \gets m.grilla[x][y][x][y]$ \Comment $O(1)$
\EndIf


\medskip
\Statex \underline{Complejidad:} $O(1)$


\end{algorithmic}
\end{algorithm}

% hayCamino

\begin{algorithm}[H]
{\textbf{iHayCamino}(\In{m}{map}, \In {c1}{coor}, \In{c2}{coor})$\to$ $res$ : bool}
\begin{algorithmic}[1]

\State $res \gets grills[latitud(c1)][longitud(c1)][latitud(c2)][longitud(c1)]$ \Comment $O(1)$

\medskip
\Statex \underline{Complejidad:} $O(1)$


\end{algorithmic}
\end{algorithm}


% agregaCoor


\begin{algorithm}[H]
{\textbf{iAgregaCoor}(\Inout{m}{map}, \In {c}{coor})}
\begin{algorithmic}[1]

\State $maximo \gets max(latitud(c), longitud(c)) $ \Comment $O(1)$

% Si hay una coordenada mayor al maximo
\State \If{maximo $>$ m.tam} \Comment $O(1)$

\State $nGrilla = iCrearGrilla(maximo)$ \Comment $O(maximo^4)$
\State $iCopiarCoordenadas(nGrilla, m.grilla)$ \Comment $O((maximo^4)$
\State $m.grilla = nGrilla$ \Comment $O(maximo^4)$
\State $m.tam = maximo$ \Comment $O(1)$

\EndIf

% La marco consigo misma para decir que la cooordenada existe
\State $m.grilla[latitud(c)][longitud(c)][latitud(c)][longitud(c)] = true$ \Comment $O(1)$
\State $ $

% Creo una grilla de visitadas, todas false
%Vector vacio ver como escribirlo bien
\State $visitados \gets Vacio() $ \Comment $O(1)$
\For{$i = 0$ to $m.tam$}	    \Comment $O(infinito)$
\State $visitadosAux \gets Vacio() $ \Comment $O(1)$	
\For{$j = 0$ to $m.tam$}      \Comment $O(Inifinito)$
\State $visitadosAux.AgregarAtras(false)$ \Comment $O(1)$
\EndFor
\State $visitados.AgregarAtras(visitadosAux)$ \Comment $O(1)$
\EndFor


% aRecorrer es una cola..
\State $aRecorrer \gets Vacio() $ \Comment $O(1)$
\State $aRecorrer.Encolar(c)$ \Comment $O(1)$

\While{$\neg aRecorrer.EsVacia()$}



\EndWhile

\medskip
\Statex \underline{Complejidad:} $O((max(maximo, m.tam))^4)$
\Statex \underline{Justificacion:} Trabajo con matrices de 4 dimensiones, en peor caso hay que redimensionar, en donde se recorren todos los elementos


\end{algorithmic}
\end{algorithm}




  
\end{Algoritmos}

\pagebreak

\section{Coordenada}
\begin{Interfaz}


  \textbf{se explica con}: \tadNombre{Coordenada}.

  \textbf{g\'eneros}: \TipoVariable{coor}.

  \Title{Operaciones b\'asicas de Coordenada}


  \InterfazFuncion{crearCoor}{\In{n_1}{nat}, \In{n_2}{nat}}{coor}%
  {$res \igobs crearCoor$}%
  [$O(1)$]
  [genera una coordenada nueva.]


  \InterfazFuncion{latitud}{\In{c}{coor}}{nat}
  {$res \igobs latitud(c)$}
  [$O(1)$]
  [devuelve la latitud de la coordenada \texttt{c}.]

  \InterfazFuncion{longitud}{\In{c}{coor}}{nat}
  {$res \igobs longitud(c)$}
  [$O(1)$]
  [devuelve la longitud de la coordenada \texttt{c}.]


  \InterfazFuncion{distEuclidea}{\In{c_1}{coor}, \In{c_2}{coor}}{nat}%
  {$res \igobs distEuclidea(c_1, c_2)$}%
  [$O(1)$]
  [devuelve la disntancia entre la coordenadas \texttt{$c_1$} y \texttt{$c_2$}.]


\end{Interfaz}

\begin{Representacion}
 
  \begin{Estructura}{coor}[estr]
      \begin{Tupla}[estr]%
      \tupItem{latitud}{nat}%
      \tupItem{longitud}{nat}%
    \end{Tupla}
  \end{Estructura}

  \RepFc[estr$/e$]{true}

  ~

  \AbsFc[estr]{coor}[e]{c : coor / e.latitud = latitud(c) $\land$ e.longitud = longitud(c)}

\end{Representacion}

\begin{Algoritmos}
   
\begin{algorithm}[H]{\textbf{crearCoor}(\In{n_1}{nat}, \In{n_2}{nat}) $\to$ $res$ : coor}
    	\begin{algorithmic}[1]
			 \State $res$ $\gets$  $\langle$ $n_1$ , $n_2$ $\rangle$ \Comment $O(1)$

			\medskip
			\Statex \underline{Complejidad:} $O(1)$
    	\end{algorithmic}
\end{algorithm}
   
\begin{algorithm}[H]{\textbf{latitud}(\In{c}{coor}) $\to$ $res$ : nat}
    	\begin{algorithmic}[1]
			 \State $res$ $\gets$  $c$.$latitud$ \Comment $O(1)$

			\medskip
			\Statex \underline{Complejidad:} $O(1)$
			
    	\end{algorithmic}
\end{algorithm}


\begin{algorithm}[H]{\textbf{longitud}(\In{c}{coor}) $\to$ $res$ : nat}
    	\begin{algorithmic}[1]
			 \State $res$ $\gets$  $c$.$longitud$ \Comment $O(1)$

			\medskip
			\Statex \underline{Complejidad:} $O(1)$
    	\end{algorithmic}
\end{algorithm}

\begin{algorithm}[H]{\textbf{distEuclidea}(\In{c_1}{coor}, \In{c_2}{coor}) $\to$ $res$ : nat}
    	\begin{algorithmic}[1]
    		 \State $a$ $\gets$ $0$ \Comment $O(1)$
    		 \If{ $c_1$.$latitud$ $<$ $c_2$.$latitud$ }	\Comment $O(1)$
				\State $a$ $\gets$ ($c_1$.$latitud$ $-$ $c_2$.$latitud$) $\times$ ($c_1$.$latitud$ $-$ $c_2$.$latitud$) \Comment $O(1)$	
			\Else 
				\State $a$ $\gets$ ($c_2$.$latitud$ $-$ $c_1$.$latitud$) $\times$ ($c_2$.$latitud$ $-$ $c_1$.$latitud$) \Comment $O(1)$	
			\EndIf
			\State $b$ $\gets$ $0$ \Comment $O(1)$

			\If{ $c_1$.$longitud$ $<$ $c_2$.$longitud$ }	\Comment $O(1)$
				\State $b$ $\gets$ ($c_1$.$longitud$ $-$ $c_2$.$longitud$) $\times$ ($c_1$.$longitud$ $-$ $c_2$.$longitud$)  \Comment $O(1)$	
			\Else 
				\State $b$ $\gets$ ($c_2$.$longitud$ $-$ $c_1$.$longitud$) $\times$ ($c_2$.$longitud$ $-$ $c_1$.$longitud$) \Comment $O(1)$	
			\EndIf
			\State $res$ $\gets$  $a$ $+$ $b$ \Comment $O(1)$

			\medskip
			\Statex \underline{Complejidad:} $O(1)$
			\Statex \underline{Justificacion:} $O(1)$ $+$ $O(1)$ $+$ $O(1)$ $+$ $O(1)$ $+$ $O(1)$ $+$ $O(1)$ $+$ $O(1)$
    	\end{algorithmic}
\end{algorithm}

\end{Algoritmos}

\pagebreak

\section{DiccString($\alpha$)}

\begin{Interfaz}
\subsection{Interfaz}

  \textbf{par\'ametros formales}\hangindent=2\parindent\\
  \parbox{1.7cm}{\textbf{g\'eneros}} $\alpha$
 
 
  \textbf{se explica con}: \tadNombre{dicc(string, $\alpha$)}, \tadNombre{conj(string)}.

  \textbf{g\'eneros}: \TipoVariable{diccString($\alpha$)}.

  \InterfazFuncion{Vacio}{}{diccString($\alpha$)}
  {$res$ $\igobs$ vac\'io}
  [$O(1)$]
  [genera un diccionario vac\'io.]

  \InterfazFuncion{Definir}{\Inout{d}{diccString($\alpha$)}, \In{s}{string}, \In{a}{$\alpha$}}{}
  [$d$ $\igobs$ $d_0$]
  {$d$ $\igobs$ definir($p$, $a$, $d$)}
  [$O(|P|)$ siendo P la clave mas larga.]
  [define $a$ en $d$ con la clave $s$.]
  [el elemento $a$ se define por referencia.]

  \InterfazFuncion{Def?}{\In{p}{string}, \In{d}{diccString($\alpha$)}}{bool}
  {$res$ $\igobs$ def?($p$, $d$)}
  [$O(|P|)$ siendo P la clave mas larga.]
  [devuelve \texttt{true} si y s\'olo si la $p$ tiene una definicion en $d$.]

  \InterfazFuncion{Obtener}{\In{p}{string}, \In{d}{diccString($\alpha$)}}{$\alpha$}
  [def?($s$, $d$)]
  {alias($res$ $\igobs$ obtener($p$, $d$))}
  [$O(|P|)$ siendo P la clave mas larga.]
  [devuelve el significado de la la clave $p$ en $d$.]
  [$res$ es modificable si y s\'olo si $d$ es modificable.]

  \InterfazFuncion{Borrar}{\In{p}{string}, \Inout{d}{diccString($\alpha$)}}{}
  [$d$ $\igobs$ $d_0$ $\land$ def?($s$, $d$)]
  {$d$ $\igobs$ borrar($p$, $d_0$)}
  [$O(|P|)$ siendo P la clave mas larga.]
  [borra la clave $p$ y su significado.]

  \InterfazFuncion{Claves}{\In{d}{diccString($\alpha$)}}{conj(string)}
  [true]
  {alias($res$ $\igobs$ claves($d$))}
  [$O(1)$]
  [Devuelve del conjunto de claves de $d$.]
  {Aliasing: $res$ se modifica sii claves($d$) se modifica}



\end{Interfaz}

\pagebreak

\begin{Representacion}
\subsection{Representacion}

  \Title{Representaci\'on del diccString}  
  
	En este m\'odulo usamos un trie para definir un diccionario cuyas claves son string.
$ $\newline La idea es que la complejidad de definir y obtener no dependa de la cantidad de claves, si no de la longitud de la clave. Bajando asi la complejidad en peor caso.  
  
  \begin{Estructura}{diccString($\alpha$)}[estr]
    \begin{Tupla}[estr]
      \tupItem{raiz}{puntero(nodo)}%
      \tupItem{claves}{conj(string)}%
    \end{Tupla}

    \begin{Tupla}[nodo]
      \tupItem{definicion}{puntero($\alpha$)}%
      \tupItem{siguientes}{arreglo[256](puntero(nodo))}%
      \tupItem{itClave}{puntero(itConj(string))}%
    \end{Tupla}
  \end{Estructura}


\Title{Invariante de representacion en castellano}

- Si raiz es $NULL$ entonces el conjunto de claves es vac\'io. \\
- Si la raiz no es $NULL$ entonces el conjunto de claves es no vac\'io. \\
- Todos los elementos del conjunto de claves estan definidos en el diccString. \\
- Ningun nodo tiene entre sus siguientes a un nodo que esta ``antes'' que \'el \\ 

\tadOperacion{Abs}{estr$/d$}{dicc(string, $\alpha$)}{Rep($d$)}%
\tadAxioma{Abs($d$)}
{
$dic$ : dicc(string, $\alpha$) / \\
	claves($dic$) $\igobs$ $d$.claves $\land$ \\
	
	($\forall s$: string) 
		((def?($s$, $dic$) $\igobs$ $s \in d$.claves)  $\yluego$ \\
		(def?($s$, $dic$) $\impluego$ obtener($s$, $dic$) $\igobs$ significado de s en la estructura))   
}%

\end{Representacion}

\begin{Algoritmos}
\subsection{Algoritmos}  
  
%crear diccTrie
\begin{algorithm}[H]
{\textbf{iVacio}() $\to$ $res$ : diccString($\alpha$)}
\begin{algorithmic}[1]

\State puntero(nodo) $iRaiz$ $\gets $ $NULL$  \Comment $O(1)$
 
\State conj($string$) $iClaves$ $\gets$ Vacio() \Comment $O(1)$

\State $res$ $\gets$  $\langle$ $iRaiz$, $iClaves$ $\rangle$ \Comment $O(1)$

\medskip
\State \underline{Complejidad:} $O(1)$
\State \underline{Justificaion:} $O(1)$ $+$ $O(1)$ $+$ $O(1)$

\end{algorithmic}
\end{algorithm}  

%CLAVES
\begin{algorithm}[H]
{\textbf{iClaves}(\In{d}{diccString($\alpha$)}}) $\to$ $res$ : conj($\alpha$)
\begin{algorithmic}[1]

\State $res$ $\gets$ $d$.$claves$ \Comment $O(1)$

\medskip
\State \underline{Complejidad:} $O(1)$

\end{algorithmic}
\end{algorithm}
  
  
  
%OBTENER
\begin{algorithm}[H]
{\textbf{iObtener}(\In{p}{string}, \In{d}{diccString($\alpha$)}}) $\to$ $res$ : $\alpha$
\begin{algorithmic}[1]

\State puntero(nodo) $n$ $\gets$ $raiz$ \Comment $O(1)$

\State nat $i$ $\gets$ $0$ \Comment $O(1)$


\While{$i$ $<$ Longitud($p$)}   \Comment Se repite $|p|$ $O(1)$
		

	\State $actual$ $\gets$ (*$actual$).$siguientes$[ord($p$[$i$])] \Comment $O(1)$

	\State $i$ $\gets$ $i + 1$ \Comment $O(1)$
\EndWhile 


\State $res$ $\gets$ (*$actual$).$definicion$ \Comment $O(1)$ 

\medskip
\State \underline{Complejidad:} $O(|P|)$
\State \underline{Justificaion:} Siendo $|P|$ el largo de la clave mas larga, sea cual sea $p$, $|p|$ $\leq$ $|P|$ entonces 	$O(|p|)$ $=$ $O(|P|)$

\end{algorithmic}
\end{algorithm}
  

%DEFINIR
\begin{algorithm}[H]
{\textbf{iDefinir}(\Inout{d}{diccString($\alpha$)}, \In{p}{string}, \In{a}{$\alpha$})}
\begin{algorithmic}[1]

\State Puntero(nodo) $actual$ $\gets$ $raiz$ \Comment $O(1)$

\State Nat $i$ $\gets$ $0$ \Comment $O(1)$

\State bool $esNueva$ $\gets$ $true$ \Comment $O(1)$

\While{$i$ $<$ Longitud($p$)}   \Comment Se repite $|p|$ $O(1)$
	
	
	\If{(*$actual$).$siguientes$[ord($p$[$i$])] $=$ $NULL$} \Comment $O(1)$
		\State (*$actual$).$siguientes$[ord($p$[$i$])] $\gets$	 $\&$ $\langle$ $NULL$, arreglo[$256$]($NULL$), $NULL$ $\rangle$  \Comment $O(1)$
		
		\State $esNueva$ $\gets$ $false$ \Comment $O(1)$	 
	
	\EndIf
	
	\State $actual$ $\gets$ (*$actual$).$siguientes$[ord($p$[$i$])] \Comment $O(1)$

	\State $i$ $\gets$ $i + 1$
\EndWhile 

\If{(*$actual$).definicion  $\neq$  $NULL$} \Comment $O(1)$
	\State (*$actual$).definicion $\gets$ $NULL$ \Comment se libera la memoria acupada por definicion $O(1)$
\EndIf
\State (*$actual$).$definicion$ $\gets$ $\&$ $a$ \Comment $O(1)$

\If{$esNueva$} \Comment $O(1)$
	\State itConj($string$) $it$ $\gets$  $claves$.AgregarRapido($s$) \Comment $O(1)$
	
	\State (*$actual$).$itClave$ $\gets$ $\&$ $it$ \Comment $O(1)$
\EndIf

\medskip
\State \underline{Complejidad:} $O(|P|)$
\State \underline{Justificaion:} Siendo $|P|$ el largo de la clave mas larga, sea cual sea $p$, $|p|$ $\leq$ $|P|$ entonces 	$O(|p|)$ $=$ $O(|P|)$


\end{algorithmic}
\end{algorithm}  
  
  
%DEFINIDO
\begin{algorithm}[H]
{\textbf{iDef?}(\In{p}{string}) $\to$ $res$ : $bool$}
\begin{algorithmic}[1]

\State Nat $i$ $\gets$ $0$ \Comment $O(1)$

\State bool $pertenece$ $\gets$ $true$ \Comment $O(1)$

\State puntero(nodo) $actual$ $\gets$ $raiz$ \Comment $O(1)$ 

\While{$i$ $<$ Longitud($p$) $\land$  $pertenece$}   \Comment Se repite $|p|$ $O(1)$
		
	
	\If{(*$actual$).$siguientes$[ord($p$[$i$])] $=$ $NULL$} \Comment $O(1)$

		\State $pertenece$ $\gets$ $false$ \Comment $O(1)$	 
	
	\EndIf
	
	\State $actual$ $\gets$ (*$actual$).siguientes[$ord(p[i])$] \Comment $O(1)$

	\State $i$ $\gets$ $i$ $+$ 1 \Comment $O(1)$
\EndWhile 

\If {(*$actual$).$significado$ $=$ $NULL$} \Comment $O(1)$ 
	\State $pertenece$ $\gets$ $false$ \Comment $O(1)$
\EndIf

\State $res$ $\gets$ $pertenece$ \Comment $O(1)$ 

\medskip
\State \underline{Complejidad:} $O(|P|)$
\State \underline{Justificaion:} Siendo $|P|$ el largo de la clave mas larga, sea cual sea $p$, $|p|$ $\leq$ $|P|$ entonces 	$O(|p|)$ $=$ $O(|P|)$

\end{algorithmic}
\end{algorithm}



%CLAVES
\begin{algorithm}[H]
{\textbf{iBorrar}(\In{p}{string}, \Inout{d}{diccString($\alpha$)}})
\begin{algorithmic}[1]

\State bool $borrarRaiz$ $\gets$ $d$.Claves() $=$ 1 \Comment $O(1)$

\State puntero(nodo) $reserva$ $\gets$ $raiz$ \Comment $O(1)$ 
\State nat $rindex$ $\gets$ 0 \Comment $O(1)$

\State puntero(nodo) $actual$ $\gets$ $raiz$ \Comment $O(1)$ 

\State nat $i$ $\gets$ 0 \Comment $O(1)$
\While{$i$ $<$ Longitud($p$) }   \Comment Se repite $|p|$ $O(1)$
		
	
	
	\State $actual$ $\gets$ (*$actual$).siguientes[$ord(p[i])$] \Comment $O(1)$

	\State bool $definido$ $\gets$ $i$ $\neq$ $|p|$ $-$ 1 $\land$ (*$actual$).definicion $\neq$ $NULL$ \Comment $O(1)$

	\If{CuentaHijos($actual$) $>$ 1 $\lor$ $definido$} \Comment $O(1)$

	\State $reserva$ $\gets$ $actual$ \Comment $O(1)$
	\State $rindex$ $\gets$ $i$ $+$ 1 \Comment $O(1)$
	
	\EndIf

	\State $i$ $\gets$ $i$ $+$ 1 \Comment $O(1)$
\EndWhile 


\State EliminarSiguiente((*$actual$).(*$itClave$)) \Comment $O(1)$

\State (*$actual$).$itClave$ $\gets$ $NULL$ \Comment Se libera la memoria ocupada por (*$actual$).$itClave$ $O(1)$

\If{CuentaHijos($actual$) $>$ 1} \Comment $O(1)$

	\State (*$actual$).definicion $\gets$ $NULL$ \Comment se libera la memoria ocupada por la definicion $O(1)$

\EndIf 

\If{CuentaHijos($actual$) $=$ 0} \Comment $O(1)$
	\State BorrarDesde($reserva$, $rindex$) \Comment $O(|P|)$
\EndIf 


\If{borrarRaiz} \Comment $O(1)$
	\State $d$.raiz $\gets$ $NULL$ \Comment se libera la memoria ocupada por raiz $O(1)$
\EndIf 


\medskip
\State \underline{Complejidad:} $O(|P|)$
\State \underline{Justificaion:} Siendo $|P|$ el largo de la clave mas larga, sea cual sea $p$, $|p|$ $\leq$ $|P|$ entonces $O(2*|p|)$ $=$	$O(|p|)$ $=$ $O(|P|)$

\end{algorithmic}
\end{algorithm}
  
\pagebreak
  
\textbf{Pre} $\equiv$ \{$desde$ y $desde$.siguientes[$a$] no nulos\}\\%
\textbf{Post} $\equiv$ \{Borra la rama a partir de $desde$.siguientes[$a$]\}%

 
\begin{algorithm}[H]
{\textbf{iBorrarDesde}(\Inout{desde}{puntero(nodo)}}, \In{a}{nat})
\begin{algorithmic}[1]

\State puntero(nodo) $temp$ $\gets$ raiz \Comment $O(1)$

\State $desde$ $\gets$ (*$desde$).$siguientes$[a] \Comment $O(1)$
\State (*$desde$).$siguientes$[a] $\gets$ $NULL$ \Comment Se libera la memoria ocupada por (*desde).siguientes[a] $O(1)$

\While{$desde$ $\neq$ $NULL$} \Comment se repite a los sumo $|P|$ veces, siendo p la clave mas larga $O(1)$

	\State puntero(nodo) $temp$ $\gets$ $desde$ \Comment $O(1)$

	\State bool $sigue$ $\gets$ false \Comment $O(1)$

	\State nat $i$ $\gets$ 0 \State $O(1)$

	\While{$i$ $<$ 256} \Comment se repite siempre 256 veces $O(1)$

		\If{(*$desde$).siguientes[$i$] $\neq$ $NULL$} \Comment $O(1)$
			\State $desde$ $\gets$ (*$desde$).siguientes[$i$] \Comment $O(1)$		
		
			\State $sigue$ $\gets$ true \Comment $O(1)$		
		
		\EndIf
		\State $i$ $\gets$ $i$ $+$ 1 \Comment $O(1)$

	\EndWhile

\EndWhile

\If{$\lnot$ $sigue$} \Comment $O(1)$
	\State $desde$ $\gets$ $NULL$ \Comment se libera la memoria ocupada por $desde$ $O(1)$

\EndIf

\State $temp$ $\gets$ $NULL$ \Comment se libera la memoria ocupada por $temp$ $O(1)$


\medskip
\State \underline{Complejidad:} $O(|P|)$
\State \underline{Justificaion:} Siendo $|P|$ el largo de la clave mas larga, sea cual sea la rama que estamos borrando, es mas corta que la rama representada por la clave mas larga. Llamo $p$ a la clave de la rama que estamos borrando y como $|p|$ $\leq$ $|P|$ entonces $O(|p|)$ = $O(|P|)$ 

\end{algorithmic}
\end{algorithm}
  
 

\textbf{Pre} $\equiv$ \{$desde$ no nulo\}\\%
\textbf{Post} $\equiv$ \{deveulve la cantidad de punteros no nulos en $desde$.siguientes\}%

\begin{algorithm}[H]
{\textbf{iCuentaHijos}(\In{desde}{puntero(nodo)}} $\to$ $res$ : $nat$)
\begin{algorithmic}[1]

\State nat $i$ $\gets$ 0 \Comment $O(1)$
\State nat $hijos$ $\gets$ 0 \Comment $O(1)$

\While{$i$ $<$ 256} \Comment se repite siempre 256 veces $O(1)$
	\If{(*$actutal$).siguiente[$i$] $\neq$ $NULL$} \State $O(1)$ 
	
	\State $suma$ $\gets$ $suma$ $+$ 1 \Comment $O(1)$
	\State $i$ $\gets$ $i$ $+$ 1 \Comment $O(1)$
	\EndIf
\EndWhile
\State $res$ $\gets$ $hijos$ \Comment $O(1)$


\medskip
\State \underline{Complejidad:} $O(1)$
\State \underline{Justificaion:}  $O(1)$ $+$ $O(1)$ $+$ $O(256)$ $+$ $O(1)$ $+$ $O(1)$ $+$ $O(1)$ $=$ $O(261)$ = $O(1)$

\end{algorithmic}
\end{algorithm}

  
\end{Algoritmos}


\subsection{Servicios usados} 

De Conjunto Lineal \\
 - Vacio() debe ser O(1) \\ 
 - AgregarRapido( conj($\alpha$), $\alpha$ $\alpha$) debe ser O(copy(a)) \\
 - EliminarSiguiente(itConj($\alpha$)) debe ser O(1) \\

De String \\
 - Longitud(string) debe ser O(1) \\
 
De Char \\
 - ord(char) debe ser O(1) \\
 

\pagebreak

\section{iterDiccString($\alpha$)}

\begin{Interfaz}

  \textbf{par\'ametros formales}\hangindent=2\parindent\\
  \parbox{1.7cm}{\textbf{g\'eneros}} $\alpha$
 
 
  \textbf{se explica con}: \tadNombre{IteradorUnidireccional($\alpha$)}, \tadNombre{DiccString($\alpha$)}.

  \textbf{g\'eneros}: \TipoVariable{iterDiccString($\alpha$)}.


\InterfazFuncion{CrearIt}{\In{d}{diccString($\alpha$)}}{iterDiccString($\alpha$)}%
[true]
{alias(esPermutaci\'on(SecuSuby($res$), claves($d$)))}%
[$O(1)$] % Por enunciado
[Devuelve un iterador al diccionario]
{Aliasing: El iterador se invalida si se modifican claves del diccionario}

\InterfazFuncion{HayMas?}{\In{it}{iterDiccString($\alpha$)}}{bool}%
[true]
{$res$ = hayMas?($it$)}%
[$O(1)$]
[Devuelve true si hay mas claves por recorrer]

% AVISAR EN INFORME SOBRE EL CAMBIO DE ARIDAD
\InterfazFuncion{Actual}{\In{it}{iterDiccString($\alpha$)}}{tupla<string, $\alpha$>}%
[hayMas?($it$)]
{$res$ = actual($it$)}%
[$O(|P|)$, donde |P| es la longitud de la clave mas larga]
[Devuelve una tupla con el elemento actual y su significado]

\InterfazFuncion{Avanzar}{\Inout{it}{iterDiccString($\alpha$)}}{}%
[$it$ = $it_0$ $\land$ hayMas?($it$)]
{$it$ = avanzar($it_0$)}%
[$O(1)$]
[Avanza a la posicion siguiente del iterador]



\end{Interfaz}

\pagebreak

\begin{Representacion}
  
  \Title{Representaci\'on del iterDiccString}  
  
  \begin{Estructura}{iterDiccString($\alpha$)}[estr]
    \begin{Tupla}[estr]
      \tupItem{itClave}{itConj(string)}%
      \tupItem{dicc}{diccString($\alpha$)}%
    \end{Tupla}
  \end{Estructura}


\Title{Invariante de representacion en castellano}

- Sarasa sarasa

\tadOperacion{Abs}{estr$/d$}{dicc(string, $\alpha$)}{Rep($d$)}%
\tadAxioma{Abs($d$)}
{
    sarasa sarasa
}%

\end{Representacion}

\begin{Algoritmos}


% crearIt 

\begin{algorithm}[H]
{\textbf{iCrearIt}(\In{d}{diccString($\alpha$)}) $\to$ $res$ : iterDiccString($\alpha$)}
\begin{algorithmic}[1]

\State $res \gets$ $\langle$ $d$ , CrearIt(Claves($d$)) $\rangle$ \Comment Por referencia $O(1)$
 
\medskip
\Statex \underline{Complejidad:} $O(1)$
\Statex \underline{Justificacion:} Crear el iterador de conjunto es O(1), obtener las claves del diccionario es O(1), crear la tupla con los dos elementos es O(1). La complejidad total es O(1)
\end{algorithmic}
\end{algorithm}


% hayMas? 

\begin{algorithm}[H]
{\textbf{iHayMas?}(\In{iter}{iterDiccString($\alpha$)}) $\to$ $res$ : bool}
\begin{algorithmic}[1]

\State $res \gets$ HaySiguiente($iter$.itClave) \Comment $O(1)$

\medskip
\Statex \underline{Complejidad:} $O(1)$
\end{algorithmic}
\end{algorithm}

% actual

\begin{algorithm}[H]
{\textbf{iActual}(\In{iter}{iterDiccString($\alpha$)}) $\to$ $res$ : tupla$<$string, $\alpha$ $>$}
\begin{algorithmic}[1]

\State $res \gets$ $\langle$ Siguiente($iter$.iClave), Obtener(Siguiente($iter$.iClave), $iter$.dicc) $\rangle$ \Comment $O(|P|)$

\medskip
\Statex \underline{Complejidad:} $O(|P|)$
\Statex \underline{Justificaci\'on:} Para crear la tupla, necesito acceder al significado de la clave. Por las complejidades del DiccString, el peor caso se corresponde con la longitud de la clave mas larga: O(|P|), donde P es la clave mas large. Acceder al Siguiente del conjunto es O(1). En total, el algoritmo cuesta O(|P|)

\end{algorithmic}
\end{algorithm}


\end{Algoritmos}  
  


Servicios usados:

- CrearIt(conj($\alpha$)) debe ser O(1) \\
- Claves(diccString($\alpha$)) debe ser O(1) \\
- HaySiguiente(itConj($\alpha$)) debe ser O(1) \\
- Siguiente(itConj($\alpha$)) debe ser O(1) \\
- Obtener(string, diccString($\alpha$)) debe ser O(|P|) \\



\pagebreak

\section{Cola de Entrenadores}
\begin{Interfaz}

  \textbf{se explica con}: \tadNombre{Cola de Prioridad($\alpha$), Iterador Cola de Prioridades($\alpha$)}.

  \textbf{g\'eneros}: \TipoVariable{colaEntr}, \TipoVariable{itcolaEntrenador)}.

  \Title{Operaciones b\'asicas de Cola de Entrenadores}


  \InterfazFuncion{Vacia}{}{colaEntr}%
  {$res$ $\igobs$ $vacia$}%
  [$O(1)$]
  [genera una nueva cola de entrenadores]


  \InterfazFuncion{Encolar}{\Inout{h}{colaEntr}, \In{j}{entrenador}}{itcolaEntrenador)}
  [$h$ $\igobs$ $h_0$]
  {$h$ $\igobs$ encolar($h_0$, $j$)}
  [$O(log(EC))$]
  [encola $j$ a $h$]
  [el elemento $j$ se encola por copia]

  \InterfazFuncion{EsVacia?}{\In{h}{colaEntr}}{bool}
  {$res$ $\igobs$ vacia?($h$)}
  [$O(1)$]
  [devuelve true si la cola es vacia]


  \InterfazFuncion{Proximo}{\In{h}{colaEntr}}{itcolaEntrenador)}
  [$\neg$ vacia?(h)]  
  {alias($res$ $\igobs$ proximo($h$))} 
  [$O(1)$]
  [devuelve el pr\'oximo de la cola]
  [el elemento se devuelve por copia]
  
  
  \InterfazFuncion{Desencolar}{\Inout{h}{colaEntr}}{}
  [$h$ $\igobs$ $h_0$ $\land$ $\neg$ vacia($h$) ]  
  {$h$ $\igobs$ desencolar($h_0$)} 
  [$O(log(EC))$]
  [desencola el pr\'oximo de h]
 


  \Title{Operaciones del iterador}
 
  \InterfazFuncion{Borrar}{\In {it}{itcolaEntrenador)}, \In{h}{colaEntr}}{}%
  [$siguiente(it) \in elementos(h)$]
  {$siguiente(it) \not\in elementos(h)$}
  [$O(log(EC))$]
  [Borra el elemento dado]
  
\end{Interfaz}

\begin{Representacion}
 
  \begin{Estructura}{colaEntr}[estr]
      \begin{Tupla}[estr]%
      \tupItem{raiz}{puntero(Nodoheap)}%
      \tupItem{ultimo}{puntero(Nodoheap)}%
    \end{Tupla}
  \end{Estructura}

	\begin{Tupla}[Nodoheap]
	\tupItem{elemento}{entrenador}%
	\tupItem{padre}{puntero(Nodoheap)}%
	\tupItem{izq}{puntero(Nodoheap)}%
	\tupItem{der}{puntero(Nodoheap)}%
	\end{Tupla}

	\begin{Tupla}[entrenador]
	\tupItem{id}{nat}%
	\tupItem{cantCapt}{nat}%
	\end{Tupla}


\Title{Invariante de representaci\'on en castellano}	
	
- Si raiz es $NULL$ entonces ultimo tambien es $NULL$. \\
- Si la raiz no es $NULL$ entonces ultimo tampoco es $NULL$.. \\
- Ningun nodo tiene como padre a ninguno de sus hijos ni a los hijos de sus hijos. \\ 


\tadOperacion{Abs}{estr$/c$}{colaPrior(Tupla(nat, nat}{Rep($d$)}%
\tadAxioma{Abs($c$)}
{
$cola$ : colaPrior(Tupla(nat, nat)) / \\
	vacia?($cola$)  $\iff$  $c$.$raiz$ $=$ $NULL$ $\yluego$ \\
	proximo($cola$) $\igobs$ $c$.(*$raiz$).elemento \\
	
}%

\end{Representacion}

\begin{Representacion}
 
  \begin{Estructura}{itcolaRntrenador)}[estr]
      \begin{Tupla}[estr]%
      \tupItem{siguiente}{puntero(Nodoheap)}%
      \tupItem{estructura}{puntero(colaEntr)}%
    \end{Tupla}
  \end{Estructura}

  \RepFc[estr$/e$]{true}

  ~

  \AbsFc[estr]{Cola de entrenadores}[e]{}

\end{Representacion}



\begin{Algoritmos}
   
\begin{algorithm}[H]{\textbf{iVacio}() $\to$ $res$ : colaEntr}
    	\begin{algorithmic}[1]
			 \State $res$ $\gets$  $<$ $NULL$, $NULL$ $>$ \Comment $O(1)$

			\medskip
			\Statex \underline{Complejidad:} $O(1)$
			\Statex \underline{Justificaci\'on:} Todas las operaciones son $O(1)$ 
    	\end{algorithmic}
\end{algorithm}
   
\begin{algorithm}[H]{\textbf{iEsVacia?}(\In{h}{colaEntr}) $\to$ $res$ : bool}
    	\begin{algorithmic}[1]
			 \State $res$ $\gets$  $h$.raiz $\neq$ $NULL$ \Comment $O(1)$

			\medskip
			\Statex \underline{Complejidad:} $O(1)$
			\Statex \underline{Justificaci\'on:} Todas las operaciones son $O(1)$ 
    	\end{algorithmic}
\end{algorithm}

[H]{\textbf{iEncolarElem}(\Inout{h}{colaEntr}, \In{j}{entrenador}) $\to$ $res$ : itcolaEntrenador)}
    	\begin{algorithmic}[1]
    	\State Nodoheap $n$ $\gets$ $<j, NULL, NULL, NULL>$ \Comment $O(1)$
    	\State puntero(Nodoheap) $p$ $\gets$ $\&$ $n$  \Comment $O(1)$
		\If{$EsVacia?(h)$} \Comment Insertamos el nodo al final de la cola $O(1)$
    		\State $h$.raiz $\gets$ $p$ \Comment $O(1)$
    		\State $h$.ultimo $\gets$ $p$ \Comment $O(1)$
    	\Else
    		\If{$h$.raiz $=$ $h$.ultimo} \Comment $O(1)$
    			\State ($p$ $\rightarrow$ $padre$) $\gets$ $h$.raiz \Comment $O(1)$
    			\State ($h$.raiz $\rightarrow$ $izq$) $\gets$ $p$ \Comment $O(1)$
    			\State $h$.ultimo $\gets$ $p$    \Comment $O(1)$
    		\Else
    			\If{($h$.raiz $\rightarrow$ $izq$) $=$ $h$.ultimo} \Comment $O(1)$
    				\State ($p$ $\rightarrow$ $padre$) $\gets$ $h$.raiz \Comment $O(1)$
    				\State ($h$.raiz $\rightarrow$ $der$) $\gets$ $p$ \Comment $O(1)$
    				\State $h$.ultimo $\gets$ $p$ \Comment $O(1)$
    			\Else
    				\If{EsHijoIzquierdo?($h$.ultimo)} \Comment $O(1)$
    					\State ($p$ $\rightarrow$ $padre$) $\gets$ ($h$.ultimo $\rightarrow$ $padre$) \Comment $O(1)$
    					\State ($h$.ultimo $\rightarrow$ $padre$ $\rightarrow$ $der$) $\gets$ $p$ \Comment $O(1)$
    					\State $h$.ultimo $\gets$ $p$ \Comment $O(1)$
    				\Else
						\State puntero(Nodoheap) $i$ $\gets$ $h$.ultimo	 \Comment $O(1)$	
    					\While {$i$ $\neq$ $NULL$ $\land$ $\neg$ EsHijoIzquierdo?($j$)} \Comment $O(log(EC))$
    						\State $i$ $\gets$ ($i$ $\rightarrow$ $padre$) \Comment $O(1)$
    					
    					\EndWhile
    					\If{$i$ $=$ $NULL$} \Comment $O(1)$
    						\State puntero(Nodoheap) $ultimoizq$ $\gets$ $h$.raiz \Comment $O(1)$
    						\While {($ultimoizq$ $\rightarrow$ $izq$) $\neq$ $NULL$} \Comment $O(log(EC))$
    							\State $ultimoizq \gets (ultimoizq \rightarrow izq)$ \Comment $O(1)$
    						\EndWhile
    						\State $(p \rightarrow padre) \gets ultimoizqu$ \Comment $O(1)$
    						\State $(ultimoizq \rightarrow izq) \gets p$ \Comment $O(1)$
    						\State $h.ultimo \gets p$ \Comment $O(1)$
    					\Else
    						\State $i \gets (i \rightarrow padre \rightarrow der)$ \Comment $O(1)$
    						\While{$(i \rightarrow izq) \not= NULL$} \Comment $O(log(EC))$
    							\State $i \gets (i \rightarrow izq)$ \Comment $O(1)$
    						\EndWhile
    						\State $(p \rightarrow padre) \gets i$ \Comment $O(1)$
    						\State $(j \rightarrow izq) \gets p$ \Comment $O(1)$
    						\State $h.ultimo \gets p$ \Comment $O(1)$
    					\EndIf
    				\EndIf		
    			\EndIf	    	
	    	\EndIf
		\EndIf
		\State $puntero(colaEntr) q \gets h$
		\State $it \gets CrearItCola(h.ultimo, q)$ \Comment $O(1)$

		\If{$h.raiz \not= h.ultimo$} \Comment $O(1)$
			\State $puntero(Nodo)$ nuevoNodo $\gets h.ultimo$			\Comment $O(1)$
			\While{$((nuevoNodo \rightarrow elemento.cantCapt)$ <\hspace{0.4mm} $(nuevoNodo \rightarrow padre \rightarrow elemento.cantCapt)) \lor (((nuevoNodo \rightarrow elemento.cantCapt)$ = $(nuevoNodo \rightarrow padre \rightarrow elemento.cantCapt)) \land ((nuevoNodo \rightarrow elemento.id)$ <\hspace{0.4mm} $(nuevoNodo \rightarrow padre \rightarrow elemento.id)))$} \Comment $O(log(EC))$
				\State $puntero(Nodoheap)$ aSwapear $\gets (nuevoNodo \rightarrow padre)$
    		\If{$h.ultimo = nuevoNodo$}
    			\State $h.ultimo \gets aSwapear$
    		\EndIf
			\If{$EsHijoIzquierdo?(nuevoNodo)$} \Comment $O(1)$
						\If{$(nuevoNodo \rightarrow izq) = NULL$} \Comment $O(1)$
								\State $(aSwapear \rightarrow izq) \gets NULL$ \Comment $O(1)$
							\Else
								\State $(aSwapear \rightarrow izq) \gets (nuevoNodo \rightarrow izq)$ \Comment $O(1)$
								\State $(nuevoNodo \rightarrow izq \rightarrow padre) \gets aSwapear$ \Comment $O(1)$
							\EndIf
							\If{$(nuevoNodo \rightarrow der) = NULL$} \Comment $O(1)$
								\State $(nuevoNodo \rightarrow der) \gets (aSwapear \rightarrow der)$ \Comment $O(1)$
								\If{$(aSwapear \rightarrow der) \not= NULL$} \Comment $O(1)$
								\State $(aSwapear \rightarrow der \rightarrow padre) \gets nuevoNodo$	 \Comment $O(1)$						
								\EndIf
								\State $(aSwapear \rightarrow der) \gets NULL$ \Comment $O(1)$
							\Else
								\If{$(aSwapear \rightarrow der) = NULL$} \Comment $O(1)$
									\State $(aSwapear \rightarrow der) \gets (nodoNuevo \rightarrow der)$ \Comment $O(1)$
									\State $(nodoNuevo \rightarrow der \rightarrow padre) \gets aSwapear$ \Comment $O(1)$
									\State $(nodoNuevo \rightarrow der) \gets NULL$ \Comment $O(1)$
								\Else
									\State $(nodoNuevo \rightarrow der \rightarrow padre) \gets aSwapear$ \Comment $O(1)$
									\State $(aSwapear \rightarrow der \rightarrow padre) \gets nuevoNodo$ \Comment $O(1)$
									\State $puntero(Nodoheap)$ auxiliar $\gets (nuevoNodo \rightarrow der)$ \Comment $O(1)$
									\State $(nuevoNodo \rightarrow der) \gets (aSwapear \rightarrow der)$ \Comment $O(1)$
									\State $(aSwapear \rightarrow der) \gets auxiliar$ \Comment $O(1)$
								\EndIf
							\EndIf
						\State $(nuevoNodo \rightarrow izq) \gets aSwapear$ \Comment $O(1)$
					\Else
						\If{$(nuevoNodo \rightarrow der) = NULL$} \Comment $O(1)$
								\State $(aSwapear \rightarrow der) \gets NULL$ \Comment $O(1)$
							\Else
								\State $(aSwapear \rightarrow der) \gets (nuevoNodo \rightarrow der)$ \Comment $O(1)$
								\State $(nuevoNodo \rightarrow der \rightarrow padre) \gets aSwapear$ \Comment $O(1)$
							\EndIf
							\If{$(nuevoNodo \rightarrow izq) = NULL$} \Comment $O(1)$
								\State $(nuevoNodo \rightarrow izq) \gets (aSwapear \rightarrow izq)$ \Comment $O(1)$
								\If{$(aSwapear \rightarrow izq) \not= NULL$} \Comment $O(1)$
									\State $(aSwapear \rightarrow izq \rightarrow padre) \gets nuevoNodo$	\Comment $O(1)$						
								\EndIf
								\State $(aSwapear \rightarrow izq) \gets NULL$ \Comment $O(1)$
							\Else
								\If{$(aSwapear \rightarrow izq) = NULL$} \Comment $O(1)$
									\State $(aSwapear \rightarrow izq) \gets (nodoNuevo \rightarrow izq)$ \Comment $O(1)$
									\State $(nodoNuevo \rightarrow izq \rightarrow padre) \gets aSwapear$ \Comment $O(1)$
									\State $(nodoNuevo \rightarrow izq) \gets NULL$ \Comment $O(1)$
								\Else
									\State $(nodoNuevo \rightarrow izq \rightarrow padre) \gets aSwapear$ \Comment $O(1)$
									\State $(aSwapear \rightarrow izq \rightarrow padre) \gets nuevoNodo$ \Comment $O(1)$
									\State $puntero(Nodoheap)$ auxiliar $\gets (nuevoNodo \rightarrow izq)$ \Comment $O(1)$
									\State $(nuevoNodo \rightarrow izq) \gets (aSwapear \rightarrow izq)$ \Comment $O(1)$
									\State $(aSwapear \rightarrow izq) \gets auxiliar$ \Comment $O(1)$
								\EndIf
							\EndIf
						\State $(nuevoNodo \rightarrow der) \gets aSwapear$ \Comment $O(1)$
					\EndIf	
					\If{$aSwapear = h.raiz$}					
					\State $(aSwapear \rightarrow padre) \gets nuevoNodo$ \Comment $O(1)$
					\State $h.raiz \gets nuevoNodo$	 \Comment $O(1)$
					\State $(nuevoNodo \rightarrow padre) \gets NULL$			 \Comment $O(1)$
					\Else
					\State $puntero(Nodoheap)$ abuelo $\gets (aSwapear \rightarrow padre)$ \Comment $O(1)$
						\If{$EsHijoIzquierdo?(aSwapear)$} \Comment $O(1)$
							\State $(abuelo \rightarrow izq) \gets nuevoNuevo$ \Comment $O(1)$
							\State $(nuevoNodo \rightarrow padre) \gets abuelo$ \Comment $O(1)$
							\State $(aSwapear \rightarrow padre) \gets nuevoNodo$ \Comment $O(1)$
						\Else
							\State $(abuelo \rightarrow der) \gets nuevoNuevo$ \Comment $O(1)$
							\State $(nuevoNodo \rightarrow padre) \gets abuelo$ \Comment $O(1)$
							\State $(aSwapear \rightarrow padre) \gets nuevoNodo$ \Comment $O(1)$
						\EndIf
					\EndIf
						
			\EndWhile
		\EndIf
		\State $res \gets it$ \Comment $O(1)$
			\medskip
			\Statex \underline{Complejidad:} $O(log(EC))$
			\Statex \underline{Justificaci\'on:} Todas las operaciones son $O(1)$ y en el ciclo se va recorriendo la cola desde una de las hojas hasta la raiz (o vice versa). Al ser la cola un \'arbol completo (ya que siempre se agregan elementos al \'ultimo lugar disponible), si la cantidad de elementos es EC en el peor cas, entonces su altura es log(EC). Por lo tanto los ciclos se repiten log(EC) veces en el peor caso.
    	\end{algorithmic}   	


\begin{algorithm}[H]{\textbf{iProximo}(\In{h}{colaEntr}) $\to$ $res$ : entrenador}
    	\begin{algorithmic}[1]
    		 \State $res \gets (h.raiz \rightarrow elemento)$ \Comment $O(1)$
    		 \medskip
			\Statex \underline{Complejidad:} $O(1)$
			\Statex \underline{Justificaci\'on:} Todas las operaciones son $O(1)$ 
    	\end{algorithmic}
\end{algorithm}

\begin{algorithm}[H]{\textbf{iDesencolar}(\In{h}{colaEntr})}
    	\begin{algorithmic}[1]
			 \State $puntero(colaEntr) q \gets h$    		 
    		 \State $it \gets CrearItCola(h.raiz, q)$ \Comment $O(1)$
    		 \State $Borrar(it)$ \Comment $O(log(EC))$
    		 \medskip
			\Statex \underline{Complejidad:} $O(log(EC))$
			\Statex \underline{Justificaci\'on:} Hay una operacion $O(1)$ y una $O(log(EC))$, por lo tanto, por \'algebra de \'ordenes $O(1)$ + $O(log(EC))$ = $O(log(EC))$
    	\end{algorithmic}
\end{algorithm}

\end{Algoritmos}

\begin{Algoritmos}


[H]{\textbf{iBorrar}(\In{it}{itcolaEntrenador)}, \In{h}{}colaEntr)}
    	\begin{algorithmic}[1]
    		\If{$(it.estructura \rightarrow raiz) = (it.estructura \rightarrow ultimo)$} \Comment $O(1)$
    			\State $it.siguiente \gets NULL$ \Comment Libera memoria $O(1)$
    			\State $(it.estructura \rightarrow raiz) \gets NULL$ \Comment $O(1)$
    			\State $(it.estructura \rightarrow ultimo) \gets NULL$ \Comment $O(1)$
    		\Else    		
    			\State $puntero(Nodo)$ cambiado $\gets (it.estructura \rightarrow ultimo)$ \Comment $O(1)$
    			\State $aPonerUltimo \gets it.siguiente$
				\If{($it.estructura \rightarrow ultimo \rightarrow padre) \not= aPonerUltimo$} \Comment Swapeamos con el \'ultimo de la cola. $O(1)$	
					\State $(it.estructura \rightarrow ultimo \rightarrow izq) \gets (aPonerUltimo \rightarrow izq)$ \Comment $O(1)$
			\State $(it.estructura \rightarrow ultimo \rightarrow der) \gets (aPonerUltimo \rightarrow der)$ \Comment $O(1)$	
			\If{$(aPonerUltimo \rightarrow izq) \not= NULL$} \Comment $O(1)$
				\State $(aPonerUltimo \rightarrow izq \rightarrow padre) \gets (it.estructura \rightarrow ultimo)$ \Comment $O(1)$
			\EndIf	
			\If{$(aPonerUltimo \rightarrow der) \not= NULL$} \Comment $O(1)$
				\State $(aPonerUltimo \rightarrow der \rightarrow padre) \gets it.estructura \rightarrow ultimo$ \Comment $O(1)$
			\EndIf		
			\If{$aPonerUltimo \not= it.estructura \rightarrow raiz$} \Comment $O(1)$
				\State $padreaPonerUltimo \gets (aPonerUltimo \rightarrow padre)$ \Comment $O(1)$
				\State $padreUltimo \gets (it.estructura \rightarrow ultimo \rightarrow padre)$			 	 \Comment $O(1)$
			 	\If{$EsHijoIzquierdo?(aPonerUltimo) \land EsHijoIzquierdo?(it.estructura \rightarrow ultimo)$} \Comment $O(1)$
			 	\State $(padreaPonerUltimo \rightarrow izq) \gets it.estructura \rightarrow ultimo$ \Comment $O(1)$
			 	\State $(it.estructura \rightarrow ultimo \rightarrow padre) \gets padreaPonerUltimo$ \Comment $O(1)$
			 	\State $(aPonerUltimo \rightarrow padre) \gets padreUltimo$ \Comment $O(1)$
			 	\State $(padreUltimo \rightarrow izq) \gets aPonerUltimo$ \Comment $O(1)$
			 	\Else
			 		\If{$EsHijoIzquierdo?(aPonerUltimo) \land \neg EsHijoIzquierdo?(it.estructura \rightarrow ultimo)$} \Comment $O(1)$
			 		\State $(padreaPonerUltimo \rightarrow izq) \gets it.estructura \rightarrow ultimo$ \Comment $O(1)$
			 		\State $(aPonerUltimo \rightarrow padre) \gets padreUltimo$ \Comment $O(1)$
			 		\State $(it.estructura \rightarrow ultimo \rightarrow padre) \gets padreaPonerUltimo$ \Comment $O(1)$
			 		\State $(padreUltimo \rightarrow der) \gets aPonerUltimo$ \Comment $O(1)$
			 		\Else
			 			\If{$\neg EsHijoIzquierdo?(aPonerUltimo) \land EsHijoIzquierdo?(it.estructura \rightarrow ultimo)$} \Comment $O(1)$
			 			\State $(padreaPonerUltimo \rightarrow izq) \gets it.estructura \rightarrow ultimo$ \Comment $O(1)$
			 			\State $(aPonerUltimo \rightarrow padre) \gets padreUltimo$ \Comment $O(1)$
			 			\State $(it.estructura \rightarrow ultimo \rightarrow padre) \gets padreaPonerUltimo$ \Comment $O(1)$
			 			\State $(padreUltimo \rightarrow izq) \gets aPonerUltimo$ \Comment $O(1)$
			 			\Else
			 				\If{$\neg EsHijoIzquierdo?(aPonerUltimo) \land \neg EsHijoIzquierdo?(it.estructura \rightarrow ultimo)$} \Comment $O(1)$
			 				\State $(padreaPonerUltimo \rightarrow der) \gets (it.estructura \rightarrow ultimo)$ \Comment $O(1)$
			 				\State $(it.estructura \rightarrow ultimo \rightarrow padre) \gets padreaPonerUltimo$ \Comment $O(1)$
			 				\State $(aPonerUltimo \rightarrow padre) \gets padreUltimo$ \Comment $O(1)$
			 				\State $(padreUltimo \rightarrow der) \gets aPonerUltimo$ \Comment $O(1)$
			 				\EndIf
			 			\EndIf
			 		\EndIf
			 	\EndIf
			\Else
				\State $(it.estructura \rightarrow raiz \rightarrow padre) \gets (it.estructura \rightarrow ultimo \rightarrow padre)$ \Comment $O(1)$
				\If{$EsHijoIzquierdo?(it.estructura \rightarrow ultimo)$} \Comment $O(1)$
					\State $(it.estructura \rightarrow ultimo \rightarrow padre \rightarrow izq) \gets it.estructura \rightarrow raiz$ \Comment $O(1)$
				\Else
					\State $(it.estructura \rightarrow ultimo \rightarrow padre \rightarrow der) \gets it.estructura \rightarrow raiz$ \Comment $O(1)$
				\EndIf			
				\State ($it.estructura \rightarrow ultimo \rightarrow padre) \gets NULL$ \Comment $O(1)$
				\State $(it.estructura \rightarrow raiz) \gets (it.estructura \rightarrow ultimo)$ \Comment $O(1)$
			\EndIf
	\Else
		\State $puntero(Nodoheap) nuevoNodo \gets (it.estructura \rightarrow ultimo)$ \Comment $O(1)$		
		\State $puntero(Nodoheap)$ aSwapear $\gets (nuevoNodo \rightarrow padre)$
    		\If{$(it.estructura \rightarrow ultimo) = nuevoNodo$}
    			\State $(it.estructura \rightarrow ultimo) \gets aSwapear$
    		\EndIf
			\If{$EsHijoIzquierdo?(nuevoNodo)$} \Comment $O(1)$
						\If{$(nuevoNodo \rightarrow izq) = NULL$} \Comment $O(1)$
								\State $(aSwapear \rightarrow izq) \gets NULL$ \Comment $O(1)$
							\Else
								\State $(aSwapear \rightarrow izq) \gets (nuevoNodo \rightarrow izq)$ \Comment $O(1)$
								\State $(nuevoNodo \rightarrow izq \rightarrow padre) \gets aSwapear$ \Comment $O(1)$
							\EndIf
							\If{$(nuevoNodo \rightarrow der) = NULL$} \Comment $O(1)$
								\State $(nuevoNodo \rightarrow der) \gets (aSwapear \rightarrow der)$ \Comment $O(1)$
								\If{$(aSwapear \rightarrow der) \not= NULL$} \Comment $O(1)$
								\State $(aSwapear \rightarrow der \rightarrow padre) \gets nuevoNodo$	 \Comment $O(1)$						
								\EndIf
								\State $(aSwapear \rightarrow der) \gets NULL$ \Comment $O(1)$
							\Else
								\If{$(aSwapear \rightarrow der) = NULL$} \Comment $O(1)$
									\State $(aSwapear \rightarrow der) \gets (nodoNuevo \rightarrow der)$ \Comment $O(1)$
									\State $(nodoNuevo \rightarrow der \rightarrow padre) \gets aSwapear$ \Comment $O(1)$
									\State $(nodoNuevo \rightarrow der) \gets NULL$ \Comment $O(1)$
								\Else
									\State $(nodoNuevo \rightarrow der \rightarrow padre) \gets aSwapear$ \Comment $O(1)$
									\State $(aSwapear \rightarrow der \rightarrow padre) \gets nuevoNodo$ \Comment $O(1)$
									\State $puntero(Nodoheap)$ auxiliar $\gets (nuevoNodo \rightarrow der)$ \Comment $O(1)$
									\State $(nuevoNodo \rightarrow der) \gets (aSwapear \rightarrow der)$ \Comment $O(1)$
									\State $(aSwapear \rightarrow der) \gets auxiliar$ \Comment $O(1)$
								\EndIf
							\EndIf
						\State $(nuevoNodo \rightarrow izq) \gets aSwapear$ \Comment $O(1)$
					\Else
						\If{$(nuevoNodo \rightarrow der) = NULL$} \Comment $O(1)$
								\State $(aSwapear \rightarrow der) \gets NULL$ \Comment $O(1)$
							\Else
								\State $(aSwapear \rightarrow der) \gets (nuevoNodo \rightarrow der)$ \Comment $O(1)$
								\State $(nuevoNodo \rightarrow der \rightarrow padre) \gets aSwapear$ \Comment $O(1)$
							\EndIf
							\If{$(nuevoNodo \rightarrow izq) = NULL$} \Comment $O(1)$
								\State $(nuevoNodo \rightarrow izq) \gets (aSwapear \rightarrow izq)$ \Comment $O(1)$
								\If{$(aSwapear \rightarrow izq) \not= NULL$} \Comment $O(1)$
									\State $(aSwapear \rightarrow izq \rightarrow padre) \gets nuevoNodo$	\Comment $O(1)$						
								\EndIf
								\State $(aSwapear \rightarrow izq) \gets NULL$ \Comment $O(1)$
							\Else
								\If{$(aSwapear \rightarrow izq) = NULL$} \Comment $O(1)$
									\State $(aSwapear \rightarrow izq) \gets (nodoNuevo \rightarrow izq)$ \Comment $O(1)$
									\State $(nodoNuevo \rightarrow izq \rightarrow padre) \gets aSwapear$ \Comment $O(1)$
									\State $(nodoNuevo \rightarrow izq) \gets NULL$ \Comment $O(1)$
								\Else
									\State $(nodoNuevo \rightarrow izq \rightarrow padre) \gets aSwapear$ \Comment $O(1)$
									\State $(aSwapear \rightarrow izq \rightarrow padre) \gets nuevoNodo$ \Comment $O(1)$
									\State $puntero(Nodoheap)$ auxiliar $\gets (nuevoNodo \rightarrow izq)$ \Comment $O(1)$
									\State $(nuevoNodo \rightarrow izq) \gets (aSwapear \rightarrow izq)$ \Comment $O(1)$
									\State $(aSwapear \rightarrow izq) \gets auxiliar$ \Comment $O(1)$
								\EndIf
							\EndIf
						\State $(nuevoNodo \rightarrow der) \gets aSwapear$ \Comment $O(1)$
					\EndIf	
					\If{$aSwapear = (it.estructura \rightarrow raiz)$}					
					\State $(aSwapear \rightarrow padre) \gets nuevoNodo$ \Comment $O(1)$
					\State $(it.estructura \rightarrow raiz) \gets nuevoNodo$	 \Comment $O(1)$
					\State $(nuevoNodo \rightarrow padre) \gets NULL$			 \Comment $O(1)$
					\Else
					\State $puntero(Nodoheap)$ abuelo $\gets (aSwapear \rightarrow padre)$ \Comment $O(1)$
						\If{$EsHijoIzquierdo?(aSwapear)$} \Comment $O(1)$
							\State $(abuelo \rightarrow izq) \gets nuevoNuevo$ \Comment $O(1)$
							\State $(nuevoNodo \rightarrow padre) \gets abuelo$ \Comment $O(1)$
							\State $(aSwapear \rightarrow padre) \gets nuevoNodo$ \Comment $O(1)$
						\Else
							\State $(abuelo \rightarrow der) \gets nuevoNuevo$ \Comment $O(1)$
							\State $(nuevoNodo \rightarrow padre) \gets abuelo$ \Comment $O(1)$
							\State $(aSwapear \rightarrow padre) \gets nuevoNodo$ \Comment $O(1)$
						\EndIf
					\EndIf
	\EndIf	
	\State $(it.estructura \rightarrow ultimo) \gets aPonerUltimo$	\Comment $O(1)$
				\If{$EsHijoIzquierdo?(it.estructura \rightarrow ultimo)$} \Comment $O(1)$
					\State $puntero(Nodoheap)$ $buscador \gets (it.estructura \rightarrow ultimo)$ \Comment $O(1)$
					\State $nat$ i $\gets 7$ \Comment $O(1)$
					\While{$(buscador \rightarrow padre) \not= NULL$ $\land i = 7$} \Comment $O(log(EC))$
						\If{$\neg EsHijoIzquierdo?(buscador)$} \Comment $O(1)$
							\State $i \gets 0$ \Comment $O(1)$
						\Else$ $ $buscador \gets (buscador \rightarrow padre)$ \Comment $O(1)$
						\EndIf
					\EndWhile
					\If{$(buscador \rightarrow padre) = NULL$} \Comment $O(1)$
						\While{$(buscador \rightarrow der) \not= NULL$} \Comment $O(log(EC))$
							\State $buscador \gets (buscador \rightarrow der)$ \Comment $O(1)$
						\EndWhile
						\State $(it.estructura \rightarrow ultimo) \gets NULL$ \Comment Libera memoria $O(1)$						
						\State $(it.estructura \rightarrow ultimo) \gets buscador$ \Comment $O(1)$
					\Else $buscador \gets (buscador \rightarrow padre \rightarrow izq)$ \Comment $O(1)$
						\While{$(buscador \rightarrow der) \not= NULL$} \Comment $O(log(EC))$
							\State $buscador \gets (buscador \rightarrow der)$ \Comment $O(1)$
						\EndWhile
						\State $(it.estructura \rightarrow ultimo) \gets NULL$ \Comment Libera memoria $O(1)$						
						\State $(it.estructura \rightarrow ultimo) \gets buscador$ \Comment $O(1)$
					\EndIf
				\Else
					\State $(it.estructura \rightarrow ultimo) \gets (it.estructura \rightarrow ultimo \rightarrow padre \rightarrow izq)$ \Comment $O(1)$
					\State $(it.estructura \rightarrow ultimo \rightarrow padre \rightarrow der) \gets NULL$ \Comment Libera memoria del nodo borrado $O(1)$
				\EndIf				
				\While {$(cambiado \rightarrow der) \not= NULL$} \Comment $O(log(EC))$
					\State $puntero(Nodoheap) minimo \gets Min(cambiado \rightarrow izq, cambiado \rightarrow der)$ \Comment $O(1)$
					\If{$((minimo \rightarrow elemento.cantCapt)$ <\hspace{0.4mm} $(cambiado \rightarrow elemento.cantCapt)) \lor (((minimo \rightarrow elemento.cantCapt)$ = $(cambiado \rightarrow elemento.cantCapt)) \land ((minimo \rightarrow elemento.id)$ <\hspace{0.4mm} $(cambiado elemento.id)))$} \Comment $O(1)$
					\State $puntero(Nodoheap) nuevoNodo \gets minimo$					
					\State $puntero(Nodoheap)$ aSwapear $\gets (nuevoNodo \rightarrow padre)$
    		\If{$(it.estructura \rightarrow ultimo) = nuevoNodo$} \Comment Swapea los Nodos $O(1)$
    			\State $it.estructura \rightarrow ultimo \gets aSwapear$ \Comment $O(1)$
    		\EndIf
			\If{$EsHijoIzquierdo?(nuevoNodo)$} \Comment $O(1)$
						\If{$(nuevoNodo \rightarrow izq) = NULL$} \Comment $O(1)$
								\State $(aSwapear \rightarrow izq) \gets NULL$ \Comment $O(1)$
							\Else
								\State $(aSwapear \rightarrow izq) \gets (nuevoNodo \rightarrow izq)$ \Comment $O(1)$
								\State $(nuevoNodo \rightarrow izq \rightarrow padre) \gets aSwapear$ \Comment $O(1)$
							\EndIf
							\If{$(nuevoNodo \rightarrow der) = NULL$} \Comment $O(1)$
								\State $(nuevoNodo \rightarrow der) \gets (aSwapear \rightarrow der)$ \Comment $O(1)$
								\If{$(aSwapear \rightarrow der) \not= NULL$} \Comment $O(1)$
								\State $(aSwapear \rightarrow der \rightarrow padre) \gets nuevoNodo$	 \Comment $O(1)$						
								\EndIf
								\State $(aSwapear \rightarrow der) \gets NULL$ \Comment $O(1)$
							\Else
								\If{$(aSwapear \rightarrow der) = NULL$} \Comment $O(1)$
									\State $(aSwapear \rightarrow der) \gets (nodoNuevo \rightarrow der)$ \Comment $O(1)$
									\State $(nodoNuevo \rightarrow der \rightarrow padre) \gets aSwapear$ \Comment $O(1)$
									\State $(nodoNuevo \rightarrow der) \gets NULL$ \Comment $O(1)$
								\Else
									\State $(nodoNuevo \rightarrow der \rightarrow padre) \gets aSwapear$ \Comment $O(1)$
									\State $(aSwapear \rightarrow der \rightarrow padre) \gets nuevoNodo$ \Comment $O(1)$
									\State $puntero(Nodoheap)$ auxiliar $\gets (nuevoNodo \rightarrow der)$ \Comment $O(1)$
									\State $(nuevoNodo \rightarrow der) \gets (aSwapear \rightarrow der)$ \Comment $O(1)$
									\State $(aSwapear \rightarrow der) \gets auxiliar$ \Comment $O(1)$
								\EndIf
							\EndIf
						\State $(nuevoNodo \rightarrow izq) \gets aSwapear$ \Comment $O(1)$
					\Else
						\If{$(nuevoNodo \rightarrow der) = NULL$} \Comment $O(1)$
								\State $(aSwapear \rightarrow der) \gets NULL$ \Comment $O(1)$
							\Else
								\State $(aSwapear \rightarrow der) \gets (nuevoNodo \rightarrow der)$ \Comment $O(1)$
								\State $(nuevoNodo \rightarrow der \rightarrow padre) \gets aSwapear$ \Comment $O(1)$
							\EndIf
							\If{$(nuevoNodo \rightarrow izq) = NULL$} \Comment $O(1)$
								\State $(nuevoNodo \rightarrow izq) \gets (aSwapear \rightarrow izq)$ \Comment $O(1)$
								\If{$(aSwapear \rightarrow izq) \not= NULL$} \Comment $O(1)$
									\State $(aSwapear \rightarrow izq \rightarrow padre) \gets nuevoNodo$	\Comment $O(1)$						
								\EndIf
								\State $(aSwapear \rightarrow izq) \gets NULL$ \Comment $O(1)$
							\Else
								\If{$(aSwapear \rightarrow izq) = NULL$} \Comment $O(1)$
									\State $(aSwapear \rightarrow izq) \gets (nodoNuevo \rightarrow izq)$ \Comment $O(1)$
									\State $(nodoNuevo \rightarrow izq \rightarrow padre) \gets aSwapear$ \Comment $O(1)$
									\State $(nodoNuevo \rightarrow izq) \gets NULL$ \Comment $O(1)$
								\Else
									\State $(nodoNuevo \rightarrow izq \rightarrow padre) \gets aSwapear$ \Comment $O(1)$
									\State $(aSwapear \rightarrow izq \rightarrow padre) \gets nuevoNodo$ \Comment $O(1)$
									\State $puntero(Nodoheap)$ auxiliar $\gets (nuevoNodo \rightarrow izq)$ \Comment $O(1)$
									\State $(nuevoNodo \rightarrow izq) \gets (aSwapear \rightarrow izq)$ \Comment $O(1)$
									\State $(aSwapear \rightarrow izq) \gets auxiliar$ \Comment $O(1)$
								\EndIf
							\EndIf
						\State $(nuevoNodo \rightarrow der) \gets aSwapear$ \Comment $O(1)$
					\EndIf	
					\If{$aSwapear = (it.estructura \rightarrow raiz)$}					
					\State $(aSwapear \rightarrow padre) \gets nuevoNodo$ \Comment $O(1)$
					\State $(it.estructura \rightarrow raiz) \gets nuevoNodo$	 \Comment $O(1)$
					\State $(nuevoNodo \rightarrow padre) \gets NULL$			 \Comment $O(1)$
					\Else
					\State $puntero(Nodoheap)$ abuelo $\gets (aSwapear \rightarrow padre)$ \Comment $O(1)$
						\If{$EsHijoIzquierdo?(aSwapear)$} \Comment $O(1)$
							\State $(abuelo \rightarrow izq) \gets nuevoNuevo$ \Comment $O(1)$
							\State $(nuevoNodo \rightarrow padre) \gets abuelo$ \Comment $O(1)$
							\State $(aSwapear \rightarrow padre) \gets nuevoNodo$ \Comment $O(1)$
						\Else
							\State $(abuelo \rightarrow der) \gets nuevoNuevo$ \Comment $O(1)$
							\State $(nuevoNodo \rightarrow padre) \gets abuelo$ \Comment $O(1)$
							\State $(aSwapear \rightarrow padre) \gets nuevoNodo$ \Comment $O(1)$
						\EndIf
					\EndIf
					\EndIf					
				\EndWhile
				\If{$(cambiado \rightarrow izq) \not= NULL$} \Comment $O(1)$
					\If{$((cambiado \rightarrow izq \rightarrow elemento.cantCapt)$ <\hspace{0.4mm} $(cambiado \rightarrow elemento.cantCapt)) \lor (((cambiado \rightarrow izq \rightarrow elemento.cantCapt)$ = $(cambiado \rightarrow elemento.cantCapt)) \implies ((cambiado \rightarrow izq \rightarrow elemento.id)$ <\hspace{0.4mm} $(cambiado elemento.id)))$} \Comment $O(1)$
						\State $puntero(Nodoheap) nuevoNodo \gets (cambiado \rightarrow izq)$						
						\State $puntero(Nodoheap)$ aSwapear $\gets (nuevoNodo \rightarrow padre)$
    		\If{$(it.estructura \rightarrow ultimo) = nuevoNodo$} \Comment Swapea los nodos $O(1)$
    			\State $it.estructura \rightarrow ultimo \gets aSwapear$ \Comment $O(1)$
    		\EndIf
			\If{$EsHijoIzquierdo?(nuevoNodo)$} \Comment $O(1)$
						\If{$(nuevoNodo \rightarrow izq) = NULL$} \Comment $O(1)$
								\State $(aSwapear \rightarrow izq) \gets NULL$ \Comment $O(1)$
							\Else
								\State $(aSwapear \rightarrow izq) \gets (nuevoNodo \rightarrow izq)$ \Comment $O(1)$
								\State $(nuevoNodo \rightarrow izq \rightarrow padre) \gets aSwapear$ \Comment $O(1)$
							\EndIf
							\If{$(nuevoNodo \rightarrow der) = NULL$} \Comment $O(1)$
								\State $(nuevoNodo \rightarrow der) \gets (aSwapear \rightarrow der)$ \Comment $O(1)$
								\If{$(aSwapear \rightarrow der) \not= NULL$} \Comment $O(1)$
								\State $(aSwapear \rightarrow der \rightarrow padre) \gets nuevoNodo$	 \Comment $O(1)$						
								\EndIf
								\State $(aSwapear \rightarrow der) \gets NULL$ \Comment $O(1)$
							\Else
								\If{$(aSwapear \rightarrow der) = NULL$} \Comment $O(1)$
									\State $(aSwapear \rightarrow der) \gets (nodoNuevo \rightarrow der)$ \Comment $O(1)$
									\State $(nodoNuevo \rightarrow der \rightarrow padre) \gets aSwapear$ \Comment $O(1)$
									\State $(nodoNuevo \rightarrow der) \gets NULL$ \Comment $O(1)$
								\Else
									\State $(nodoNuevo \rightarrow der \rightarrow padre) \gets aSwapear$ \Comment $O(1)$
									\State $(aSwapear \rightarrow der \rightarrow padre) \gets nuevoNodo$ \Comment $O(1)$
									\State $puntero(Nodoheap)$ auxiliar $\gets (nuevoNodo \rightarrow der)$ \Comment $O(1)$
									\State $(nuevoNodo \rightarrow der) \gets (aSwapear \rightarrow der)$ \Comment $O(1)$
									\State $(aSwapear \rightarrow der) \gets auxiliar$ \Comment $O(1)$
								\EndIf
							\EndIf
						\State $(nuevoNodo \rightarrow izq) \gets aSwapear$ \Comment $O(1)$
					\Else
						\If{$(nuevoNodo \rightarrow der) = NULL$} \Comment $O(1)$
								\State $(aSwapear \rightarrow der) \gets NULL$ \Comment $O(1)$
							\Else
								\State $(aSwapear \rightarrow der) \gets (nuevoNodo \rightarrow der)$ \Comment $O(1)$
								\State $(nuevoNodo \rightarrow der \rightarrow padre) \gets aSwapear$ \Comment $O(1)$
							\EndIf
							\If{$(nuevoNodo \rightarrow izq) = NULL$} \Comment $O(1)$
								\State $(nuevoNodo \rightarrow izq) \gets (aSwapear \rightarrow izq)$ \Comment $O(1)$
								\If{$(aSwapear \rightarrow izq) \not= NULL$} \Comment $O(1)$
									\State $(aSwapear \rightarrow izq \rightarrow padre) \gets nuevoNodo$	\Comment $O(1)$						
								\EndIf
								\State $(aSwapear \rightarrow izq) \gets NULL$ \Comment $O(1)$
							\Else
								\If{$(aSwapear \rightarrow izq) = NULL$} \Comment $O(1)$
									\State $(aSwapear \rightarrow izq) \gets (nodoNuevo \rightarrow izq)$ \Comment $O(1)$
									\State $(nodoNuevo \rightarrow izq \rightarrow padre) \gets aSwapear$ \Comment $O(1)$
									\State $(nodoNuevo \rightarrow izq) \gets NULL$ \Comment $O(1)$
								\Else
									\State $(nodoNuevo \rightarrow izq \rightarrow padre) \gets aSwapear$ \Comment $O(1)$
									\State $(aSwapear \rightarrow izq \rightarrow padre) \gets nuevoNodo$ \Comment $O(1)$
									\State $puntero(Nodoheap)$ auxiliar $\gets (nuevoNodo \rightarrow izq)$ \Comment $O(1)$
									\State $(nuevoNodo \rightarrow izq) \gets (aSwapear \rightarrow izq)$ \Comment $O(1)$
									\State $(aSwapear \rightarrow izq) \gets auxiliar$ \Comment $O(1)$
								\EndIf
							\EndIf
						\State $(nuevoNodo \rightarrow der) \gets aSwapear$ \Comment $O(1)$
					\EndIf	
					\If{$aSwapear = (it.estructura \rightarrow raiz)$}					
					\State $(aSwapear \rightarrow padre) \gets nuevoNodo$ \Comment $O(1)$
					\State $(it.estructura \rightarrow raiz) \gets nuevoNodo$	 \Comment $O(1)$
					\State $(nuevoNodo \rightarrow padre) \gets NULL$			 \Comment $O(1)$
					\Else
					\State $puntero(Nodoheap)$ abuelo $\gets (aSwapear \rightarrow padre)$ \Comment $O(1)$
						\If{$EsHijoIzquierdo?(aSwapear)$} \Comment $O(1)$
							\State $(abuelo \rightarrow izq) \gets nuevoNuevo$ \Comment $O(1)$
							\State $(nuevoNodo \rightarrow padre) \gets abuelo$ \Comment $O(1)$
							\State $(aSwapear \rightarrow padre) \gets nuevoNodo$ \Comment $O(1)$
						\Else
							\State $(abuelo \rightarrow der) \gets nuevoNuevo$ \Comment $O(1)$
							\State $(nuevoNodo \rightarrow padre) \gets abuelo$ \Comment $O(1)$
							\State $(aSwapear \rightarrow padre) \gets nuevoNodo$ \Comment $O(1)$
						\EndIf
					\EndIf
					\EndIf
				\EndIf
			\EndIf
			\medskip
			\Statex \underline{Complejidad:} $O(log(EC))$
			\Statex \underline{Justificaci\'on:} Todas las operaciones son $O(1)$ y en cada uno de los ciclos se va recorriendo la cola desde una de las hojas hasta la raiz (o vice versa). Al ser la cola un \'arbol completo (ya que siempre se agregan elementos al \'ultimo lugar disponible), si la cantidad de elementos es EC en el peor cas, entonces su altura es log(EC). Por lo tanto estos ciclos se repiten log(EC) veces en el peor caso.
    	\end{algorithmic}
    	
\begin{algorithm}[H]{\textbf{iEsHijoIzquierdo?}(\In{p}{puntero(Nodoheap)}) $\to$ $res$ : bool} $ $\newline
\Comment Esta funci\'on es privada y devuelve true si el nodo apuntado es hijo izquierdo de su padre. Pre: el padre del nodo tiene que ser $\not=$ NULL
    	\begin{algorithmic}[1] 
			 \State $res \gets  (p = p \rightarrow padre \rightarrow izq) $ \Comment $O(1)$

			\medskip
			\Statex \underline{Complejidad:} $O(1)$
			\Statex \underline{Justificaci\'on:} Todas las operaciones son $O(1)$
    	\end{algorithmic}   	
\end{algorithm}

\begin{algorithm}[H]{\textbf{iMin}(\In{izq}{puntero(Nodoheap)}, \In{der}{puntero(Nodoheap)}) $\to$ $res$ : puntero(Nodoheap)}  $ $\newline
\Comment Esta funci\'on es privada y devuelve el nodo con menor cantidad de pokemones capturados, y en caso de empate, devuelve al que tiene menor id.
    	\begin{algorithmic}[1] 
			 \If{$((izq \rightarrow elemento.cantCapt)$ <\hspace{0.4mm} $(der \rightarrow elemento.cantCapt)) \lor (((izq \rightarrow elemento.cantCapt)$ = $(der \rightarrow elemento.cantCapt)) \land ((izq \rightarrow elemento.id)$ <\hspace{0.4mm} $(der \rightarrow elemento.id)))$} \Comment $O(1)$
			 	\State $res \gets izq$ \Comment $O(1)$
			 \Else
			 	\State $res \gets der$ \Comment $O(1)$
			 \EndIf		
			\medskip
			\Statex \underline{Complejidad:} $O(1)$
			\Statex \underline{Justificaci\'on:} Todas las operaciones son $O(1)$ 
    	\end{algorithmic}   	
\end{algorithm}

\begin{algorithm}[H]{\textbf{iCrearItCola}(\In{h}{colaEntr}, \In{p}{puntero(Nodoheap)}) $\to$ $res$ : itcolaEntrenador)} $ $\newline
\Comment Funci\'on privada que devuelve un iterador a un elemento de la cola de prioridad.
    	\begin{algorithmic}[1]
    		 \State $puntero(crearEntr) q \gets h$
			 \State $res \gets  <p, q> $ \Comment $O(1)$

			\medskip
			\Statex \underline{Complejidad:} $O(1)$
			\Statex \underline{Justificaci\'on:} Todas las operaciones son $O(1)$
			Pre: el puntero es $\not=$ NULL, tiene que apuntar a un nodo perteneciente a la estructura de la cola de entrenadores. La cola es no vac\'ia Post: se devuelve un iterador a dicho nodo.
    	\end{algorithmic}
\end{algorithm}


\end{Algoritmos}

\pagebreak

\section{TAD Iterador Cola}

\begin{tad}{\tadNombre{Iterador Cola de Entrenadores($\alpha$)}}
\tadGeneros{itcola($\alpha$)}


\tadIgualdadObservacional{it}{it'}{itcola($\alpha$)}{$siguiente(it) \igobs siguiente(it')$}

\tadExporta{itcola($\alpha$), generadores, observadores}
\tadUsa{\tadNombre{Conjunto($\alpha$)}, \tadNombre{Cola De Prioridad($\alpha$)}}


\tadAlinearFunciones{elementos}{itcola($\alpha$) $it$, colaPrior($\alpha$) cp}{colaPrior($\alpha$)}{}

\tadObservadores

\tadOperacion{siguiente}{itcola}{$\alpha$}{}



\tadGeneradores
\tadOperacion{crearIt}{colaPrior($\alpha$), $\alpha$}{itcola($\alpha$)}{}


\tadOtrasOperaciones

\tadOperacion{borrar}{itcola($\alpha$) $it$, colaPrior($\alpha$) cp}{colaPrior($\alpha$)}{siguiente(it) $\in$ elementos(cp)}

\tadOperacion{elementos}{colaPrior($\alpha$)}{conj($\alpha$)}{}

\tadOperacion{agregarSin}{$\alpha$, conj($\alpha$)}{colaPrior($\alpha$)}{}

\tadAxiomas[\paratodo{colaPrior($\alpha$)}{cp, sp},  \paratodo{$\alpha$}{e}, \paratodo{Conj($\alpha$)}{con}] 

\tadAlinearAxiomas{damePokemon(crearPknodo($pk$, c))~}


$ $\newline
\tadAxioma{siguiente(crearIt($cp$, e))}{e}

$ $\newline
\tadAxioma{borrar(crearIt($cp$, e), $sp$)}{agregarSin(e, elementos(sp))}

$ $\newline
\tadAxioma{elementos($sp$)}
{
\IF
vacia?(sp)
THEN
$\emptyset$
ELSE
Ag(proximo(sp), elementos(desencolar(sp)))
FI
}

$ $\newline
\tadAxioma{agregarSin(e, con))}
{
\IF
$\emptyset$?(con)
THEN
vacia
ELSE
	{\IF
	dameUno(con) = e
	THEN
	agregarSin(e, sinUno(conj))
	ELSE
	encolar(dameUno(con), agregarSin(e, sinUno(con)))
	FI}
FI
}


\end{tad}

\end{document}