\documentclass[a4paper,10pt]{article}
\usepackage[paper=a4paper, hmargin=1.5cm, bottom=1.5cm, top=3.5cm]{geometry}
\usepackage[latin1]{inputenc}
\usepackage[T1]{fontenc}
\usepackage[spanish]{babel}
\usepackage{fancyhdr}
\usepackage{lastpage}
\usepackage{xspace}
\usepackage{xargs}
\usepackage{calc}
\usepackage{ifthen}
\usepackage{aed2-symb,aed2-itef,aed2-tad, caratula}
\usepackage{algorithmicx, algpseudocode, algorithm}
\usepackage[colorlinks=true, linkcolor=blue]{hyperref}


% Estos comandos deben ir antes del \maketitle
\materia{Algoritmos y Estructuras de Datos II} 
\submateria{Segundo Cuatrimestre de 2016} 
\titulo{Trabajo Pr\'actico 2} 
\subtitulo{Dise\~no}
\grupo{Grupo: 4 gigas de RAM}

\integrante{Jonathan Seijo}{592/15}{jon.seijo@gmail.com} % obligatorio 
\integrante{Lucas Mauricio C\'ordoba}{094/15}{lmcordobaa@gmail.com} % obligatorio 
\integrante{Lucas Gabriel De Bortoli}{736/15}{lu\_cas\_.97@hotmail.com.ar} % obligatorio 
\integrante{Luciano Galli}{534/15}{lucianogalli@outlook.com} % obligatorio 

\pagestyle{fancy}
\thispagestyle{fancy}
\addtolength{\headheight}{1pt}
\lhead{Algoritmos y Estructuras de Datos II}
\rhead{Trabajo Pr\'actico 2}

%%%%%%%%%%%%%%%%%%%%%%%%%%%%%%%%%%%%%%%%%%%%%%%%%%%%%%%%%%%%
% COMANDOS QUE ALGUN DIA PUEDAN FORMAR UN PAQUETE.
%%%%%%%%%%%%%%%%%%%%%%%%%%%%%%%%%%%%%%%%%%%%%%%%%%%%%%%%%%%%
\newcommand{\moduloNombre}[1]{\textbf{#1}}

\let\NombreFuncion=\textsc
\let\TipoVariable=\texttt
\let\ModificadorArgumento=\textbf
\newcommand{\res}{$res$\xspace}
\newcommand{\tab}{\hspace*{7mm}}

\newcommandx{\TipoFuncion}[3]{%
  \NombreFuncion{#1}(#2) \ifx#3\empty\else $\to$ \res\,: \TipoVariable{#3}\fi%
}
\newcommand{\In}[2]{\ModificadorArgumento{in} \ensuremath{#1}\,: \TipoVariable{#2}\xspace}
\newcommand{\Out}[2]{\ModificadorArgumento{out} \ensuremath{#1}\,: \TipoVariable{#2}\xspace}
\newcommand{\Inout}[2]{\ModificadorArgumento{in/out} \ensuremath{#1}\,: \TipoVariable{#2}\xspace}
\newcommand{\Aplicar}[2]{\NombreFuncion{#1}(#2)}

\newlength{\IntFuncionLengthA}
\newlength{\IntFuncionLengthB}
\newlength{\IntFuncionLengthC}
%InterfazFuncion(nombre, argumentos, valor retorno, precondicion, postcondicion, complejidad, descripcion, aliasing)
\newcommandx{\InterfazFuncion}[9][4=true,6,7,8,9]{%
  \hangindent=\parindent
  \TipoFuncion{#1}{#2}{#3}\\%
  \textbf{Pre} $\equiv$ \{#4\}\\%
  \textbf{Post} $\equiv$ \{#5\}%
  \ifx#6\empty\else\\\textbf{Complejidad:} #6\fi%
  \ifx#7\empty\else\\\textbf{Descripci\'on:} #7\fi%
  \ifx#8\empty\else\\\textbf{Aliasing:} #8\fi%
  \ifx#9\empty\else\\\textbf{Requiere:} #9\fi%
}

\newenvironment{Interfaz}{%
  \parskip=2ex%
  \noindent\textbf{\Large Interfaz}%
  \par%
}{}

\newenvironment{Representacion}{%
  \vspace*{2ex}%
  \noindent\textbf{\Large Representaci\'on}%
  \vspace*{2ex}%
}{}

\newenvironment{Algoritmos}{%
  \vspace*{2ex}%
  \noindent\textbf{\Large Algoritmos}%
  \vspace*{2ex}%
}{}


\newcommand{\Title}[1]{
  \vspace*{1ex}\par\noindent\textbf{\large #1}\par
}

\newenvironmentx{Estructura}[2][2={estr}]{%
  \par\vspace*{2ex}%
  \TipoVariable{#1} \textbf{se representa con} \TipoVariable{#2}%
  \par\vspace*{1ex}%
}{%
  \par\vspace*{2ex}%
}%

\newboolean{EstructuraHayItems}
\newlength{\lenTupla}
\newenvironmentx{Tupla}[1][1={estr}]{%
    \settowidth{\lenTupla}{\hspace*{3mm}donde \TipoVariable{#1} es \TipoVariable{tupla}$($}%
    \addtolength{\lenTupla}{\parindent}%
    \hspace*{3mm}donde \TipoVariable{#1} es \TipoVariable{tupla}$($%
    \begin{minipage}[t]{\linewidth-\lenTupla}%
    \setboolean{EstructuraHayItems}{false}%
}{%
    $)$%
    \end{minipage}
}

\newcommandx{\tupItem}[3][1={\ }]{%
    %\hspace*{3mm}%
    \ifthenelse{\boolean{EstructuraHayItems}}{%
        ,#1%
    }{}%
    \emph{#2}: \TipoVariable{#3}%
    \setboolean{EstructuraHayItems}{true}%
}

\newcommandx{\RepFc}[3][1={estr},2={e}]{%
  \tadOperacion{Rep}{#1}{bool}{}%
  \tadAxioma{Rep($#2$)}{#3}%
}%

\newcommandx{\Rep}[3][1={estr},2={e}]{%
  \tadOperacion{Rep}{#1}{bool}{}%
  \tadAxioma{Rep($#2$)}{true \ssi #3}%
}%

\newcommandx{\Abs}[5][1={estr},3={e}]{%
  \tadOperacion{Abs}{#1/#3}{#2}{Rep($#3$)}%
  \settominwidth{\hangindent}{Abs($#3$) \igobs #4: #2 $\mid$ }%
  \addtolength{\hangindent}{\parindent}%
  Abs($#3$) \igobs #4: #2 $\mid$ #5%
}%

\newcommandx{\AbsFc}[4][1={estr},3={e}]{%
  \tadOperacion{Abs}{#1/#3}{#2}{Rep($#3$)}%
  \tadAxioma{Abs($#3$)}{#4}%
}%


\newcommand{\DRef}{\ensuremath{\rightarrow}}



\begin{document}

\maketitle

\thispagestyle{fancy}
\tableofcontents


\pagebreak

\section{Informe}

\subsection{Juego}

\begin{itemize}
\item{La operaci\'on cantMismaEspecie de la especificaci\'on recive como parametro un multiconjunto. 
Reemplazamos ese parametro por un $juego$, porque usando el $juego$ podemos obtener la cantidad de cada especie pokemon}

\end{itemize}

\subsection{Cola de Prioridad}

\begin{itemize}
\item{En la cola de entrenadores, se usand nodos y punteros para la estructura. Los nodos se mantienen ``fijos'' una vez que se agregar, hasta que son borrados. Cada vez que hay que hacer alg\'un cambio o ``swap'' lo \'unico que se modifican son los punteros ``padre'', ``izq'' y ``der''. Es decir que si hay alg\'un puntero apuntando al nodo y se realiza un swap entre ese nodo y otro, dicho puntero seguir\'a apunt\'andolo.}

\end{itemize}


\subsection{Algoritmos privados}

\begin{itemize}

\item{Se inlcuye pre y post en castellano de los algoritmos privados que hacen manejo de memoria.}

\end{itemize}


\subsection{IterDiccString}

\begin{itemize}

\item{Si bien iterDiccString se explica con IteradorUnidireccional, hicimos un cambio en la aridad de ``siguiente''. En su TAD, el tipo que se devuelve es del mismo tipo que recibe en $crear$, pero nosotros cambiamos eso para que devuelva las tuplas que al usuario le interesan (particularmente en la funcion Pokemons() del juego)}

\end{itemize}




\pagebreak

\section{Coordenada}
\begin{Interfaz}


  \textbf{se explica con}: \tadNombre{Coordenada}.

  \textbf{g\'eneros}: \TipoVariable{coor}.

  \Title{Operaciones b\'asicas de Coordenada}


  \InterfazFuncion{crearCoor}{\In{n_1}{nat}, \In{n_2}{nat}}{coor}%
  {$res \igobs crearCoor$}%
  [$O(1)$]
  [genera una coordenada nueva.]


  \InterfazFuncion{latitud}{\In{c}{coor}}{nat}
  {$res \igobs latitud(c)$}
  [$O(1)$]
  [devuelve la latitud de la coordenada \texttt{c}.]

  \InterfazFuncion{longitud}{\In{c}{coor}}{nat}
  {$res \igobs longitud(c)$}
  [$O(1)$]
  [devuelve la longitud de la coordenada \texttt{c}.]


  \InterfazFuncion{distEuclidea}{\In{c_1}{coor}, \In{c_2}{coor}}{nat}%
  {$res \igobs distEuclidea(c_1, c_2)$}%
  [$O(1)$]
  [devuelve la disntancia entre la coordenadas \texttt{$c_1$} y \texttt{$c_2$}.]


\end{Interfaz}

\begin{Representacion}
 
  \begin{Estructura}{coor}[estr]
      \begin{Tupla}[estr]%
      \tupItem{latitud}{nat}%
      \tupItem{longitud}{nat}%
    \end{Tupla}
  \end{Estructura}

  \RepFc[estr$/e$]{true}

  ~

  \AbsFc[estr]{coor}[e]{c : coor / e.latitud = latitud(c) $\land$ e.longitud = longitud(c)}

\end{Representacion}

\begin{Algoritmos}
   
\begin{algorithm}[H]{\textbf{crearCoor}(\In{n_1}{nat}, \In{n_2}{nat}) $\to$ $res$ : coor}
    	\begin{algorithmic}[1]
			 \State $res$ $\gets$  $\langle$ $n_1$ , $n_2$ $\rangle$ \Comment $O(1)$

			\medskip
			\Statex \underline{Complejidad:} $O(1)$
    	\end{algorithmic}
\end{algorithm}
   
\begin{algorithm}[H]{\textbf{latitud}(\In{c}{coor}) $\to$ $res$ : nat}
    	\begin{algorithmic}[1]
			 \State $res$ $\gets$  $c$.$latitud$ \Comment $O(1)$

			\medskip
			\Statex \underline{Complejidad:} $O(1)$
			
    	\end{algorithmic}
\end{algorithm}


\begin{algorithm}[H]{\textbf{longitud}(\In{c}{coor}) $\to$ $res$ : nat}
    	\begin{algorithmic}[1]
			 \State $res$ $\gets$  $c$.$longitud$ \Comment $O(1)$

			\medskip
			\Statex \underline{Complejidad:} $O(1)$
    	\end{algorithmic}
\end{algorithm}

\begin{algorithm}[H]{\textbf{distEuclidea}(\In{c_1}{coor}, \In{c_2}{coor}) $\to$ $res$ : nat}
    	\begin{algorithmic}[1]
    		 \State $a$ $\gets$ $0$ \Comment $O(1)$
    		 \If{ $c_1$.$latitud$ $<$ $c_2$.$latitud$ }	\Comment $O(1)$
				\State $a$ $\gets$ ($c_1$.$latitud$ $-$ $c_2$.$latitud$) $\times$ ($c_1$.$latitud$ $-$ $c_2$.$latitud$) \Comment $O(1)$	
			\Else 
				\State $a$ $\gets$ ($c_2$.$latitud$ $-$ $c_1$.$latitud$) $\times$ ($c_2$.$latitud$ $-$ $c_1$.$latitud$) \Comment $O(1)$	
			\EndIf
			\State $b$ $\gets$ $0$ \Comment $O(1)$

			\If{ $c_1$.$longitud$ $<$ $c_2$.$longitud$ }	\Comment $O(1)$
				\State $b$ $\gets$ ($c_1$.$longitud$ $-$ $c_2$.$longitud$) $\times$ ($c_1$.$longitud$ $-$ $c_2$.$longitud$)  \Comment $O(1)$	
			\Else 
				\State $b$ $\gets$ ($c_2$.$longitud$ $-$ $c_1$.$longitud$) $\times$ ($c_2$.$longitud$ $-$ $c_1$.$longitud$) \Comment $O(1)$	
			\EndIf
			\State $res$ $\gets$  $a$ $+$ $b$ \Comment $O(1)$

			\medskip
			\Statex \underline{Complejidad:} $O(1)$
			\Statex \underline{Justificacion:} $O(1)$ $+$ $O(1)$ $+$ $O(1)$ $+$ $O(1)$ $+$ $O(1)$ $+$ $O(1)$ $+$ $O(1)$
    	\end{algorithmic}
\end{algorithm}

\end{Algoritmos}

\pagebreak

\section{Mapa}
\begin{Interfaz}


\textbf{se explica con}: \tadNombre{Mapa}.

\textbf{g\'eneros}: \TipoVariable{map}.

\Title{Operaciones b\'asicas de Mapa}


\InterfazFuncion{crearMapa}{}{map}%
{$res \igobs crearMapa$}%
[$O(1)$]
[genera una mapa vac\'io.]


\InterfazFuncion{agregarCoor}{ \In{c}{coord}, \Inout{m}{map}}{}
[$m_0 \igobs m \land \lnot posExistente(c, m_0) $]  
{$m \igobs agregarCoor(c, m_0)$}
[$O(  max(latitud(c), longitud(c))^4  )$]
[agrega la coordenada \texttt{c} al mapa \texttt{m}]


\InterfazFuncion{coordenadas}{\In{m}{map}}{$conj(coord)$}%
{$res \igobs coordenadas(m)$}%
[$O(tam(m))$]  % Algoritmo: recorro la diagonal principal
[devuelve el conjunto de todas las coordenadas de m.]


\InterfazFuncion{posExistente}{ \In{c}{coord}, \In{m}{map}}{bool}
{$res \igobs posExistente(c, m)$}
[$O(1)$]
[verifica si la coordenada \texttt{c} existe en el mapa \texttt{m}]


\InterfazFuncion{hayCamino}{ \In{c_1}{coord}, \In{c_2}{coord}, \In{m}{map}}{bool}
[$posExistente(c_1, m) \land posExistente(c_2, m) $]
{$res \igobs hayCamino(c_1, c_2, m)$}
[$O(1)$]
[verifica si existe una forma de llegar desde \texttt{$c_1$} a \texttt{$c_2$}]


\end{Interfaz}

\begin{Representacion}

\begin{Estructura}{mapa}[map]
	\begin{Tupla}[map]  
		\tupItem{tam}{nat}
		\tupItem{grilla}{vector(vector(vector(vector(bool)))}  
  	\end{Tupla}
  	
\end{Estructura}

\RepFc[]{true}



  

  \AbsFc[]{}

\end{Representacion}

\begin{Algoritmos}


% crearMapa
\begin{algorithm}[H]
{\textbf{iCrearMapa}()$\to$ $res$ : map}
\begin{algorithmic}[1]
   
  							% Como digo que aca va un vector(vector(v.... vacio? 	
\State $res \gets  \langle 0 , Vacio() \rangle $     COMO SABER QUE VACIO ES UN VECTOR?  \Comment Crear vector vacio es O(1) // $O(1)$

\medskip
\Statex \underline{Complejidad:} $O(1)$

\end{algorithmic}
\end{algorithm}

% coordenadas
\begin{algorithm}[H]
{\textbf{iCoordenadas}(\In{m}{map})$\to$ $res$ : conj(coor)}
\begin{algorithmic}[1]
   
 
\State $res \gets Vacio() $     COMO SABER QUE VACIO ES UN CTO?  \Comment Crear conjunto vacio es O(1) // $O(1)$

\For{$i=0$ to $tam(m)$}
	
	\For{$j=0$ to $tam(m)$}
		
		\If{iPosExistente(i, j)}  \Comment $O(1)$
			
			\State $c \gets NuevaCoor(i, j) $  \Comment $O(1)$			
			\State $Agregar(res, c) $  \Comment $O(log(\#res))$			
			
		\EndIf
		
	\EndFor

\EndFor


\medskip
\Statex \underline{Complejidad:} $O(log(tam(m))*(tam(m))^2)$

\Statex \underline{Justificacion:} Son dos for anidados, por cada iteracion externa, se ejecutan $tam(m)$ iteraciones internas. En total son $(tam(m))^2$ iteraciones. En cada una de ellas, puede ejecutarse el $Agregar(rec, c)$ con complejidad $O(log(\#res))$. $res$ solo contiene coordenadas recorridas por la iteracion, por lo que $\#res <= tam(m)$, es decir $O(log(\#res)) = O(log(tam(m)))$, por lo que la complejidad final es $O(log(tam(m))*(tam(m))^2)$.

\end{algorithmic}
\end{algorithm}

% posExistente

% EXPLICACION:
% CUANDO AGREGO UNA COORDENADA, LA CONECTO EN LA GRILLA CONSIGO MISMA, SIGNIFICANDO QUE EXISTE 
\begin{algorithm}[H]
{\textbf{iPosExistente}(\In{m}{map}, \In{c}{coor})$\to$ $res$ : bool}
\begin{algorithmic}[1]
   

\If{latitud(c) $>=$ tam(m) OR longitud(c) $>=$ tam(m)}
	\State $res \gets false$ \Comment $O(1)$
\Else
	\State $x \gets latitud(c)$ \Comment $O(1)$
	\State $y \gets longitud(c)$ \Comment $O(1)$
	\State $res \gets m.grilla[x][y][x][y]$ \Comment $O(1)$
\EndIf


\medskip
\Statex \underline{Complejidad:} $O(1)$


\end{algorithmic}
\end{algorithm}

% hayCamino

\begin{algorithm}[H]
{\textbf{iHayCamino}(\In{m}{map}, \In {c1}{coor}, \In{c2}{coor})$\to$ $res$ : bool}
\begin{algorithmic}[1]

\State $res \gets grills[latitud(c1)][longitud(c1)][latitud(c2)][longitud(c1)]$ \Comment $O(1)$

\medskip
\Statex \underline{Complejidad:} $O(1)$


\end{algorithmic}
\end{algorithm}


% agregaCoor


\begin{algorithm}[H]
{\textbf{iAgregaCoor}(\Inout{m}{map}, \In {c}{coor})}
\begin{algorithmic}[1]

\State $maximo \gets max(latitud(c), longitud(c)) $ \Comment $O(1)$

% Si hay una coordenada mayor al maximo
\State \If{maximo $>$ m.tam} \Comment $O(1)$

\State $nGrilla = iCrearGrilla(maximo)$ \Comment $O(maximo^4)$
\State $iCopiarCoordenadas(nGrilla, m.grilla)$ \Comment $O((maximo^4)$
\State $m.grilla = nGrilla$ \Comment $O(maximo^4)$
\State $m.tam = maximo$ \Comment $O(1)$

\EndIf

% La marco consigo misma para decir que la cooordenada existe
\State $m.grilla[latitud(c)][longitud(c)][latitud(c)][longitud(c)] = true$ \Comment $O(1)$
\State $ $

% Creo una grilla de visitadas, todas false
%Vector vacio ver como escribirlo bien
\State $visitados \gets Vacio() $ \Comment $O(1)$
\For{$i = 0$ to $m.tam$}	    \Comment $O(infinito)$
\State $visitadosAux \gets Vacio() $ \Comment $O(1)$	
\For{$j = 0$ to $m.tam$}      \Comment $O(Inifinito)$
\State $visitadosAux.AgregarAtras(false)$ \Comment $O(1)$
\EndFor
\State $visitados.AgregarAtras(visitadosAux)$ \Comment $O(1)$
\EndFor


% aRecorrer es una cola..
\State $aRecorrer \gets Vacio() $ \Comment $O(1)$
\State $aRecorrer.Encolar(c)$ \Comment $O(1)$

\While{$\neg aRecorrer.EsVacia()$}



\EndWhile

\medskip
\Statex \underline{Complejidad:} $O((max(maximo, m.tam))^4)$
\Statex \underline{Justificacion:} Trabajo con matrices de 4 dimensiones, en peor caso hay que redimensionar, en donde se recorren todos los elementos


\end{algorithmic}
\end{algorithm}




  
\end{Algoritmos}

\end{document}