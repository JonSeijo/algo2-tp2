
\subsection{Juego}

\begin{itemize}
\item{La operaci\'on cantMismaEspecie de la especificaci\'on recive como parametro un multiconjunto. 
Reemplazamos ese parametro por un $juego$, porque usando el $juego$ podemos obtener la cantidad de cada especie pokemon}

\end{itemize}

\subsection{Cola de Prioridad}

\begin{itemize}
\item{En la cola de entrenadores, se usand nodos y punteros para la estructura. Los nodos se mantienen ``fijos'' una vez que se agregar, hasta que son borrados. Cada vez que hay que hacer alg\'un cambio o ``swap'' lo \'unico que se modifican son los punteros ``padre'', ``izq'' y ``der''. Es decir que si hay alg\'un puntero apuntando al nodo y se realiza un swap entre ese nodo y otro, dicho puntero seguir\'a apunt\'andolo.}

\end{itemize}


\subsection{Algoritmos privados}

\begin{itemize}

\item{Se inlcuye pre y post en castellano de los algoritmos privados que hacen manejo de memoria.}

\end{itemize}

