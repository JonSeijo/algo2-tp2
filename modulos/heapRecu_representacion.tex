\begin{Representacion}
\subsection{Representacion de Cola de Entrenadores}
 
AHORA SOLO COPIE LA ESTRUCTURA QUE TENIAMOS, HAY QUE VER SI TENEMOS QUE REPENSARLA 
 
\begin{Estructura}{colaEntr}[estr]
    \begin{Tupla}[estr]%
    \tupItem{raiz}{puntero(Nodoheap)}%
    \tupItem{ultimo}{puntero(Nodoheap)}%
  \end{Tupla}
\end{Estructura}

\begin{Tupla}[Nodoheap]
	\tupItem{elemento}{entrenador}%
	\tupItem{padre}{puntero(Nodoheap)}%
	\tupItem{izq}{puntero(Nodoheap)}%
	\tupItem{der}{puntero(Nodoheap)}%
\end{Tupla}

\begin{Tupla}[entrenador]
	\tupItem{id}{nat}%
	\tupItem{cantCapt}{nat}%
\end{Tupla}


\Title{Invariante de representaci\'on en castellano}	
	
HACER

\tadOperacion{Abs}{estr$/c$}{colaPrior(Tupla(nat, nat))}{Rep($d$)}%
\tadAxioma{Abs($c$)}
{
%$cola$ : colaPrior(Tupla(nat, nat)) / \\
%	vacia?($cola$)  $\iff$  $c$.$raiz$ $=$ $NULL$ $\yluego$ \\
%	proximo($cola$) $\igobs$ $c$.(*$raiz$).elemento \\

	
}%

\end{Representacion}
