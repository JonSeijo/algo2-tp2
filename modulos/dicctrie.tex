
\begin{Interfaz}
\subsection{Interfaz}

  \textbf{par\'ametros formales}\hangindent=2\parindent\\
  \parbox{1.7cm}{\textbf{g\'eneros}} $\alpha$


  \textbf{se explica con}: \tadNombre{dicc(string, $\alpha$)}, \tadNombre{conj(string)}.

  \textbf{g\'eneros}: \TipoVariable{diccString($\alpha$)}.

  \InterfazFuncion{Vacio}{}{diccString($\alpha$)}
  {$res$ $\igobs$ vac\'io}
  [$O(1)$]
  [genera un diccionario vac\'io.]

  \InterfazFuncion{Definir}{\Inout{d}{diccString($\alpha$)}, \In{s}{string}, \In{a}{$\alpha$}}{}
  [$d$ $\igobs$ $d_0$]
  {$d$ $\igobs$ definir($p$, $a$, $d$)}
  [$O(|P| + copiar(a))$ siendo P la clave mas larga.]
  [define $a$ en $d$ con la clave $s$.]
  [el elemento $a$ se define por copia.]

  \InterfazFuncion{Def?}{\In{p}{string}, \In{d}{diccString($\alpha$)}}{bool}
  {$res$ $\igobs$ def?($p$, $d$)}
  [$O(|P|)$ siendo P la clave mas larga.]
  [devuelve \texttt{true} si y s\'olo si la $p$ tiene una definicion en $d$.]

  \InterfazFuncion{Obtener}{\In{p}{string}, \In{d}{diccString($\alpha$)}}{$\alpha$}
  [def?($s$, $d$)]
  {alias($res$ $\igobs$ obtener($p$, $d$))}
  [$O(|P|)$ siendo P la clave mas larga.]
  [devuelve el significado de la la clave $p$ en $d$.]
  [$res$ es modificable si y s\'olo si $d$ es modificable.]

  \InterfazFuncion{Borrar}{\In{p}{string}, \Inout{d}{diccString($\alpha$)}}{}
  [$d$ $\igobs$ $d_0$ $\land$ def?($s$, $d$)]
  {$d$ $\igobs$ borrar($p$, $d_0$)}
  [$O(|P|)$ siendo P la clave mas larga.]
  [borra la clave $p$ y su significado.]

  \InterfazFuncion{Claves}{\In{d}{diccString($\alpha$)}}{conj(string)}
  [true]
  {alias($res$ $\igobs$ claves($d$))}
  [$O(1)$]
  [Devuelve del conjunto de claves de $d$.]
  {Aliasing: $res$ se modifica sii claves($d$) se modifica}



\end{Interfaz}

\pagebreak

\begin{Representacion}
\subsection{Representacion}

  \Title{Representaci\'on del diccString}

	En este m\'odulo usamos un trie para definir un diccionario cuyas claves son string.
$ $\newline La idea es que la complejidad de definir y obtener no dependa de la cantidad de claves, si no de la longitud de la clave. Bajando asi la complejidad en peor caso.

  \begin{Estructura}{diccString($\alpha$)}[estr]
    \begin{Tupla}[estr]
      \tupItem{raiz}{puntero(nodo)}%
      \tupItem{claves}{conj(string)}%
    \end{Tupla}

    \begin{Tupla}[nodo]
      \tupItem{definicion}{puntero($\alpha$)}%
      \tupItem{\\ siguientes}{arreglo(puntero(nodo))}%
      \tupItem{\\ itClave}{puntero(itConj(string))}%
    \end{Tupla}
  \end{Estructura}


\Title{Invariante de representacion en castellano}

- Si raiz es $NULL$ entonces el conjunto de claves es vac\'io. \\
- Si la raiz no es $NULL$ entonces el conjunto de claves es no vac\'io. \\
- Todos los elementos del conjunto de claves estan definidos en el diccString. \\
- Ningun nodo tiene entre sus siguientes a un nodo que esta ``antes'' que \'el \\

\tadOperacion{Abs}{estr$/d$}{dicc(string, $\alpha$)}{Rep($d$)}%
\tadAxioma{Abs($d$)}
{
$dic$ : dicc(string, $\alpha$) / \\
	claves($dic$) $\igobs$ $d$.claves $\land$ \\

	($\forall s$: string)
		((def?($s$, $dic$) $\igobs$ $s \in d$.claves)  $\yluego$ \\
		(def?($s$, $dic$) $\impluego$ obtener($s$, $dic$) $\igobs$ significado de s en la estructura))
}%

\end{Representacion}

\begin{Algoritmos}
\subsection{Algoritmos}

%crear diccTrie
\begin{algorithm}[H]
{\textbf{iVacio}() $\to$ $res$ : diccString($\alpha$)}
\begin{algorithmic}[1]

\State puntero(nodo) $iRaiz$ $\gets $ $NULL$  \Comment $O(1)$

\State conj($string$) $iClaves$ $\gets$ Vacio() \Comment $O(1)$

\State $res$ $\gets$  $\langle$ $iRaiz$, $iClaves$ $\rangle$ \Comment $O(1)$

\medskip
\State \underline{Complejidad:} $O(1)$
\State \underline{Justificaion:} $O(1)$ $+$ $O(1)$ $+$ $O(1)$

\end{algorithmic}
\end{algorithm}

%CLAVES
\begin{algorithm}[H]
{\textbf{iClaves}(\In{d}{diccString($\alpha$)}}) $\to$ $res$ : conj($\alpha$)
\begin{algorithmic}[1]

\State $res$ $\gets$ $d$.$claves$ \Comment $O(1)$

\medskip
\State \underline{Complejidad:} $O(1)$

\end{algorithmic}
\end{algorithm}



%OBTENER
\begin{algorithm}[H]
{\textbf{iObtener}(\In{p}{string}, \In{d}{diccString($\alpha$)}}) $\to$ $res$ : $\alpha$
\begin{algorithmic}[1]

\State puntero(nodo) $n$ $\gets$ $raiz$ \Comment $O(1)$

\State nat $i$ $\gets$ $0$ \Comment $O(1)$


\While{$i$ $<$ Longitud($p$)}   \Comment Se repite $|p|$ $O(1)$


	\State $actual$ $\gets$ (*$actual$).$siguientes$[ord($p$[$i$])] \Comment $O(1)$

	\State $i$ $\gets$ $i + 1$ \Comment $O(1)$
\EndWhile


\State $res$ $\gets$ (*$actual$).$definicion$ \Comment $O(1)$

\medskip
\State \underline{Complejidad:} $O(|P|)$
\State \underline{Justificaion:} Siendo $|P|$ el largo de la clave mas larga, sea cual sea $p$, $|p|$ $\leq$ $|P|$ entonces 	$O(|p|)$ $=$ $O(|P|)$

\end{algorithmic}
\end{algorithm}


%DEFINIR
\begin{algorithm}[H]
{\textbf{iDefinir}(\Inout{d}{diccString($\alpha$)}, \In{p}{string}, \In{a}{$\alpha$})}
\begin{algorithmic}[1]

\State Puntero(nodo) $actual$ $\gets$ $raiz$ \Comment $O(1)$

\State Nat $i$ $\gets$ $0$ \Comment $O(1)$

\State bool $esNueva$ $\gets$ $false$ \Comment $O(1)$

\While{$i$ $<$ Longitud($p$)}   \Comment Se repite $|p|$ $O(1)$


	\If{(*$actual$).$siguientes$[ord($p$[$i$])] $=$ $NULL$} \Comment $O(1)$

		\State arreglo(puntero(Nodo)) $sig$ $\gets$ CrearArreglo(256) \Comment $O(1)$

		\For{$i$ $=$ 0 to 256} \Comment se repite siempre 256 veces $O(1)$

			\State $sig$[i] $\gets$ $NULL$ \Comment $O(1)$

		\EndFor

		\State (*$actual$).$siguientes$[ord($p$[$i$])] $\gets$	 $\&$ $\langle$ $NULL$, $sig$, $NULL$ $\rangle$  \Comment $O(1)$

		\State $esNueva$ $\gets$ $true$ \Comment $O(1)$

	\EndIf

	\State $actual$ $\gets$ (*$actual$).$siguientes$[ord($p$[$i$])] \Comment $O(1)$

	\State $i$ $\gets$ $i + 1$
\EndWhile

\If{(*$actual$).definicion  $\neq$  $NULL$} \Comment $O(1)$
	\State (*$actual$).definicion $\gets$ $NULL$ \Comment se libera la memoria acupada por definicion $O(1)$
\EndIf
\State (*$actual$).$definicion$ $\gets$ $\&$ copiar($a$) \Comment $O(copiar(\alpha))$

\If{$esNueva$} \Comment $O(1)$
	% \State itConj($string$) $it$ $\gets$  $claves$.AgregarRapido($s$) \Comment $O(1)$
	% \State (*$actual$).$itClave$ $\gets$ $\&$ $it$ \Comment $O(1)$
	\State (*$actual$).$itClave$ $\gets$ $claves$.AgregarRapido($s$) \Comment $O(1)$
\EndIf

\medskip
\State \underline{Complejidad:} $O(|P| + copiar(\alpha))$
\State \underline{Justificaion:} Siendo $|P|$ el largo de la clave mas larga, sea cual sea $p$, $|p|$ $\leq$ $|P|$ entonces 	$O(|p|)$ $=$ $O(|P|)$ mas lo que cueste copiar a.


\end{algorithmic}
\end{algorithm}


%DEFINIDO
\begin{algorithm}[H]
{\textbf{iDef?}(\In{p}{string}) $\to$ $res$ : $bool$}
\begin{algorithmic}[1]

\State Nat $i$ $\gets$ $0$ \Comment $O(1)$

\State bool $pertenece$ $\gets$ $true$ \Comment $O(1)$

\State puntero(nodo) $actual$ $\gets$ $raiz$ \Comment $O(1)$

\While{$i$ $<$ Longitud($p$) $\land$  $pertenece$}   \Comment Se repite $|p|$ $O(1)$


	\If{(*$actual$).$siguientes$[ord($p$[$i$])] $=$ $NULL$} \Comment $O(1)$

		\State $pertenece$ $\gets$ $false$ \Comment $O(1)$

	\EndIf

	\State $actual$ $\gets$ (*$actual$).siguientes[$ord(p[i])$] \Comment $O(1)$

	\State $i$ $\gets$ $i$ $+$ 1 \Comment $O(1)$
\EndWhile

\If {(*$actual$).$significado$ $=$ $NULL$} \Comment $O(1)$
	\State $pertenece$ $\gets$ $false$ \Comment $O(1)$
\EndIf

\State $res$ $\gets$ $pertenece$ \Comment $O(1)$

\medskip
\State \underline{Complejidad:} $O(|P|)$
\State \underline{Justificaion:} Siendo $|P|$ el largo de la clave mas larga, sea cual sea $p$, $|p|$ $\leq$ $|P|$ entonces 	$O(|p|)$ $=$ $O(|P|)$

\end{algorithmic}
\end{algorithm}



%CLAVES
\begin{algorithm}[H]
{\textbf{iBorrar}(\In{p}{string}, \Inout{d}{diccString($\alpha$)}})
\begin{algorithmic}[1]

\State bool $borrarRaiz$ $\gets$ $d$.Claves() $=$ 1 \Comment $O(1)$

\State puntero(nodo) $reserva$ $\gets$ $raiz$ \Comment $O(1)$
\State nat $rindex$ $\gets$ 0 \Comment $O(1)$

\State puntero(nodo) $actual$ $\gets$ $raiz$ \Comment $O(1)$

\State nat $i$ $\gets$ 0 \Comment $O(1)$
\While{$i$ $<$ Longitud($p$) }   \Comment Se repite $|p|$ $O(1)$



	\State $actual$ $\gets$ (*$actual$).siguientes[$ord(p[i])$] \Comment $O(1)$

	\State bool $definido$ $\gets$ $i$ $\neq$ $|p|$ $-$ 1 $\land$ (*$actual$).definicion $\neq$ $NULL$ \Comment $O(1)$

	\If{CuentaHijos($actual$) $>$ 1 $\lor$ $definido$} \Comment $O(1)$

	\State $reserva$ $\gets$ $actual$ \Comment $O(1)$
	\State $rindex$ $\gets$ $i$ $+$ 1 \Comment $O(1)$

	\EndIf

	\State $i$ $\gets$ $i$ $+$ 1 \Comment $O(1)$
\EndWhile


\State EliminarSiguiente((*$actual$).(*$itClave$)) \Comment $O(1)$

\State (*$actual$).$itClave$ $\gets$ $NULL$ \Comment Se libera la memoria ocupada por (*$actual$).$itClave$ $O(1)$

\If{CuentaHijos($actual$) $>$ 1} \Comment $O(1)$

	\State (*$actual$).definicion $\gets$ $NULL$ \Comment se libera la memoria ocupada por la definicion $O(1)$

\EndIf

\If{CuentaHijos($actual$) $=$ 0} \Comment $O(1)$
	\State BorrarDesde($reserva$, $rindex$) \Comment $O(|P|)$
\EndIf


\If{borrarRaiz} \Comment $O(1)$
	\State $d$.raiz $\gets$ $NULL$ \Comment se libera la memoria ocupada por raiz $O(1)$
\EndIf


\medskip
\State \underline{Complejidad:} $O(|P|)$
\State \underline{Justificaion:} Siendo $|P|$ el largo de la clave mas larga, sea cual sea $p$, $|p|$ $\leq$ $|P|$ entonces $O(2*|p|)$ $=$	$O(|p|)$ $=$ $O(|P|)$

\end{algorithmic}
\end{algorithm}

\pagebreak

\textbf{Pre} $\equiv$ \{$desde$ y $desde$.siguientes[$a$] no nulos\}\\%
\textbf{Post} $\equiv$ \{Borra la rama a partir de $desde$.siguientes[$a$]\}%


\begin{algorithm}[H]
{\textbf{iBorrarDesde}(\Inout{desde}{puntero(nodo)}}, \In{a}{nat})
\begin{algorithmic}[1]

\State puntero(nodo) $temp$ $\gets$ raiz \Comment $O(1)$

\State $desde$ $\gets$ (*$desde$).$siguientes$[a] \Comment $O(1)$
\State (*$desde$).$siguientes$[a] $\gets$ $NULL$ \Comment Se libera la memoria ocupada por (*desde).siguientes[a] $O(1)$

\While{$desde$ $\neq$ $NULL$} \Comment se repite a los sumo $|P|$ veces, siendo p la clave mas larga $O(1)$

	\State puntero(nodo) $temp$ $\gets$ $desde$ \Comment $O(1)$

	\State bool $sigue$ $\gets$ false \Comment $O(1)$

	\State nat $i$ $\gets$ 0 \State $O(1)$

	\While{$i$ $<$ 256} \Comment se repite siempre 256 veces $O(1)$

		\If{(*$desde$).siguientes[$i$] $\neq$ $NULL$} \Comment $O(1)$
			\State $desde$ $\gets$ (*$desde$).siguientes[$i$] \Comment $O(1)$

			\State $sigue$ $\gets$ true \Comment $O(1)$

		\EndIf
		\State $i$ $\gets$ $i$ $+$ 1 \Comment $O(1)$

	\EndWhile

\EndWhile

\If{$\lnot$ $sigue$} \Comment $O(1)$
	\State $desde$ $\gets$ $NULL$ \Comment se libera la memoria ocupada por $desde$ $O(1)$

\EndIf

\State $temp$ $\gets$ $NULL$ \Comment se libera la memoria ocupada por $temp$ $O(1)$


\medskip
\State \underline{Complejidad:} $O(|P|)$
\State \underline{Justificaion:} Siendo $|P|$ el largo de la clave mas larga, sea cual sea la rama que estamos borrando, es mas corta que la rama representada por la clave mas larga. Llamo $p$ a la clave de la rama que estamos borrando y como $|p|$ $\leq$ $|P|$ entonces $O(|p|)$ = $O(|P|)$

\end{algorithmic}
\end{algorithm}



\textbf{Pre} $\equiv$ \{$desde$ no nulo\}\\%
\textbf{Post} $\equiv$ \{deveulve la cantidad de punteros no nulos en $desde$.siguientes\}%

\begin{algorithm}[H]
{\textbf{iCuentaHijos}(\In{desde}{puntero(nodo)}} $\to$ $res$ : $nat$)
\begin{algorithmic}[1]

\State nat $i$ $\gets$ 0 \Comment $O(1)$
\State nat $hijos$ $\gets$ 0 \Comment $O(1)$

\While{$i$ $<$ 256} \Comment se repite siempre 256 veces $O(1)$
	\If{(*$actutal$).siguiente[$i$] $\neq$ $NULL$} \State $O(1)$

	\State $suma$ $\gets$ $suma$ $+$ 1 \Comment $O(1)$
	\State $i$ $\gets$ $i$ $+$ 1 \Comment $O(1)$
	\EndIf
\EndWhile
\State $res$ $\gets$ $hijos$ \Comment $O(1)$


\medskip
\State \underline{Complejidad:} $O(1)$
\State \underline{Justificaion:}  $O(1)$ $+$ $O(1)$ $+$ $O(256)$ $+$ $O(1)$ $+$ $O(1)$ $+$ $O(1)$ $=$ $O(261)$ = $O(1)$

\end{algorithmic}
\end{algorithm}


\end{Algoritmos}


\subsection{Servicios usados}

De Conjunto Lineal \\
 - Vacio() debe ser O(1) \\
 - AgregarRapido( conj($\alpha$), $\alpha$ $\alpha$) debe ser O(copy(a)) \\
 - EliminarSiguiente(itConj($\alpha$)) debe ser O(1) \\

De String \\
 - Longitud(string) debe ser O(1) \\

De Char \\
 - ord(char) debe ser O(1) \\

