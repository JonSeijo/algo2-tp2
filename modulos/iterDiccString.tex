\begin{Interfaz}
\subsection{Interfaz}

  \textbf{par\'ametros formales}\hangindent=2\parindent\\
  \parbox{1.7cm}{\textbf{g\'eneros}} $\alpha$
 
 
  \textbf{se explica con}: \tadNombre{IteradorUnidireccional($\alpha$)}, \tadNombre{DiccString($\alpha$)}.

  \textbf{g\'eneros}: \TipoVariable{iterDiccString($\alpha$)}.


\InterfazFuncion{CrearIt}{\In{d}{diccString($\alpha$)}}{iterDiccString($\alpha$)}%
[true]
{alias(esPermutaci\'on(SecuSuby($res$), claves($d$)))}%
[$O(1)$] % Por enunciado
[Devuelve un iterador al diccionario.]
[El iterador se invalida si se modifican claves del diccionario.]

\InterfazFuncion{HayMas?}{\In{it}{iterDiccString($\alpha$)}}{bool}%
[true]
{$res$ = hayMas?($it$)}%
[$O(1)$]
[Devuelve true si hay mas claves por recorrer.]

% AVISAR EN INFORME SOBRE EL CAMBIO DE ARIDAD
\InterfazFuncion{Actual}{\In{it}{iterDiccString($\alpha$)}}{tupla<string, $\alpha$>}%
[hayMas?($it$)]
{$res$ = actual($it$)}%
[$O(|P|)$, donde |P| es la longitud de la clave mas larga.]
[Devuelve una tupla con el elemento actual y su significado.]

\InterfazFuncion{Avanzar}{\Inout{it}{iterDiccString($\alpha$)}}{}%
[$it$ = $it_0$ $\land$ hayMas?($it$)]
{$it$ = avanzar($it_0$)}%
[$O(1)$]
[Avanza a la posicion siguiente del iterador.]



\end{Interfaz}

%\pagebreak


\begin{Representacion}

Este es el iterador que se usa para recorrer el DiccString. Para recorrerlo, aprovechando que tenemos un conjunto de claves en nuestro diccionario, vamos a recorrer el conjunto de claves usando el iterador de conjunto que ya existe. Para obtener el elemento, buscamos la clave ``actual'' (seg\'un iterador de claves) en el diccionario. Esto implica que obtener el elemento sea un poco costoso, porque hay que buscarlo en el diccionario, pero decidimos hacerlo as\'i ya que nos pareci\'o la forma mas simple que cumpl\'ia los requisitos de complejidad.

\subsection{Representacion del iterDiccString}
    
  \begin{Estructura}{iterDiccString($\alpha$)}[estr]
    \begin{Tupla}[estr]
      \tupItem{itClave}{itConj(string)}%
      \tupItem{dicc}{diccString($\alpha$)}%
    \end{Tupla}
  \end{Estructura}


\Title{Invariante de representacion}


\tadOperacion{Rep}{estr$/e$}{bool}{}%
\tadAxioma{Rep($e$)}
{Rep($e$.itClave) $\yluego$ ((siguiente($e$.itClave) $=$ NULL $\oluego$ siguiente($e$.itClave) $\in$ claves($e$.dicc))}

$ $\newline
$ $\newline


\Title{Funcion de abstraccion}

\tadOperacion{Abs}{estr$/e$}{itUni($\alpha$)}{Rep($d$)}%
\tadAxioma{Abs($e$)}{
    $uni$ : itUni($\alpha$) / \\
    siguientes($uni$) $\igobs$ siguientes($e$.itClave)
}%

\end{Representacion}

\begin{Algoritmos}
\subsection{Algoritmos}
% crearIt 


\begin{algorithm}[H]
{\textbf{iCrearIt}(\In{d}{diccString($\alpha$)}) $\to$ $res$ : iterDiccString($\alpha$)}
\begin{algorithmic}[1]

\State $res \gets$ $\langle$ $d$ , CrearIt(Claves($d$)) $\rangle$ \Comment Por referencia $O(1)$
 
\medskip
\Statex \underline{Complejidad:} $O(1)$
\Statex \underline{Justificacion:} Crear el iterador de conjunto es O(1), obtener las claves del diccionario es O(1), crear la tupla con los dos elementos es O(1). La complejidad total es O(1)
\end{algorithmic}
\end{algorithm}


% hayMas? 

\begin{algorithm}[H]
{\textbf{iHayMas?}(\In{iter}{iterDiccString($\alpha$)}) $\to$ $res$ : bool}
\begin{algorithmic}[1]

\State $res \gets$ HaySiguiente($iter$.itClave) \Comment $O(1)$

\medskip
\Statex \underline{Complejidad:} $O(1)$
\end{algorithmic}
\end{algorithm}


% actual

\begin{algorithm}[H]
{\textbf{iActual}(\In{iter}{iterDiccString($\alpha$)}) $\to$ $res$ : tupla$<$string, $\alpha$ $>$}
\begin{algorithmic}[1]

\State $res \gets$ $\langle$ Siguiente($iter$.iClave), Obtener(Siguiente($iter$.iClave), $iter$.dicc) $\rangle$ \Comment $O(|P|)$

\medskip
\Statex \underline{Complejidad:} $O(|P|)$
\Statex \underline{Justificaci\'on:} Para crear la tupla, necesito acceder al significado de la clave. Por las complejidades del DiccString, el peor caso se corresponde con la longitud de la clave mas larga: O(|P|), donde P es la clave mas large. Acceder al Siguiente del conjunto es O(1). En total, el algoritmo cuesta O(|P|)

\end{algorithmic}
\end{algorithm}


% avanzar

\begin{algorithm}[H]
{\textbf{iAvanzar}(\Inout{iter}{iterDiccString($\alpha$)})}
\begin{algorithmic}[1]

\State Avanzar($iter$.itClave) \Comment $O(1)$

\medskip
\Statex \underline{Complejidad:} $O(1)$

\end{algorithmic}
\end{algorithm}

\end{Algoritmos}  
  
\subsection{Servicios usados}

De conj($\alpha$) \\
- CrearIt(conj($\alpha$)) debe ser O(1) \\

De itConj($\alpha$) \\
- HaySiguiente(itConj($\alpha$)) debe ser O(1) \\
- Siguiente(itConj($\alpha$)) debe ser O(1) \\
- Avanzar(itConj($\alpha$)) debe ser O(1) \\

De diccString($\alpha$) \\
- Claves(diccString($\alpha$)) debe ser O(1) \\
- Obtener(string, diccString($\alpha$)) debe ser O(|P|) \\

