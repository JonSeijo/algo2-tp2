\begin{Interfaz}


\textbf{se explica con}: \tadNombre{Mapa}.

\textbf{g\'eneros}: \TipoVariable{map}.

\Title{Operaciones b\'asicas de Mapa}


\InterfazFuncion{CrearMapa}{}{map}%
{$res \igobs crearMapa$}%
[$O(1)$]
[Genera una mapa vac\'io.]


\InterfazFuncion{AgregarCoor}{ \In{c}{coord}, \Inout{m}{map}}{}
[$m_0 \igobs m \land \lnot posExistente(c, m_0) $]  
{$m \igobs agregarCoor(c, m_0)$}
[$O(  max(latitud(c), longitud(c), tam(m))^4  )$]
[Agrega la coordenada \texttt{c} al mapa \texttt{m}]


\InterfazFuncion{Coordenadas}{\In{m}{map}}{$conj(coord)$}%
{$res \igobs coordenadas(m)$}%
[$O(tam(m))$]  % Algoritmo: recorro la diagonal principal
[Devuelve el conjunto de todas las coordenadas del mapa m.]


\InterfazFuncion{PosExistente}{ \In{c}{coord}, \In{m}{map}}{bool}
{$res \igobs posExistente(c, m)$}
[$O(1)$]
[Verifica si la coordenada \texttt{c} existe en el mapa \texttt{m}]


\InterfazFuncion{HayCamino}{ \In{c_1}{coord}, \In{c_2}{coord}, \In{m}{map}}{bool}
[$posExistente(c_1, m) \land posExistente(c_2, m) $]
{$res \igobs hayCamino(c_1, c_2, m)$}
[$O(1)$]
[verifica si existe una forma de llegar desde \texttt{$c_1$} a \texttt{$c_2$}]


\Title{Especificacion de auxiliares usadas en la interfaz}
  
\begin{tad}{\tadNombre{Mapa Extension}}
\tadExtiende{\tadNombre{Mapa}}

\tadTitulo{Otras operaciones}{}

\tadOperacion{tam}{mapa/$m$}{nat}{}
\tadOperacion{max}{nat/$x$, nat/$y$}{nat}{}
\tadOperacion{maxLatitud}{conj(coor)/$cs$}{nat}{$\neg \emptyset ?(cs)$}
\tadOperacion{maxLongitud}{conj(coor)/$cs$}{nat}{$\neg \emptyset ?(cs)$}

\tadAxioma{tam(mapa)}
{
\IF
$\#$(coordenadas(mapa)) = 0
THEN
0
ELSE
max(maxLatitud(coordenadas(mapa)), maxLongitud(coordenadas(mapa)))
FI
}

\tadAxioma{max(x ,y)}{\IF x $\geq$ y THEN x ELSE y FI}

\tadAxioma{maxLatitud(cs)}
{
\IF
$\#cs = 1$
THEN
latitud(dameUno(cs))
ELSE
{
\IF latitud(dameUno(cs)) $\geq$ maxLatitud(sinUno(cs))
THEN
damUno(cs)
ELSE
maxLatitud(sinUno(cs))
FI
}
FI}


\tadAxioma{maxLongitud(cs)}
{
\IF
$\#cs = 1$
THEN
longitud(dameUno(cs))
ELSE
{
\IF longitud(dameUno(cs)) $\geq$ maxLongitud(sinUno(cs))
THEN
damUno(cs)
ELSE
maxLongitud(sinUno(cs))
FI
}
FI}


\end{tad}



\end{Interfaz}


\begin{Representacion}

\begin{Estructura}{mapa}[map]
	\begin{Tupla}[map]  
		\tupItem{tam}{nat}
		\tupItem{grilla}{vector(vector(vector(vector(bool)))}  
  	\end{Tupla}
  	
\end{Estructura}

\Title{Invariante de representacion en castellano}

$ $\newline
[Comentarios para la parte de explicacion de estructura] $ $\newline
(decimos que C=(x,y) $"$esta en la grilla$"$ cuando grilla[x][y][x][y] == true) $ $\newline
(decimos que hay camino entre C=(x,y) y C'=(z,w) cuando grilla[x][y][z][w] == true) $ $\newline

(1) Todos los vectores que forman la grilla son de la misma longitud$ $\newline
(2) tam es consistente con la longitud de la grilla $ $\newline
(3) Si vale grilla[x][y][z][w] vale tambien grilla[z][w][x][y] $ $\newline
(4) Si C no esta en la grilla, entonces no tiene camino con ninguna $ $\newline
(5) Si C y C' son contiguas y ambas estan en la grilla, entonces hay camino entre ellas $ $\newline
(6) Si hay camino entre C y C', y hay camino entre C' y C'' entonces hay camino entre C y C'' $ $\newline
Y Si hay camino entre C y C', y NO hay camino entre C' y C'' entonces NO hay camino entre C y C'' $ $\newline

\Title{Invariante de representacion en logica}

$ $\newline
$ $\newline

(1)
(($\forall i$: nat)(i $<$ long(m.grilla)) $\impluego$ long(m.grilla[i]) = long(m.grilla)) $\yluego$ $ $\newline
(($\forall i, j$: nat)(i $<$ long(m.grilla) $\land$ j $<$ long(m.grilla)) $\impluego$ long(m.grilla[i][j]) = long(m.grilla)) $\yluego$ $ $\newline
(($\forall i, j, k$: nat)(i $<$ long(m.grilla) $\land$ j $<$ long(m.grilla) $\land$ k $<$ long(m.grilla)) $\impluego$ $ $\newline long(m.grilla[i][j][k]) = long(m.grilla))

$ $\newline
(2)
m.tam = long(e.grilla)

$ $\newline
(3) 
(($\forall x, y, z, w$: nat)(enRango(m, x, y) $\land$ enRango(m, z, w)) $\impluego$ (grilla[x][y][z][w] = grilla[z][w][x][y])

$ $\newline
(4)
(($\forall x, y$: nat)(enRango(m, x, y)) $\impluego$ $ $\newline 
($\neg$ grilla[x][y][x][y] $\implies$
($\forall z, w$: nat)(enRango(m, z, w)) $\impluego$
$\neg$ grilla[x][y][z][w] ))) 

$ $\newline
(5)
(($\forall x, y, z, w$: nat)$ $\newline
((enRango(m, x, y) $\yluego$ m.grilla[x][y][x][y]) $\land$ $ $\newline
(enRango(m, z, w) $\yluego$ m.grilla[z][w][z][w]) $\yluego$ $ $\newline
esContigua(x,y,z,w) $\impluego$ m.grilla[x][y][z][w]))


$ $\newline
(6)
(($\forall a, b, c, d, e, f$: nat)$ $\newline
(enRango(m, a, b) $\yluego$ m.grilla[a][b][a][b]) $\land$ $ $\newline
(enRango(m, c, d) $\yluego$ m.grilla[c][d][c][d]) $\land$ $ $\newline
(enRango(m, e, f) $\yluego$ m.grilla[e][f][e][f]) $\land$ $ $\newline
(a $\neq$ c $\lor$ b $\neq$ d) $\land$ (c $\neq$ e $\lor$ d $\neq$ f)) $ $\newline
$\impluego$ ( $ $\newline
((m.grilla[a][b][c][d] $\land$ m.grilla[c][d][e][f]) $\implies$ m.grilla[a][b][e][f]) $\land $\newline
((m.grilla[a][b][c][d] $\land$ $\neg$m.grilla[c][d][e][f]) $\implies$ $\neg$ m.grilla[a][b][e][f])$ $\newline
)

% Auxiliar EN RANGO
$ $\newline
\tadOperacion{enRango}{estr$/m$, nat$/x$, nat$/y$}{bool}{}%
\tadAxioma{enRango($m, x, y$)}{(x $<$ m.tam $\land$ y $<$ m.tam)} 

% Auxiliar ES CONTIGUA
$ $\newline
\tadOperacion{esContigua}{nat$/x$, nat$/y$, nat$/z$, nat$/w$}{bool}{}%
\tadAxioma{esContigua($x, y, z, w$)}
{
(x = z $\land$ y = w+1) $\lor$ \\
(x = z+1 $\land$ y = w) $\lor$ \\
(\IF z > 0 THEN (x = z-1 $\land$ y = w) ELSE false FI) $\lor$ \\
(\IF w > 0 THEN (x = z $\land$ y = w-1) ELSE false FI)
} 


$ $\newline

% REP  
% REVISAR LOS YLUEGOS
\tadOperacion{Rep}{estr$/m$}{bool}{}%
\tadAxioma{Rep($m$)}
{{(1) $\yluego$ (2) $\yluego$ (3) $\land$ (4) $\land$ (5) $\land$ (6)}} 

$ $\newline
$ $\newline


% ABS

\tadOperacion{Abs}{estr$/m$}{Mapa}{Rep($m$)}%
\tadAxioma{Abs($m$)}
{map : Mapa / ($\forall c$: coord)(c $\in$ coordenadas(map)) $\impluego$ \\
(m.grilla[latitud(c)][longitud(c)][latitud(c)][longitud(c)] = true )}%

\end{Representacion}

$ $\newline
$ $\newline

\begin{Algoritmos}


% crearMapa
\begin{algorithm}[H]
{\textbf{iCrearMapa}()$\to$ $res$ : map}
\begin{algorithmic}[1]
   
\State vector(vector(vector(vector(bool)) $mapa$ $\gets$ Vacio()    \Comment Crear vector vacio es O(1) // $O(1)$
\State $res$ $\gets$  $\langle$ 0 , $mapa$ $\rangle$   \Comment $O(1)$

\medskip
\Statex \underline{Complejidad:} $O(1)$
\State \underline{Justificaion:} $O(1)$ $+$ $O(1)$

\end{algorithmic}
\end{algorithm}

% coordenadas
\begin{algorithm}[H]
{\textbf{iCoordenadas}(\In{m}{map})$\to$ $res$ : conj(coor)}
\begin{algorithmic}[1]
   

\State conj(coor) $coors$ $\gets$ Vacio()      \Comment Crear conjunto vacio es O(1) // $O(1)$

\For{$i=0$ to m.tam}
	
	\For{$j=0$ to m.tam}
		
		\If{iPosExistente(i, j)}  \Comment $O(1)$
			
			\State $c$ $\gets$ NuevaCoor($i$, $j$)   \Comment $O(1)$			
			\State AgregarRapido($coors$, $c$)   \Comment $O(1)$			
			
		\EndIf
		
	\EndFor

\EndFor

\State $res$ $\gets$ $coors$


\medskip
\Statex \underline{Complejidad:} $O((tam(m))^2)$

\Statex \underline{Justificacion:} Las operaciones interiores son O(1). Hay dos for anidados, donde cada uno se ejecuta $tam(m)$ veces. En total son $(tam(m))^2$ iteraciones. 

\end{algorithmic}
\end{algorithm}

% posExistente

% EXPLICACION:
% CUANDO AGREGO UNA COORDENADA, LA CONECTO EN LA GRILLA CONSIGO MISMA, SIGNIFICANDO QUE EXISTE 
\begin{algorithm}[H]
{\textbf{iPosExistente}(\In{m}{map}, \In{c}{coor})$\to$ $res$ : bool}
\begin{algorithmic}[1]
   
\State bool $existe$

\If{latitud($c$) $\geq$ $m$.tam $\lor$ longitud($c$) $\geq$ $m$.tam} \Comment $O(1)$
	\State $existe \gets$ false \Comment $O(1)$
\Else
	\State nat $x \gets$ Latitud($c$) \Comment $O(1)$
	\State nat $y \gets$ Longitud($c$) \Comment $O(1)$
	\State $existe \gets$ $m$.grilla[$x$][$y$][$x$][$y$] \Comment $O(1)$
\EndIf

\State $res$ $\gets$ $existe$ \Comment $O(1)$

\medskip
\Statex \underline{Complejidad:} $O(1)$
\Statex \underline{Justificacion:} Evaluar la guarda del if es O(1) por que son comparaciones de nats. Si es verdadera se produce una asignacion O(1). Si es falsa se ejecutan asignaciones O(1) y acceso directo a vector O(1). O(1) + O(1) + O(1) + O(1) = O(1). Es constante en cualquier caso.

\end{algorithmic}
\end{algorithm}

% hayCamino

\begin{algorithm}[H]
{\textbf{iHayCamino}(\In{m}{map}, \In {c_1}{coor}, \In{c_2}{coor})$\to$ $res$ : bool}
\begin{algorithmic}[1]

\State $res$ $\gets$ $m$.grilla[Latitud($c_1$)][Longitud($c_1$)][Latitud($c_2$)][Longitud($c_2$)] \Comment $O(1)$

\medskip
\Statex \underline{Complejidad:} $O(1)$
\Statex \underline{Justificacion:} Latitud() y Longitud() son O(1). Accesos directos a vectores es O(1)

\end{algorithmic}
\end{algorithm}


% agregaCoor
%\begin{algorithm}[H]
{\textbf{iAgregaCoor}(\Inout{m}{map}, \In {c}{coor})}
\begin{algorithmic}[1]

\State nat $maximo$ $\gets$ max(Latitud($c$), Longitud($c$))  \Comment $O(1)$

% Si hay una coordenada mayor al maximo
\State \If{$maximo$ $>$ $m$.tam} \Comment $O(1)$

\State vector(vector(vector(vector(bool))))) $nGrilla$

\State $nGrilla \gets iCrearGrilla(maximo)$ \Comment O(max(Latitud($c$), Longitud($c$))$^4$)
\State iCopiarCoordenadas($nGrilla$, $m$.grilla) \Comment O(max(Latitud($c$), Longitud($c$))$^4$)
\State $m$.grilla $\gets nGrilla$ \Comment O(max(Latitud($c$), Longitud($c$))$^4$)
\State $m$.tam $\gets maximo$ \Comment $O(1)$

\EndIf

% La marco consigo misma para decir que la cooordenada existe
\State $m$.grilla[Latitud($c$)][Longitud($c$)][Latitud($c$)][Longitud($c$)] $\gets$ true \Comment $O(1)$
\State $ $

% Creo una grilla de visitadas, todas false
%Vector vacio ver como escribirlo bien

\State vector(vector(coor))$visitados$ $\gets$ Vacio() \Comment $O(1)$

\For{$i = 0$ to m.tam $-$ 1}	    \Comment $m.tam^2$ iteraciones, pero m.tam pudo haber cambiado // O(max(Latitud($c$), Longitud($c$), $m$.tam)$^2$)

	\State vector(coor) $visitadosAux \gets$ Vacio()  \Comment $O(1)$	
	\For{$j = 0$ to m.tam $-$ 1}      \Comment Realiza m.tam iteraciones $O(m.tam)$
		\State $visitadosAux$.AgregarAtras(false) \Comment $O(1)$
	\EndFor
	\State $visitados$.AgregarAtras($visitadosAux$) \Comment $O(1)$
\EndFor


% aRecorrer es una cola..
\State cola(coor) $aRecorrer$ $\gets$ Vacio() \Comment $O(1)$
\State $aRecorrer$.Encolar($c$) \Comment $O(1)$

\State $ $

\While{$\neg$ EsVacia($aRecorrer$)} \Comment Como maximo se recorren todas las coordenadas del mapa // O(max(Latitud($c$), Longitud($c$), m.tam))$^2$)

\State coor $act$ $\gets$ Proximo($aRecorrer$) \Comment  $O(1)$
\State Desencolar($aRecorrer$) \Comment $O(1)$

\State $ $


% Reviso la coordenada de la izquierda


% Reviso la coordenada de la izquierda
\If{Latitud($act$) $>$ 0}      \Comment $O(1)$

\State nat $x$ $\gets$ Latitud(CoordenadaALaIzquierda($act$))   \Comment $O(1)$
\State nat $y$ $\gets$ Longitud(CoordenadaALaIzquierda($act$))  \Comment $O(1)$
\If{$\neg$ visitados[$x$][$y$]}    \Comment $O(1)$
\State visitados[$x$][$y$]$ \gets$ true    \Comment $O(1)$
	\If{Existe(coordenadaALaIzquierda($act$))} 	\Comment $O(1)$
		\State $m$.Grilla[Latitud($c$)][Longitud($c$)][$x$][$y$] $ \gets $ true \Comment $O(1)$
		\State $m$.Grilla[$x$][$y$][Latitud($c$)][Longitud($c$)] $ \gets $ true \Comment $O(1)$
		\State Encolar(coordenadaALaIzquierda($act$), aRecorrer) \Comment $O(1)$	
	\EndIf
\EndIf
\EndIf

\State $ $

\If{Longitud($act$) $>$ 0}      \Comment $O(1)$

\State nat $x$ $\gets$ Latitud(CoordenadaAbajo($act$))   \Comment $O(1)$
\State nat $y$ $\gets$ Longitud(CoordenadaAbajo($act$))  \Comment $O(1)$
\If{$\neg$ visitados[$x$][$y$]}    \Comment $O(1)$
\State visitados[$x$][$y$]$ \gets$ true    \Comment $O(1)$
	\If{Existe(CoordenadaAbajo($act$))} 	\Comment $O(1)$
		\State $m$.Grilla[Latitud($c$)][Longitud($c$)][$x$][$y$] $ \gets $ true \Comment $O(1)$
		\State $m$.Grilla[$x$][$y$][Latitud($c$)][Longitud($c$)] $ \gets $ true \Comment $O(1)$
		\State Encolar(CoordenadaAbajo($act$), aRecorrer) \Comment $O(1)$	
	\EndIf
\EndIf
\EndIf

\State $ $


\If{Latitud($act$) $<$ $m$.$Tam$ $-$ 1}      \Comment $O(1)$

\State nat $x$ $\gets$ Latitud(CoordenadaALaDerecha($act$))   \Comment $O(1)$
\State nat $y$ $\gets$ Longitud(CoordenadaALaDerecha($act$))  \Comment $O(1)$
\If{$\neg$ visitados[$x$][$y$]}    \Comment $O(1)$
\State visitados[$x$][$y$]$ \gets$ true    \Comment $O(1)$
	\If{Existe(CoordenadaALaDerecha($act$))} 	\Comment $O(1)$
		\State $m$.Grilla[Latitud($c$)][Longitud($c$)][$x$][$y$] $ \gets $ true \Comment $O(1)$
		\State $m$.Grilla[$x$][$y$][Latitud($c$)][Longitud($c$)] $ \gets $ true \Comment $O(1)$
		\State Encolar(CoordenadaALaDerecha($act$), aRecorrer) \Comment $O(1)$	
	\EndIf
\EndIf
\EndIf
\State $ $

\If{Longitud($act$) $<$ m.Tam$-$1}      \Comment $O(1)$

\State nat $x \gets$ Latitud(CoordenadaArriba($act$))   \Comment $O(1)$
\State nat $y \gets$ Longitud(CoordenadaArriba($act$))  \Comment $O(1)$
\If{$\neg$ visitados[$x$][$y$]}    \Comment $O(1)$
\State visitados[$x$][$y$]$ \gets$ true    \Comment $O(1)$
	\If{Existe(CoordenadaArriba($act$))} 	\Comment $O(1)$
		\State $m$.Grilla[Latitud($c$)][Longitud($c$)][$x$][$y$] $ \gets $ true \Comment $O(1)$
		\State $m$.Grilla[$x$][$y$][Latitud($c$)][Longitud($c$)] $ \gets $ true \Comment $O(1)$
		\State Encolar(CoordenadaArriba($act$), aRecorrer) \Comment $O(1)$	
	\EndIf
\EndIf
\EndIf

\EndWhile

\medskip
\Statex \underline{Complejidad:} O(max(Latitud($c$), Longitud($c$), $m$.tam))$^4$)
\Statex \underline{Justificacion:} Trabajo con matrices de 4 dimensiones, en peor caso hay que redimensionar, en donde se crea una nueva grilla con el nuevo tamanio. CrearGrilla tiene complejidad O($n^4$), por lo que en este caso es O(max(Latitud($c$), Longitud($c$))$^4$). Si se redimensiona, es porque:  max(Latitud($c$), Longitud($c$)) $> $ m.tam. Entonces en ese caso,  max(Latitud($c$), Longitud($c$)) $=$ O(max(Latitud($c$), Longitud($c$), $m$.tam))$^4$). 
\Statex En peor caso se recorren luego todas las coordenadas de la grilla "principal" (las primeras 2 dimensiones) La grilla pudo haber sido redimensionada, por lo que la complejidad del while es O(max(Latitud($c$), Longitud($c$), $m$.tam))$^2$), donde m.tam es el tamanio original de la grilla. Entonces la complejidad en peor caso es la de mayor exponente por algebra de ordenes: O(max(Latitud($c$), Longitud($c$), $m$.tam))$^4$)


\end{algorithmic}
%\end{algorithm}


% tam

\begin{algorithm}[H]
{\textbf{iTam}(\In{m}{map})$\to$ $res$ : nat}
\begin{algorithmic}[1]

\State $res$ $\gets$ $m$.$tam$ \Comment $O(1)$

\medskip
\Statex \underline{Complejidad:} $O(1)$

\end{algorithmic}
\end{algorithm}



% crearGrilla



\begin{algorithm}[H]
{\textbf{iCrearGrilla}(\In{n}{nat})$\to$ $res$ : vector(vector(vector(vector(bool))))}
\begin{algorithmic}[1]

\Statex Funcion privada 
\Statex Pre: n $>$ 0 
\Statex Post: res es una grilla de tam n

\State vector(vector(vector(vector(bool)))) $nGrilla \gets$ Vacio() \Comment $O(1)$

\For{nat $i \gets$ 0 to $n - 1$}   \Comment O($n^4$)

	\State vector(vector(vector(bool))) $nGrilla2 \gets$ Vacio()  \Comment $O(1)$
	
	\For{nat $j \gets$ 0 to $n - 1$}  \Comment O($n^3$)
	
		\State vector(vector(bool)) $nGrilla3 \gets$ Vacio()   \Comment $O(1)$
		
		\For{nat $k \gets$ 0 to $n - 1$}  \Comment O($n^2$)

			\State vector(bool) $nGrilla4 \gets$ Vacio() \Comment $O(1)$
			
			 \For{nat $l \gets$ 0 to $n - 1$}  \Comment O($n$)
			 	\State AgregarAtras($nGrilla4$, false) \Comment $O(1)$			 	
			 \EndFor		
			 
			 \State AgregarAtras($nGrilla3$, $nGrilla4$)
		
		\EndFor
		
		\State AgregarAtras($nGrilla2$, $nGrilla3$)
	
	\EndFor
	
	\State AgregarAtras($nGrilla$, $nGrilla2$)


\EndFor

\medskip
\Statex \underline{Complejidad:} $O(n^4)$
\Statex \underline{Justificacion:} Son 4 fors anidados que se ejecutan $n$ veces cada uno. O($n$) $*$ O($n$) $*$ O($n$) $*$ O($n$) $=$  O($n^4$)
 
\end{algorithmic}
\end{algorithm}



  
\end{Algoritmos}