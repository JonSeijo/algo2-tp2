\begin{Interfaz}


\textbf{se explica con}: \tadNombre{Juego}.

\textbf{g\'eneros}: \TipoVariable{juego}.

\Title{Operaciones b\'asicas de Juego}



% GENERADORES %


\InterfazFuncion{crearJuego}{\In{m}{map}}{juego}%
{$res \igobs crearJuego(m)$}%
[$O((Tam(m))^2)$]
[Genera una juego con el mapa \texttt{m} y sin jugadores.]


\InterfazFuncion{agregarPok\'emon}{\In{p}{pokemon}, \In{c}{coord}, \Inout{j}{juego}}{}
[$j_0 \igobs j \land posExistente(c, mapa(j)) \land p \not \in pokemones(j) Y PUEDOAGREGARPOKEMON() QUE CHECKEE LAS DISTANCIAS $]  
{$j \igobs agregarPokemon(p, c, j_0)  \land p \in pokemones(j)$}
[$\Theta(¡¡completar!!)$]
[agrega pok\'emon \texttt{p} al juego \texttt{j} en la coordenada \texttt{c}.]


\InterfazFuncion{agregarJugador}{\In{j}{juego}}{nat}% REVISAR SI HAY QUE DEVOLVER NAT
[$ j_0 \igobs j $]
{$res \igobs ProxId(j_0) \land j \igobs agregarJugador(j_0)$}%
[$O(1)$]
[agrega un jugador al juego \texttt{j} con id igual a ProxId(j).]


\InterfazFuncion{conectarse}{\Inout{j}{juego}, \In{c}{coor}, \In{e}{jugador}}{}%
[$j \igobs j_0 \land e \in jugadores(j) \yluego \neg estaConectado(e, j) \land posExistente(c, mapa(j)) $]
{$estaConectado(e, j) Y cambian capturables y contadores y sarasa$}%
[$O(log(EC))$] % Es asi por enunciado
[Conecta al jugador e en la posicion c]


\InterfazFuncion{desconectarse}{\Inout{j}{juego}, \In{e}{jugador}}{}%
[$j \igobs j_0 \land e \in jugadores(j) \yluego estaConectado(e, j)$]
{$\neg estaConectado(e, j) Y cambian capturables y contadores y sarasa$}%
[$O(log(EC))$] % Es asi por enunciado
[Desconecta al jugador e del juego]


\InterfazFuncion{moverse}{\Inout{j}{juego}, \In{c}{coor}, \In{e}{jugador}}{}%
[$j \igobs j_0 \land e \in jugadores(j) \yluego estaConectado(e, j) \land posExistente(c, mapa(j)) $]
{$cambia posicion y cambian capturables y contadores y sarasa$}%
[$O((PC + PS) * |P| + EC * log(EC))$] % Es asi por enunciado
[Mueve al jugador e en la posicion c si es valido, sino sanciona o captura o sarasa]

% OBSERVADORES

%mapa : juego −→ map
\InterfazFuncion{Mapa}{\In{j}{juego}}{map}%
[true]
{$res = mapa(j)$}%
[$O(1)$] % HAGAMOS QUE LO DEVUELVA POR REFERENCIA
[Devuelve el mapa del juego]
{Es por referencia, produce aliasing}


%jugadores : juego −→ conj(jugador)
\InterfazFuncion{Jugadores}{\In{j}{juego}, \In{e}{jugador}}{ITERADOR}%
[true]
{$TENGO QUE DEVOLVER ITERADOR VER ENUNCIADO$}%
[$O(1)$] % por enunciado. "Ver que siguiente no tiene por que ser O(1)"
[Devuelve un iterador al jugador. Puede borrarse]



%estaConectado : jugador e × juego j −→ bool  {e ∈ jugadores(j)}
\InterfazFuncion{EstaConectado}{\In{j}{juego}, \In{e}{jugador}}{bool}%
[$e \in jugadores(j)$]
{$res = estaConectado(e, j)$}%
[$O(1)$] % por enunciado. "Ver que siguiente no tiene por que ser O(1)"
[Devuelve true si el jugador esta conectado]



%sanciones : jugador e × juego j −→ nat {e ∈ jugadores(j)}
\InterfazFuncion{Sanciones}{\In{e}{jugador}, \In{j}{juego}}{nat}%
[$e \in jugadores(j)$]
{$res = sanciones(e, j)$}%
[$O(1)$] % por enunciado. 
[Devuelve la cantidad de sanciones de un jugador]



%posicion : jugador e × juego j −→ coor  {e ∈ jugadores(j) ∧ L estaConectado(e, j)}
\InterfazFuncion{Posicion}{\In{j}{juego}, \In{e}{jugador}}{coor}%
[$e \in jugadores(j)  \yluego estaConectado(e, j)$]
{$res = posicion(e, j)$}%
[$O(1)$] % por enunciado. 
[Devuelve la posicion actual de un jugador]



%pokémons : jugador e × juego j −→ multiconj(pokémon) {e ∈ jugadores(j)}
%VER
\InterfazFuncion{Pokemons}{\In{j}{juego}, \In{e}{jugador}}{multiconj}%
[$e \in jugadores(j)$]
{$res = ITERADOR?? VER ENUNCIADO POCO CLARO$}%
[$O(1)$] % por enunciado. 
[FRUTA]
{TENEMOS QUE HACER EL MULTICONJUNTO? ITERADOR A TUPLA POR ENUNCIADO? VER SI ESTA EN MODS BASICOS SINO QUE SEA VECTOR Y CHAU}



%expulsados : juego −→ conj(jugador)
\InterfazFuncion{Expulsados}{\In{j}{juego}}{conj(jugador)}%
[true]
{$res = expulsados(j)$}%
[$O(J)$]  
[Devuelve un conjunto con los jugadores expulsados]


%posConPokémons : juego −→ conj(coor)
\InterfazFuncion{PosConPokemons}{\In{j}{juego}}{conj(coor)}%
[true]
{$res = posConPokemons(j)$}%
[$O(1)$] % Devuelve por referencia el conjunto, asi que habria que ir construyendolo
[Devuelve un conjunto con las posiciones del mapa que tienen pokemons. Aliasing: devuelve el conjunto por referencia]
% ACA DEVOLVEMOS LAS COORDENADAS DE LOS POKENODOS



%pokémonEnPos : coor c × juego j −→ pokémon {c ∈ posConPokémons(j)}
\InterfazFuncion{PokemonEnPos}{\In{j}{juego}, \In{c}{coor}}{pokemon}%
[$c \in posConPokemons(j)$]
{$res = pokemonEnPos(c, j)$}%
[$O(1)$] % Ibamos a guardar el pokemon en la tupla del pokenodo
[Devuelve el pokemon que se encuentra en la posicion c. Sin aliasing, devuelve por copia]


%cantMovimientosParaCaptura : coor c × juego j −→ nat {c ∈ posConPokémons(j)}
\InterfazFuncion{CantMovimientosParaCaptura}{\In{c}{coor}, \In{j}{juego}}{nat}%
[$c \in posConPokemons(j)$] %requiere que c sea pos de pokenodo
{$res = cantMovimientosParaCaptura(c, j)$}%
[$O(1)$] % Ibamos a guardar el contador en la tupla del pokenodo
[Devuelve el numero de movimientos que indican cuando se captura un pokemon]


% otras operaciones publicas

% puedoAgregarPokémon : coor c × juego j −→ bool
\InterfazFuncion{PuedoAgregarPokemon}{\In{c}{coor}, \In{j}{juego}}{bool}%
[true] 
{$res = puedoAgregarPokemon(c, j)$}  % Ver que mucho no me convence
[$O(1)$] 
[Devuelve verdadero si la coordenada es valida y no hay ningun pokemon en el territorio]


% hayPokémonCercano : coor c × juego j −→ bool
\InterfazFuncion{HayPokemonCercano}{\In{c}{coor}, \In{j}{juego}}{bool}%
[true] 
{$res = hayPokemonCercano(c, j)$}  % Ver que mucho no me convence
[$O(1)$] 
[Devuelve verdadero si hay algun pokemon en el territorio]


% posPókemonCercano : coor c × juego j −→ coor {hayPokémonCercano(c,j)}
\InterfazFuncion{PosPokemonCercano}{\In{c}{coor}, \In{j}{juego}}{coor}%
[hayPokemonCercano(c, j)] 
{$res = posPokemonCercano(c, j)$}  % Ver que mucho no me convence
[$O(1)$] 
[Devuelve la posicion del pokemon que esta en territorio]


% entrenadoresPosibles : coor c × conj(jugador) es × juego j −→ conj(jugador)
% {hayPokémonCercano(c,j) ∧ es ⊆ jugadoresConectados(j)}
\InterfazFuncion{EntrenadoresPosibles}{\In{c}{coor},\In{es}{conj(jugador)}, \In{j}{juego}}{conj(jugador)}%
[hayPokemonCercano(c, j) $\land$ es $\subset$ jugadoresConectados(j)] 
{$res = entrenadoresPosibles(c, es, j)$}  % Ver que mucho no me convence
[$O(1)$] % Se supone que vamos a guardar los posibles en un "AVL" en la coordenada del pokenodo
[Devuelve un conjunto con los entrenadores que estan en condiciones de capturar el pokemon que se encuentra en el rango de c. Que esten en condiciones de capturar significa que estan en rango2 del pokemon y que existe un camino hacia el]


% indiceRareza : pokémon p × juego j −→ nat  {p ∈ todosLosPokémons(j)}
\InterfazFuncion{IndiceRareza}{\In{p}{pokemon}, \In{j}{juego}}{nat}%
[p $\in$ todosLosPokemons(j)] 
{$res = indiceRareza(p, j)$}  % Ver que mucho no me convence
[$O(|P|)$] % es una cuenta, obtener los numeros de p nos cuesta recorrer el trie
[Devuelve el indice de rareza del pokemon dado]


% cantPokémonsTotales : juego −→ nat
\InterfazFuncion{cantPokemonsTotales}{\In{j}{juego}}{nat}%
[true] 
{$res = cantPokemonsTotales(p, j)$}  % Ver que mucho no me convence
[$O(1)$] % tiene que ser O(1) para que el indice de rareza lo calculemos rapido, 
% Podemos guardar el dato como un valor de la estructura del juego
[Devuelve la cantidad de pokemons totales del juego]


% cantMismaEspecie : pokémon × multiconj(pokémon) −→ nat

\InterfazFuncion{cantMismaEspecie}{{\In{p}{pokemon}, \In{j}{juego}}{nat}
[$p$ $\in$ todosLosPokemons($j$)] 
{$res$ = cantMismaEspecie($p$, $j$)}  % Ver que mucho no me convence
[$O(|p|)$] % tiene que ser O(1) para que el indice de rareza lo calculemos rapido, 
% Podemos guardar el dato como un valor de la estructura del juego
[Devuelve la cantidad de pokemons totales del juego]


\end{Interfaz}

~ 

\begin{Representacion}

% se representa con una tupla(variable, variable, pokenodos, )
 
\begin{Estructura}{Juego}[pokgo]

\begin{Tupla}[pokgo]  
    \tupItem{cantPokemon}{diccTrie}  % Tendria la cantidad total de cada pokemon
    \tupItem{\\ cantPokemonsTotales}{nat} % Para acceso O(1) al total
    \tupItem{\\ map}{mapa}  % Tendria el mapa con el que se crea
    \tupItem{\\ jugadores}{vector(jugStruc)}  % Tendria los datos de los jugadores, o iteradores, VER
    \tupItem{\\ grillaJugs}{vector(vector(lista(jugador)))} % Una grilla R2 del tamaño del mapa donde en cada posicion hay una lista de los jugadores que estan en esa posicion
    \tupItem{\\ pokenodos}{vector(vector(puntero(pokeStruc)))}  % Una grilla R2 del tamaño del mapa donde en cada coordenada puede haber un pokenodo (ver en REP que tienen que estar a cierta distancia)
\end{Tupla}
\end{Estructura}

    \begin{Tupla}[pokeStruc]
      \tupItem{poke}{pokemon}
      \tupItem{\\ contador}{nat}%
      \tupItem{\\ entrenadores}{Heap(tupla<id: nat, cantCapt: nat>)} % Deberia ser como sea que se llame el modulo, cola_de_prioridadJug o pokeHeap, nose, pero en el fondo seria un heap
    \end{Tupla}
    
    
    \begin{Tupla}[jugStruc]
      \tupItem{id}{nat}
      \tupItem{\\ sanciones}{nat}%
      \tupItem{\\ conectado}{bool}
      \tupItem{\\ pos}{coor}
      \tupItem{\\ pokemons}{lista(tupla<pokemon, cantidad>)} % Hay que ver como nos conviene mantener a los pokemones que capturo el jugador para que no nos arruine la complejidad    
      \tupItem{\\ iteradorAHeap}{itHeap(tupla<id: nat, cantCapt: nat>)} % Iterador al heap
      \tupItem{\\ iteradorAPos}{itLista(nat)}   
      \tupItem{\\ cantCapt}{nat}
    \end{Tupla}


$ $\newline
$ $\newline

\Title{Invariante de representacion}

$ $\newline

(1) El indice de la posicion del vector es igual al id del juegador en ese indice (de esto se desprende que los ids son unicos) $ $\newline

($\forall i$: nat)((i $<$ Longitud(j.jugadores)) $\impluego$ j.jugadores[$i$].id = i )
$ $\newline


(2) Todo jugador de j.jugadores que este conectado, tiene una posicion que es una coordenada existente en el mapa $ $\newline

%($\forall e$: jugador) ((esJugadorConectado($e$)) $\impluego$ PosExistente(j.jugadores[$e$].pos, j.mapa))
($\forall jug$: jugStruc) ((esJugadorConectado($jug$)) $\impluego$ PosExistente($jug$.pos, j.mapa))
$ $\newline


(3) Dimensiones de la grillaJugs (vector de vectores) es igual al tama\~no del mapa $ $\newline

Longitud($j$.grillaJugs) = Tam($j$.mapa) $\yluego$ 
($\forall i$: nat) (($i <$ Logitud(j.grillaJugs)) Longitud($j$.grillaJugs[i]) = Tam($j$.mapa)) 
$ $\newline


(4) No hay elementos repetidos en las listas de grillaJugs  $ $\newline

($\forall x, y$: nat) ((enRango($x$, $y$, $j$.mapa) $\impluego$ sinRepetidos($j$.grillaJugs[$x$][$y$]))
$ $\newline

(5) Todo jugador que esta conectado tiene su id en la lista que se encuentra en grillaJugs para su posicion  $ $\newline

%($\forall e:$ jugador) (esJugadorConectado($e$, $j$) $\impluego$ \\
%	$j$.jugadores[$e$].id $\in$ $j$.grillaJugs[latitud($j$.jugadores[$e$])][longitud($j$.jugadores[$e$])])

($\forall jug:$ jugStruc) (esJugadorConectado($jug$, $j$) $\impluego$ \\
	$jug$.id $\in$ $j$.grillaJugs[latitud($jug$.pos)][longitud($jug$.pos)])
	
$ $\newline

(6) Toda id en toda lista de grillaJugs es un de un jugador del juego que este conectado  $ $\newline

($\forall x, y:$ nat) (enRango($x$, $y$, $j$.mapa) $\impluego$ \\
	($\forall i:$ nat) ($i \leq$ Longitud($j$.grillaJugs[$x$][$y$]) \\
		$j$.jugadores($j$.grillaJugs[$x$][$y$][$i$]).conectado

$ $\newline

(7) iteradorAPos apunta al elemento correcto (misma id) en la lista de grillaJugs correspondiente a su pos $ $\newline

($\forall jug:$ jugStruc) ( esJugadorConectado($jug$, $j$) $\impluego$ Siguiente($jug$.iteradorAPos) = $jug$.id )

$ $\newline

(8) cantCapt es consistente con las cantidades de su lista de pokemons capturados $ $\newline
$ $\newline
($\forall jug:$ jugStruc) (esJugadorNoEliminado($jug$, $j$) $\impluego$ $jug$.cantCapt = sumaCantidades($jug$.pokemons))
$ $\newline


(9) cantPokemonTotales es igual a la sumatoria de todos los significados del diccionario $ $\newline
$j$.cantPokemonTotales = sumaSignif($j$.cantPokemon)
$ $\newline
%(MMMMM NO, creo que se deduce) Los pokemons que un jugador (NO eliminado) capturo, esta en las claves del diccionario del juego con un significado $>$ 0 $ $\newline

(10) Para todo pokemon del diccionario, la cantidad que hay es igual a la suma de los salvajes mas los capturados por jugadores (no eliminados) $ $\newline
($\forall p: $ pokemon) (($p \in$ Claves($j$.cantPokemon)) $\impluego$ \\
	Obtener($p$, $j$.cantPokemon) = cantSalvajes($p$, $j$) $+$ sumaPokesCapturados($p$, $j$))

$ $\newline

(11) Para todo pokemon salvaje, su cantidad es la resta entre la cantidad del trie menos los capturados por jugadores (no eliminados) $ $\newline

($\forall p: $ pokemon) (($p \in$ Claves($j$.cantPokemon)) $\impluego$ \\ 
	(cantSalvajes($p$, $j$) = Obtener($p$, $j$.cantPokemon) $-$ sumaPokesCapturados($p$, $j$)))
$ $\newline


(12) Dimensiones de pokenodos (vector de vectores) es igual al tama\~no del mapa $ $\newline

longitud($j$.pokenodos) = tam($j$.mapa) $\yluego$ 
($\forall i$: nat) (($i <$ Logitud(j.pokenodos)) Longitud($j$.pokenodos[i]) = Tam($j$.mapa)) 
$ $\newline

(13) Todo pokenodo que tenga un pokestruc, esta en una coordenada valida del mapa $ $\newline

($\forall x, y$: nat) ((enRango($x$, $y$, $j.mapa$) $\yluego$ $j$.pokenodos[$x$][$y$] $\neg$ NULL) \\
	$\impluego$ posExistente(crearCoordenada(x,y), $j$.mapa))
$ $\newline


(14) No hay pokenodos con pokestrucs que esten a distancia menor a 5 $ $\newline

($\forall x, y$: nat) ((enRango($x$, $y$, $j.mapa$) $\yluego$ $j$.pokenodos[$x$][$y$] $\neq$ NULL) $ \land$\\
($\forall z, w$: nat) ((enRango($z$, $w$, $j.mapa$) $\yluego$ $j$.pokenodos[$z$][$w$] $\neq$ NULL) $\land$ \\
($x \neq z \land y \neq w$) $\impluego$  distEuclidea(crearCoordenada($x$, $y$), crearCoordenada($z$, $w$)) > 25)

$ $\newline

(15) El contador de todo pokenodo es $<$ 10 $ $\newline

($\forall x, y$: nat) ((enRango($x$, $y$, $j.mapa$) $\yluego$ $j$.pokenodos[$x$][$y$] $\neq$ NULL) \\
	$\impluego$ (*($j$.pokenodos[$x$][$y$])).contador < 10)

$ $\newline

(16) Todo pokestruc tiene un pokemon que esta bien definido $ $\newline
($\forall x, y$: nat) ((enRango($x$, $y$, $j.mapa$) $\yluego$ $j$.pokenodos[$x$][$y$] $\neq$ NULL) \\
	$\impluego$ Def?((*($j$.pokenodos[$x$][$y$])).poke), $j$.cantPokemon)

$ $\newline

% jonathan: creo que con solo hablar del minimo del heap es suficiente, porque es lo unico que necesitamos en realidad. El resto de los jugadores del heap estan solo para poder actualizar al minimo rapido

(17) Para todas los pokenodos con pokemons, de todos jugadores validos, conectados, que esten en un radio menor a 2, con un camino a la posicion del pokemon, el que tiene menos cantidad de pokemons capturados (y menor id en caso de empate) se corresponde con el Proximo del Heap $ $\newline

($\forall x, y$: nat) ((enRango($x$, $y$, $j.mapa$) $\yluego$ $j$.pokenodos[$x$][$y$] $\neq$ NULL) \\
	(minJ(entrenadoresPosibles(crearCoordenada($x$, $y$), jugadoresConectados($j$), $j$) = \\
	(proximo(*($j$.pokenodos[$x$][$y$]).entrenadores)).id) $\land$ \\
	((proximo(*($j$.pokenodos[$x$][$y$]).entrenadores)).cant = $j$.jugadores[proximo(*($j$.pokenodos[$x$][$y$]).entrenadores)).id].cantcapt))
	
$ $\newline

(18) Todo jugador valido conectado que tenga un pokemon cercano, si tiene un camino hace ese pokemon entonces su iterador a heap esta bien definido $ $\newline

($\forall jug$: jugStruc) (esJugadorConectado($jug$, $j$) $\land$ hayPokemonCercano($jug$.pos, $j$) $\impluego$ \\
hayCamino(posPokemonCercano($j.pos$, $j$), $j$) $\implies$ \\
siguiente($jug$.iterAHeap).id = $jug$.id $\land$ siguiente($jug$.iterAHeap).cant = $j$.cantCapt) 

$ $\newline

% Auxiliares

%\tadOperacion{esContigua}{nat$/x$, nat$/y$}{bool}{}%
%\tadAxioma{esContigua($x, y$)}{contenido} 
%$ $\newline

\tadOperacion{esJugadorConectado}{jugStruc$/jug$, juego$/j$}{bool}{}%
\tadAxioma{esJugadorConectado($jug, j$)}
{$jug \in j$.jugadores $\yluego$ $jug$.conectado} 
$ $\newline

\tadOperacion{esJugadorNoEliminado}{jugStruc$/jug$, juego$/j$}{bool}{}%
\tadAxioma{esJugadorNoEliminado($jug$, $j$)}
{$jug \in j$.jugadores $\yluego$ $jug$.sanciones < 5} 
$ $\newline

\tadOperacion{enRango}{nat$/x$, nat$/y$, juego$/j$}{bool}{}%
\tadAxioma{enRango($x, y, j$)}
{posExistente(crearCoordenada($x$, $y$), $j$.mapa)}
%{$x$ < tam($j$.mapa) $\land$ $y$ < tam($j$.mapa) $\land$} 
$ $\newline

\tadOperacion{sumaCantidades}{secu(tupla<p: pokemon, cant: nat>)}{nat}{}%
\tadAxioma{sumaCantidades(<>)}{0} 
\tadAxioma{sumaCantidades(a $\puntito$ s)}{a.cant + sumaCantidades(s)}
$ $\newline

\tadOperacion{sumaSignif}{dicc(string, nat)}{nat}{}%
\tadAxioma{sumaSignif($d$)}{sumaSignifAux(claves($d$), $d$)} 
$ $\newline

\tadOperacion{sumaSignifAux}{conj(string), dicc(string, nat)}{nat}{}%
\tadAxioma{sumaSignifAux($cs$, $d$)}
{\IF $\emptyset$?(cs) THEN 0 ELSE obtener(dameUno($cs$), $d$) + sumaSignifAux(sinUno($cs$), $d$) FI} 
$ $\newline

\tadOperacion{cantSalvajes}{pokemon$/p$, juego$/j$}{nat}{}%
\tadAxioma{cantSalvajes($p, j$)}
{$\#$($p$, pokemonsSalvajes(posConPokemons($j$)))} 
$ $\newline

\tadOperacion{sumaPokesCapturados}{pokemon$/p$, juego$/j$}{nat}{}%
\tadAxioma{sumaPokesCapturados($p$, $j$)}{sumaPokesCapturadosAux($p$, $j$, jugadoresConectados($j$))} 
$ $\newline

\tadOperacion{sumaPokesCapturadosAux}{pokemon$/p$, juego$/j$, conj(jugador)$/js$}{nat}{}%
\tadAxioma{sumaPokesCapturadosAux($p$, $j$, $js$)}
{\IF $\emptyset$?($js$) THEN 0 ELSE  
$\#$($p$, dameUno($js$).pokemons) + sumaPokesCapturadosAux($p$, $j$, sinUno($js$))
FI} 
$ $\newline

% REP  
% REVISAR LOS YLUEGOS

\tadOperacion{Rep}{juego$/j$}{bool}{}%
\tadAxioma{Rep($j$)}
{(1) $\yluego$ (2) $\land$ (3) $\yluego$ (4) $\land$ (5) $\land$ (6) $\yluego$ (7) $\land$ (8) $\land$ (9) $\yluego$\\
 (10) $\land$ (11) $\land$ (12) $\yluego$ (13) $\land$ (14) $\land$ (15) $\land$ (16) $\yluego$ (17) $\yluego$ (18)}
%{{(1) $\yluego$ (2) $\yluego$ (3) $\land$ (4) $\land$ (5) $\land$ (6)}} 

$ $\newline
$ $\newline


% ABS

\tadOperacion{Abs}{juego$/j$}{Juego}{Rep($j$)}%
\tadAxioma{Abs($j$)}
{INSERTAR ABS}%

\end{Representacion}

\begin{Algoritmos}


% crearJuego
\begin{algorithm}[H]
{\textbf{iCrearJuego}(\In{map}{mapa}) $\to$ $res$ : juego}
\begin{algorithmic}[1]

\State dictTrie $cantPokemon \gets $ Vacio()  \Comment $O(1)$
\State nat $cantPokemonsTotales \gets $ 0  \Comment $O(1)$
$ $\newline
\State vector(jugStruc)  $jugs \gets$ Vacio()     \Comment $O(1)$

% Creo la grilla de listas de posiciones (con las listas vacias)
\State vector(vector(lista(jugador)))   $grillaJugs$
$ $\newline
\For{$i \gets 0$ to Tam($map$) $-$ 1} \Comment Se repite Tam(map) veces $O(1)$
    
    \State vector(lista(jugador)) $vectorInterno \gets$ Vacio()  \Comment $O(1)$
    
    \For{$j \gets 0$ to Tam($map$) $-$ 1}  \Comment Se repite Tam(map) veces $O(1)$
        \State lista(jugador) $jugsVacia \gets$ Vacia()   \Comment $O(1)$
        \State AgregarAtras($vectorInterno$, $jugsVacia$) \Comment $O(1)$    
    \EndFor
    
    AgregarAtras($jugs$, $vectorInterno$)   \Comment $O(1)$

\EndFor
$ $\newline
\State vector(vector(pokeStruc))  $pokenodos \gets$ Vacio()  \Comment $O(1)$

\For{$i \gets 0$ to Tam($map$) $-$ 1}       \Comment Se repite Tam(map) veces $O(1)$
    \State vector(puntero(pokeStruc)) $vectorInterno \gets$ Vacio()  \Comment $O(1)$
    \For{$j \gets 0$ to Tam($map$) $-$ 1}    \Comment Se repite Tam(map) veces $O(1)$
        \State puntero(pokeStruc) $pokePuntero \gets$ NULL   \Comment $O(1)$
        \State AgregarAtras($vectorInterno$, $pokePuntero$) \Comment $O(1)$
    \EndFor
    AgregarAtras($pokenodos$, $vectorInterno$)   \Comment $O(1)$
\EndFor

\State $res \gets \langle cantPokemon$, $cantPokemonsTotales$, $map$, $jugs$, $grillaJugs$, $pokenodos \rangle$   \Comment $O(1)$

\medskip
\Statex \underline{Complejidad:} $O((Tam(map))^2)$
\Statex \underline{Justificaion:} Se crean 2 vectores de vectores, de Tam(map) elementos tanto el vector interno como el externo $O((Tam(map))^2) + O((Tam(map))^2) = O((Tam(map))^2)$. Crear vectores vacios cuesta O(1) y el mapa lo pasamos por referencia, $O(1) + O((Tam(map))^2) = O((Tam(map))^2)$

\end{algorithmic}
\end{algorithm}


\begin{algorithm}[H]
{\textbf{iAgregarJugador}(\In{j}{juego}) $\to$ $res$ : nat}
\begin{algorithmic}[1]

\State nat $proxId$ $\gets$ Longitud($j$.jugadores)   \Comment $O(1)$

%\State $nat sanciones \gets 0$ 
%\State $bool conectado \gets false$

% Le pongo (0,0) pero en realidad no importa porque no tenemos que chequear la posicion si esta desconectado)
\State coor $pos$ $\gets$ CrearCoor(0,0) \Comment $O(1)$

\State vector($<$pokemon, nat$>$) $pokes$ $\gets$ Vacio() \Comment $O(1)$
 
% Le pongo 0 pero en realidad no nos importa porque no hay que usarlo si no tiene pokemon cerca
\State nat $indiceEnPokenodo$ $\gets$ 0  \Comment $O(1)$

% Creo el iterador de una lista vacia para poder crearlo
\State lista(nat) $listaDummy$ $\gets$ Vacia() \Comment $O(1)$
\State itLista(nat) $iteradorAPos$ $\gets$ CrearIt($listaDummy$) \Comment $O(1)$

\State $j$.jugadores.AgregarAtras($<proxId$, 0, $false$, $pos$, $pokes$, $indiceEnPokenodo$, $iteradorAPos>$) \Comment $O(1)$

\State $res \gets proxId$ \Comment $O(1)$

\medskip
\State \underline{Complejidad:} $O(1)$
\State \underline{Justificaion:} Todas las operaciones son O(1)

\end{algorithmic}
\end{algorithm}

\begin{algorithm}[H]
{\textbf{iConectarse}(\Inout{p}{juego}, \In{c}{coordenada}, \In{j}{jugador})}
\begin{algorithmic}[1]

\State $p$.jugadores[\hspace{0.5mm}j\hspace{0.5mm}].conectado $\gets$ true \Comment $O(1)$
\State itLista(nat) $it$ $\gets$ AgregarAtras($p$.grillaJugs[latitud(c)][longitud(c)], $j$) \Comment $O(1)$
\State $p$.jugadores[j].iteradorAPos $\gets$ $it$ \Comment $O(1)$
\State $p$.jugadores[j].pos $\gets$ $c$
\If {$HayPokemonCercano(c, p) \land HayCamino(c, PosPokemonCercano(c, p), Mapa(p))$} \Comment $O(1)$
    \State nat $latPok$ $\gets$ latitud(PosPokemonCercano($c$, $p$)) \Comment $O(1)$
    \State nat $lonPok$ $\gets$ longitud(PosPokemonCercano($c$, $p$)) \Comment $O(1)$
    \State ($p$.pokenodos[$latPok$][$lonPok$] $\rightarrow$ contador) $\gets$ 0 \Comment $O(1)$
    \State tupla<id: nat, cantCap: nat> $t \gets$ <j, $p$.jugadores[j].camtCap> 
    \State $itpokenodo \gets Encolar((p.pokenodos[latPok][lonPok] \rightarrow entrenadores) , t)$
\Comment $O(log(EC))$
    \State $p$.jugadores[j].iteradorAHeap $\gets$ $itpokenodo$ \Comment $O(1)$
\EndIf

\medskip
\State \underline{Complejidad:} $O(log(EC))$ 
\State \underline{Justificaion:} Todas las operaciones de asignaci\'on, acceso a posiciones de vectores y desreferenciaci\'on de punteros son O(1). Las funciones "HayPokemonCercano", "HayCamino", "PosPokemonCercano", "Mapa"\hspace{0.5mm}son O(1). La funci\'on AgregarAtras de lista enlazada es O(1). La funci\'on de Heap Encolar es $O(log(EC))$ en el peor caso, y como 1 $\leq$ $log(EC)$, por \'algebra de \'ordenes, sumando todos los costos da que el algoritmo es $O(log(EC))$. Donde EC es la cantidad m\'axima de jugadores esperando a capturar un pokemon, y por lo tanto, la cantidad m\'axima de elementos del heap.


\end{algorithmic}
\end{algorithm}

\begin{algorithm}[H]
{\textbf{iDesconectarse}(\Inout{p}{juego}, \In{j}{jugador})}
\begin{algorithmic}[1]

\State $p$.jugadores[\hspace{0.5mm}j\hspace{0.5mm}].conectado $\gets$ false \Comment $O(1)$
\State EliminarSiguiente($p$.jugadores[j].iteradorAPos) \Comment $O(1)$
\State $it \gets$ $<NULL, NULL>$ \Comment $O(1)$
\State $p$.jugadores[j].iteradorAPos $\gets$ $it$ \Comment $O(1)$
\State $c \gets$ $p$.jugadores[j].pos 
\If {$HayPokemonCercano(c, p) \land HayCamino(c, PosPokemonCercano(c, p), Mapa(p))$} \Comment $O(1)$
    \State nat $latPok$ $\gets$ latitud(PosPokemonCercano($c$, $p$)) \Comment $O(1)$
    \State nat $lonPok$ $\gets$ longitud(PosPokemonCercano($c$, $p$)) \Comment $O(1)$
    \State Borrar($p$.jugadores[j].iteradorAHeap, ($p$.pokenodos[latPok][lonPok] $\rightarrow$ entrenadores)) \Comment $O(log(EC))$
    \State itHeap(tupla<id: nat, cantCap: nat>) $ith \gets$ $<NULL, NULL>$ \Comment $O(1)$
    \State $p$.jugadores[j].iteradorAHeap $\gets ith$ \Comment $O(1)$
\EndIf
\medskip
\State \underline{Complejidad:} $O(log(EC))$ 
\State \underline{Justificaion:} Todas las operaciones de asignaci\'on, acceso a posiciones de vectores y desreferenciaci\'on de punteros son O(1). Las funciones "HayPokemonCercano", "HayCamino", "PosPokemonCercano", "Mapa"\hspace{0.5mm}son O(1). La funci\'on EliminarSiguiente del iterador de lista es $O(1)$. La funci\'on de itHeap Borrar es $O(log(EC))$ en el peor caso, y como 1 $\leq$ $log(EC)$, por \'algebra de \'ordenes, sumando todos los costos da que el algoritmo es $O(log(EC))$. Donde EC es la cantidad m\'axima de jugadores esperando a capturar un pokemon, y por lo tanto, la cantidad m\'axima de elementos del heap.


\end{algorithmic}
\end{algorithm}

% agregarPokemon

\begin{algorithm}[H]
{\textbf{iAgregarPokemon}(\In{p}{pokemon}, \In{c}{coord}, \Inout{j}{juego})}
\begin{algorithmic}[1]

\State $j$.cantPokemonsTotales $\gets$ $j$.cantPokemonsTotales + 1 \Comment $O(1)$

% Lo defino bien
\If{Def?($p$, $j$.cantPokemon)} \Comment $O(|P|)$
	\State Definir($j$.cantPokemon, $p$, Obtener($p$, $j$.cantPokemon) + 1) \Comment $O(|P|)$
\EndIf
\If{$\neg$ Def?($p$, $j$.cantPokemon)}  \Comment $O(|P|)$
	\State Definir($j$.cantPokemon, $p$, 1)  \Comment $O(|P|)$
\EndIf

\State heap(tupla<id: nat, cantCapt: nat>) $h \gets$ Vacia() \Comment $O(1)$
\State vector<jugador> \hspace{0.5mm} posiblesCapturadores $\gets$ DameJugadoresEnPokenodo(j, c) \Comment $O(EC)$
\State nat $i \gets$ 0 \Comment $O(1)$

\While{$i$ <\hspace{0.5mm} Longitud($posiblesCapturadores$)} \Comment $O(EC*log(EC))$
	\State $aInsertar$ $\gets$ <posiblesCapturadores[$i$], $j$.jugadores[posiblesCapturadores[$i$]].cantCapt> \Comment $O(1)$
	\State itheap(tupla<id: nat, cantCapt: nat>) it $\gets$ Encolar($h$, $aInsertar$) \Comment $O(log(EC))$
	\State $j$.jugadores[posiblesCapturadores[$i$]].iteradorAHeap $\gets$ $it$ \Comment $O(1)$
	\State $i \gets i$ + 1 \Comment $O(1)$
\EndWhile

\State pokeStruc $pok$ $\gets$ <$p$, 0, $h$> \Comment $O(1)$
\State puntero(pokeStruc) puntPok $\gets$ $pok$ \Comment $O(1)$
\State $j$.pokenodos[latitud($c$)][longitud($c$)] $\gets$ puntPok \Comment $O(1)$

\medskip
\State \underline{Complejidad:} $O(EC*log(EC) + |P|)$ 
\State \underline{Justificaion:} Las operaciones de asignaci\'on, acceso a posiciones y desreferenciaci\'on de punteros son $O(1)$. La funci\'ones Def, Definir y Obtener son $O(|P|)$ donde |P| es la longitud del pokemon m\'as largo. La operaci\'on Encolar de Heaps es $O(log(EC))$ donde EC es la cantidad m\'axima de jugadores en un pokenodo, y por ende, la cantidad m\'axima de elementos del Heap. DameJugadoresEnPokenodo es $O(EC)$ y devuelve un vector de EC elementos como m\'aximo. En el while se realiza una iteaci\'on por cada elemento de dicho vector, por lo tanto en el peor caso se realizan EC iteraciones, y como dentro del ciclo hay 3 funciones $O(1)$ y una $O(log(EC))$, la cantidad de operaciones que realiza el while hasta terminar, en el peor caso es $O(EC*log(EC))$. Sumando los costos de las operaciones independientes, por \'algebra de \'ordenes queda que el costo del algoritmo es de $O(1)$ + $O(|P|)$ + $O(1)$ + $O(EC)$ + $O(1)$ + $O(EC*log(EC))$ + $O(1)$ + $O(1)$ + $O(1)$ = $O(|P|)$ + $O(EC)$ + $O(EC*log(EC))$ = $O(|P|)$ + $O(EC*log(EC))$ (ya que EC $\leq$ EC*log(EC)) = $O(|P| + EC*log(EC))$

\end{algorithmic}
\end{algorithm}

% cantPokemonsTotales
\begin{algorithm}[H]
{\textbf{iCantPokemonTotales}(\In{j}{juego}) $\to$ $res$ : nat}
\begin{algorithmic}[1]

\State $res \gets$ $j$.cantPokemonsTotales \Comment $O(1)$

\medskip
\State \underline{Complejidad:} $O(1)$

\end{algorithmic}
\end{algorithm}

% indiceRareza

\begin{algorithm}[H]
{\textbf{iIndiceRareza}(\In{p}{pokemon}, \In{j}{juego}) $\to$ $res$ : nat}
\begin{algorithmic}[1]

\State nat $pokecant \gets$ Obtener(j.cantPokemon, $p$) \Comment $O(|P|)$

\State $res \gets$ 100 $-$ (100 $*$ $pokecant$ $/$ j.cantPokemonsTotales) \Comment $O(1)$

\medskip
\State \underline{Complejidad:} $O(|P|)$
\State \underline{Justificacion:} j.cantPokemon es un diccionario sobre trie. La complejidad de buscar (y obtener el significado) en peor caso es la longitud de la string mas larga entre sus claves, eso es O($|P|$). j.cantPokemonsTotales es un dato guardado en la estructura del juego, y se accede en O(1). El resto son una resta, multiplicacion y division, que tambien son O(1). O($|P|$) $+$ O(1) $=$ O($|P|$)

\end{algorithmic}
\end{algorithm}

% Posicion

\begin{algorithm}[H]
{\textbf{iPosicion}(\In{j}{juego}, \In{e}{jugador}) $\to$ $res$ : coor}
\begin{algorithmic}[1]

\State $res \gets$ $j$.jugadores[$e$].pos \Comment $O(1)$

\medskip
\State \underline{Complejidad:} $O(1)$
\State \underline{Justificacion:} Todas las operaciones son $O(1)$


\end{algorithmic}
\end{algorithm}


% Sanciones

\begin{algorithm}[H]
{\textbf{iSanciones}(\In{e}{jugador}, \In{j}{juego}) $\to$ $res$ : nat}
\begin{algorithmic}[1]

\State $res \gets$ $j$.jugadores[$e$].sanciones \Comment $O(1)$

\medskip
\State \underline{Complejidad:} $O(1)$
\State \underline{Justificacion:} Todas las operaciones son $O(1)$


\end{algorithmic}
\end{algorithm}

% entrenadoresPosibles
% por precondicion estan conectados

\begin{algorithm}[H]
{\textbf{iEntrenadoresPosibles}(\In{c}{coor}, \In{ec}{conj(jugador)}, \In{j}{juego}) $\to$ $res$ : conj(jugador)}
\begin{algorithmic}[1]

\State coor $pokeCoor \gets$ PosPokemonCercano($c$, $j$) \Comment $O(1)$
\State puntero(pokeStruc) $pokePuntero \gets j.pokeNodos[longitud(pokeCoor)][latitud(pokeCoor)]$ \Comment $O(1)$

\State itConj(jugador) $itPosibles \gets$ CrearIt($ec$) \Comment $O(1)$

\While{HaySiguiente($itPosibels$)}   \Comment O($Cardinal(ec) * O(HaySiguiente) * Cardinal(jugsEnNodo)$)
    
    % Asumo que tengo una forma de recorrer el heap
    \State itHeap $itJugsEnNodo \gets$ CrearIt(($*pokePuntero$).entrenadores)     \Comment Depende del HEAP $O(Inifinito)$
    
    \While{HaySiguiente($itJugsEnNodo$)}   \Comment O($HaySiguiente * \#jugsEnNodo$)

        \If{Siguiente($itPosibles$) = Siguiente($itJugsEnNodo$)}   \Comment Depende del HEAP $O(Inifinito)$
        
            \State Agregar($res$, Siguiente($itPosible$))
    
        \EndIf
        
        \State Avanzar($itJugsEnNodo$))
    
    \EndWhile
    
    \State Avanzar($itPosible$)   \Comment()



\EndWhile


\medskip
\State \underline{Complejidad:} $O(|P|)$
\State \underline{Justificacion:} 

\end{algorithmic}
\end{algorithm}


$ $\newline
$ $\newline
$ $\newline
% hayPokemonCercano
%\begin{algorithm}[H]
{\textbf{iHayPokemonCercano}(\In{c}{coor}, \In{j}{juego}) $\to$ $res$ : coor}
\begin{algorithmic}[1]

\State nat $x \gets$ latitud(c)    \Comment $O(1)$
\State nat $y \gets$ longitud(c)   \Comment $O(1)$

% j.pokenodos[x][y] != NULL   =>  pos con pokemon

\State bool $hayPokemon \gets$ false    \Comment $O(1)$

\If{j.pokenodos[$x$][$y$] $\neq$ NULL}     \Comment $O(1)$
    \State $hayPokemon \gets$ true
\EndIf

\If{$x > 0$}
    \If{j.pokenodos[$x-1$][$y$] $\neq$ NULL}     \Comment $O(1)$
        \State $hayPokemon \gets$ true
    \EndIf

    \If{$y > 0$}

        \If{j.pokenodos[$x-1$][$y-1$] $\neq$ NULL}     \Comment $O(1)$
            \State $hayPokemon \gets$ true
        \EndIf

    \EndIf

    % tam(m) no puede ser 0 porque existe una coordenada al menos (x,y)
    \If{$y < tam(m) - 1$}
        \If{j.pokenodos[$x-1$][$y+1$] $\neq$ NULL}     \Comment $O(1)$
            \State $hayPokemon \gets$ true
        \EndIf
    \EndIf

    \If{$x-1 > 0$}
        \If{j.pokenodos[$x-2$][$y$] $\neq$ NULL}     \Comment $O(1)$
            \State $hayPokemon \gets$ true
        \EndIf
    \EndIf

\EndIf

\If{$y > 0$}

    \If{j.pokenodos[$x$][$y-1$] $\neq$ NULL}     \Comment $O(1)$
        \State $hayPokemon \gets$ true
    \EndIf

    \If{$y-1 > 0$}

        \If{j.pokenodos[$x$][$y-2$] $\neq$ NULL}     \Comment $O(1)$
            \State $hayPokemon \gets$ true
        \EndIf

    \EndIf

\EndIf

\If{$y < tam(m) - 1$}

    \If{j.pokenodos[$x$][$y+1$] $\neq$ NULL}     \Comment $O(1)$
        \State $hayPokemon \gets$ true
    \EndIf


    \If{$tam(m) > 1 \land y < tam(m) - 2$}
        \If{j.pokenodos[$x$][$y+2$] $\neq$ NULL}     \Comment $O(1)$
            \State $hayPokemon \gets$ true
        \EndIf


    \EndIf

\EndIf

\If{$x < tam(m) - 1$}
    \If{j.pokenodos[$x+1$][$y$] $\neq$ NULL}     \Comment $O(1)$
        \State $hayPokemon \gets$ true
        \EndIf

    \If{$y > 0$}
        \If{j.pokenodos[$x+1$][$y-1$] $\neq$ NULL}     \Comment $O(1)$
            \State $hayPokemon \gets$ true
        \EndIf
    \EndIf

    \If{$y < tam(m)-1$}
        \If{j.pokenodos[$x+1$][$y+1$] $\neq$ NULL}     \Comment $O(1)$
            \State $hayPokemon \gets$ true
        \EndIf

    \EndIf
\EndIf

\If{$tam(m) > 1 \land x < tam(m) - 2$}
    \If{j.pokenodos[$x+2$][$y$] $\neq$ NULL}     \Comment $O(1)$
        \State $hayPokemon \gets$ true
    \EndIf
\EndIf

\State $res \gets hayPokemon$

\medskip
\State \underline{Complejidad:} $O(1)$ % por enunciado
\State \underline{Justificacion:} Reviso 13 posiciones O(1)

\end{algorithmic}
%\end{algorithm}

$ $\newline
$ $\newline
$ $\newline


% posPokemonCercano

%\begin{algorithm}[H]
{\textbf{iPosPokemonCercano}(\In{c}{coor}, \In{j}{juego}) $\to$ $res$ : coor}
\begin{algorithmic}[1]

\State nat $x \gets$ latitud(c)    \Comment $O(1)$
\State nat $y \gets$ longitud(c)   \Comment $O(1)$

% j.pokenodos[x][y] != NULL   =>  pos con pokemon

\State coor $coorConPokemon$    \Comment $O(1)$

\If{j.pokenodos[$x$][$y$] $\neq$ NULL}     \Comment $O(1)$
    \State $coorConPokemon \gets CrearCoor(x, y)$    \Comment $O(1)$
\EndIf

\If{$x > 0$}
    \If{j.pokenodos[$x-1$][$y$] $\neq$ NULL}     \Comment $O(1)$
        \State $coorConPokemon \gets CrearCoor(x, y)$    \Comment $O(1)$
    \EndIf

    \If{$y > 0$}

        \If{j.pokenodos[$x-1$][$y-1$] $\neq$ NULL}     \Comment $O(1)$
            \State $coorConPokemon \gets CrearCoor(x, y)$    \Comment $O(1)$
        \EndIf

    \EndIf

    % tam(m) no puede ser 0 porque existe una coordenada al menos (x,y)
    \If{$y < tam(m) - 1$}
        \If{j.pokenodos[$x-1$][$y+1$] $\neq$ NULL}     \Comment $O(1)$
            \State $coorConPokemon \gets CrearCoor(x, y)$    \Comment $O(1)$
        \EndIf
    \EndIf

    \If{$x-1 > 0$}
        \If{j.pokenodos[$x-2$][$y$] $\neq$ NULL}     \Comment $O(1)$
            \State $coorConPokemon \gets CrearCoor(x, y)$    \Comment $O(1)$
        \EndIf
    \EndIf

\EndIf

\If{$y > 0$}

    \If{j.pokenodos[$x$][$y-1$] $\neq$ NULL}     \Comment $O(1)$
        \State $coorConPokemon \gets CrearCoor(x, y)$    \Comment $O(1)$
    \EndIf

    \If{$y-1 > 0$}

        \If{j.pokenodos[$x$][$y-2$] $\neq$ NULL}     \Comment $O(1)$
            \State $coorConPokemon \gets CrearCoor(x, y)$    \Comment $O(1)$
        \EndIf

    \EndIf

\EndIf

\If{$y < tam(m) - 1$}

    \If{j.pokenodos[$x$][$y+1$] $\neq$ NULL}     \Comment $O(1)$
        \State $coorConPokemon \gets CrearCoor(x, y)$    \Comment $O(1)$
    \EndIf


    \If{$tam(m) > 1 \land y < tam(m) - 2$}
        \If{j.pokenodos[$x$][$y+2$] $\neq$ NULL}     \Comment $O(1)$
            \State $coorConPokemon \gets CrearCoor(x, y)$    \Comment $O(1)$
        \EndIf


    \EndIf

\EndIf

\If{$x < tam(m) - 1$}
    \If{j.pokenodos[$x+1$][$y$] $\neq$ NULL}     \Comment $O(1)$
        \State $coorConPokemon \gets CrearCoor(x, y)$    \Comment $O(1)$
        \EndIf

    \If{$y > 0$}
        \If{j.pokenodos[$x+1$][$y-1$] $\neq$ NULL}     \Comment $O(1)$
            \State $coorConPokemon \gets CrearCoor(x, y)$    \Comment $O(1)$
        \EndIf
    \EndIf

    \If{$y < tam(m)-1$}
        \If{j.pokenodos[$x+1$][$y+1$] $\neq$ NULL}     \Comment $O(1)$
            \State $coorConPokemon \gets CrearCoor(x, y)$    \Comment $O(1)$
        \EndIf

    \EndIf
\EndIf

\If{$tam(m) > 1 \land x < tam(m) - 2$}
    \If{j.pokenodos[$x+2$][$y$] $\neq$ NULL}     \Comment $O(1)$
        \State $coorConPokemon \gets CrearCoor(x, y)$    \Comment $O(1)$
    \EndIf
\EndIf

\State $res \gets coorConPokemon$

\medskip
\State \underline{Complejidad:} $O(1)$ % por enunciado
\State \underline{Justificacion:} Reviso 13 posiciones O(1)

\end{algorithmic}
%\end{algorithm}



% puedoAgregarPokemon

\begin{algorithm}[H]
{\textbf{iPuedoAgregarPokemon}(\In{c}{coor}, \In{j}{juego}) $\to$ $res$ : bool}
\begin{algorithmic}[1]

\State bool $puedo \gets$ false  \Comment $O(1)$

\If{PosExistente(c, j.mapa)}   \Comment $O(1)$
    \If{$\neg$ HayPokemonCercano(c, j)}   \Comment $O(1)$
        \State $puedo \gets$ true        \Comment $O(1)$
    \EndIf
\EndIf

\State $res \gets puedo$  \Comment $O(1)$

\medskip
\State \underline{Complejidad:} $O(1)$
\State \underline{Justificacion:} Todas las operaciones son O(1)

\end{algorithmic}
\end{algorithm}


% cantMovimientosParaCaptura

\begin{algorithm}[H]
{\textbf{iCantMovimientosParaCaptura}(\In{c}{coor}, \In{j}{juego}) $\to$ $res$ :    nat}
\begin{algorithmic}[1]

\State puntero(pokeStruc) $pokenodo \gets$ j.pokenodos[latitud($c$)][longitud($c$)] \Comment $O(1)$
\State $res \gets$ (*pokenodo).contador  \Comment $O(1)$

\medskip
\State \underline{Complejidad:} $O(1)$
\State \underline{Justificacion:} Todas las operaciones son O(1)

\end{algorithmic}
\end{algorithm}


% expulsados

\begin{algorithm}[H]
{\textbf{iExpulsados}(\In{j}{juego}) $\to$ $res$ : conj(jugador)}
\begin{algorithmic}[1]

\For{$nat i \gets$ 0 to Longitud(j.jugadores) $-$ 1}   \Comment $O(J)$
    \If{j.jugadores[$i$].sanciones $\geq$ 5}   \Comment $O(1)$
        AgregarRapido(res, j.jugadores[$i$].id)  \Comment $O(1)$        
    \EndIf
\EndFor

\medskip
\State \underline{Complejidad:} $O(J)$
\State \underline{Justificacion:} Aplico operaciones que son O(1) la cantidad de veces que ejecuto el ciclo. El ciclo se ejecuta J veces (porque j.jugadores tiene todos los jugadores que fueron agregados) Entonces es O(J), siendo J la cantidad de jugadores que fueron agregados.

\end{algorithmic}
\end{algorithm}


% pokemons

VER TODO EL TEMILLA DEL MULTICONJUNTO... HAY QUE HACERLO?? 

\begin{algorithm}[H]
{\textbf{iPokemons}(\In{e}{jugador}, \In{j}{juego}) $\to$ $res$ : multiconj(jugador)}
\begin{algorithmic}[1]

\State $res \gets $ CrearIt(j.jugadores[e].pokemons) \Comment $O(1)$

\medskip
\State \underline{Complejidad:} $O(1)$
\State \underline{Justificacion:} Devuelvo un iterador

\end{algorithmic}
\end{algorithm}



\begin{algorithm}[H]
{\textbf{icantMismaEspecie}(\In{p}{pkemon}, \In{j}{juego}) $\to$ $res$ : nat}
\begin{algorithmic}[1]

\State $res$ $\gets$ $cantPokemon$.Obtener($p$)  \Comment $O(|p|)$

\medskip
\Statex \underline{Complejidad:} $O(|p|)$
\Statex \underline{Justificaion:} La unica operacion es $O(|p|)$

\end{algorithmic}
\end{algorithm}  
  
\end{Algoritmos}